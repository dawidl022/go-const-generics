\section{Project Plan}

Minor changes were made since the initial project plan. Most notably, arithmetic
operators are to become part of the scope of the formal rules and interpreter,
since they have the potential to play an important role in compelling code
examples, showcasing generic array sizes. It is in the nature of arrays for
there to be a way to loop over them, and operators are primitives that can allow
for that.

The less significant change was a slight offset in schedule of the tasks. This
is not a concern since the project timeline had a large margin to begin with,
with optional ``extension'' tasks (not required for the submission of the
project) making up the final third of the timeline.

Each horizontal section in the chart denotes a milestone, comprised of one or
more tasks. Grey rectangles indicate tasks already completed at the time of
updating the project plan.

\begin{figure}[h]
    \includegraphics[trim={0 15cm 0 0},width=\textwidth]{project-plan.pdf}
    \caption{Timeplan of tasks, milestones and deliverables}
\end{figure}
