\section{Featherweight Generic Go with Arrays addendum}

\subsection{FGGA Typing}
\label{sec:fgg-appx-typing}

The $\methods_\Delta$ auxiliary is now a partial function, not defined on the
\kw{const} type. This is because there is currently never a context in which the
$\methods_\Delta$ would need to be applied to \kw{const}. In a future extension,
type parameters bounded by \kw{const} could be used as expressions (since they will
be instantiated with integer values in user programs), in which case we could
define $\methods_\Delta(\const) = \{\}$ so that they implement the empty
interface. We could then additionally have a rule stating that type parameters
bounded by \kw{const} implement \kw{int}.

Because $\methods_\Delta$ is not defined on \kw{const}, there is an additional
subtyping rule that specifies that type parameters which are bound by
\kw{const}, are also subtypes of \kw{const} (rule $\imp_{const-\text{Param}}$).
All type parameters are subtypes of their bounds, but where the bounds are a
regular interface (i.e. not \kw{const}), this can already be derived via the
$\imp_I$ rule.

The remainder of the T-Type rule checks that the type declaration's type
parameters constraints $\ov{\Phi}$ are well-formed (via T-Formal), and that the
type literal is well-typed in the typing environment formed from $\ov{\Phi}$.

The rule T-Const asserts that \kw{const} type passes the bounds type check in
the T-Formal rule in any environment $\Delta$. The T-Specification rule has been
updated to type check the parameter and return types under the typing
environment $\ov{\Phi}$. Additionally, neither the parameter nor return types
may be \kw{const} types. T-Struct has similarly been updated to check the field
types in the environment $\ov{\Phi}$, and to only permit non-\kw{const} field
types.

The T-Func and T-Func-Arrayset rules have been updated to look up the type
parameter constraints based on the receiver type parameters to construct typing
environments for the updated type-checking rules. Rules T-Int-Literal, T-Var,
T-Call, T-Struct-Literal, T-Field and T-Program are all updated to use types
$\tau$ and typing environments $\Delta$. The supertype of an array literal's
elements is determined by type parameter substitution on the array's element
type.

As before, there are two rules for performing an array index operation, one
where the type of the index is \kw{int}, and one where the type is an integer
literal (rules T-Array-Index and T-Array-Index-Literal).

\subsubsection{Not Referenced predicate}
\label{fgg-appx-typing-not-ref}

The recursive case of $\notref$ exists for named types $t[\ov{\tau}]$, and the
base cases are the remaining type kinds; integer literals cannot be used as a
type name, and neither can the keyword \kw{const}, so there is no possibility of
a self-reference for those types. If a type is named the same as a type
parameter, the type parameter will shadow the type name, and no self-reference
occurs.
% TODO verify this is the case in Go (shadowing of type name with type param,
% both for self and non-self referential)
When checking for self-reference in a named type, first we check whether the
named-type's type name $t$ is equal to any of the types we have already seen
$\ov{t_r}$. Initially, the seen types sequence only contains the type we are
type checking in T-Type. $\notref$ is performed recursively on all type
arguments, as those could also be referencing one of the types from $\ov{t_r}$.
To detect indirect self-references (i.e. circular references), $\notref$ is
applied on all type parameter bounds of the named type's declaration, appending
the type name $t$ to the sequence of seen type names $\ov{t_r}$ to check for
self-references. While theoretically, we could have just a single type $t_r$ to
check for at a time, rather than a sequence, this would lead to problems when
checking types that are self-referential, but do not contain the initial type
$t_r$ in the cycle, as the rule would recurse indefinitely.
