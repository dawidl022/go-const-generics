\begin{abstract}
    Go is a language developed by Google and the open source community and is
    widely used in industry. Many have praised Go for its simplicity, while
    others criticised its lack of certain features, most notably generic
    programming. While the introduction of generics in Go 1.18 addressed the
    largest criticism, there are still areas where a more expressive version of
    the language would benefit users. One such area, which this work focuses on,
    is generically sized (static) arrays. To achieve this, a new kind of type
    parameter is introduced --- one which can be instantiated with
    (compile-time) constant integers \emph{values}. This work formalises a
    subset of Go, and then extends the formalisation with type parameters, that
    include the new numerical type parameters. A translation from the extended
    language to regular Go is also formalised (monomorphisation). To test the
    formalisations, two interpeters (with static type-checking) and a
    monomorphiser have been implemented, along with rigorous testing.
\end{abstract}
