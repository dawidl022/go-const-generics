\section{Featherweight Go with Arrays}
\label{ch:fg}

\emph{Featherweight Go} is a small, functional, Turing-complete subset of the Go
programming language, introduced by \textcite{fg} for the purpose of showing
how generics can be added to the language. This section will
extend \emph{Featherweight Go} with arrays (FGA), as found in Go. In a similar
fashion, only a subset of array features are included to keep things manageable.
In particular, slices are excluded. Since FGA is a subset of Go, many constructs
and syntax that are allowed in Go, are disallowed in FGA to keep the rules
simpler.

A couple of notes on formal notation: a bar above a term or group of terms
denotes a sequence or a rule to be applied to each element in the sequence
\autocite{newOldLang}. A sequence may contain 0 or more instances of the terms.
In actual programs, various delimiters are required between terms in a sequence.
Depending on the construct, this is either a comma or a semicolon
(interchangeable with a newline), but these details are omitted from the formal
rules. The metavariables $e$ and $\ov{e}$ are considered distinct, i.e. unless
stated explicitly, it is not automatically implied that $e \in \ov{e}$. A box
around a syntactical term means it cannot appear in a normal user program but
can be used internally during reduction. Rules or rule fragments appearing in
grey have been taken directly from the original \emph{Featherweight Go}
\autocite{fg} without any modification. The rules (or rule fragments) in black
show the changes introduced when extending the rules to include arrays.

\subsection{FGA Syntax}

An FGA program is defined in a single file. The program starts with the package
name, and since we limit the language to a single file, the package must be
named $\main$, allowing the program to be compiled into an executable in regular
Go. It is then followed by a sequence of 0 or more declarations, and finally the
$\main$ function is required at the end. The $\main$ function consists of a
single expression, and to make it compatible with Go, the expression is assigned
to the blank identifier, denoted by an underscore. This makes it so that the Go
compiler does not reject the program because of an unused expression or
variable.

Two main types of declarations exist: a type declaration and a method
declaration. The latter is subdivided further into a regular method declaration
and an ``array set'' method declaration. The reason for having a special
syntactical form for methods that set a particular element of an array (more
precisely a copy of an array, due to the value semantics of arrays), is that
FGA is a functional subset of Go and does not support variable assignment in
the general sense. By encapsulating the operation of creating an array copy with
a certain element taking on a new value, FGA is able to maintain its functional
property. It also makes the rules simpler since there is no need to introduce
general assignment, yet the ``array set'' syntax is fully compatible with Go and
will behave as expected.

A type declaration is defined as the keyword \type, followed by a declared type
name, followed by a type literal. A ``declared'' type name can be any identifier
as defined by \textcite{spec}, except for the predeclared \kw{int}. In Go, it is
actually legal to redefine predeclared type names, such as the built-in
\kw{int}, however, to keep things simple FGA does not permit this. That is to
say, the only predeclared type name in FGA is \kw{int}, for values of integer
type. No other primitive types (e.g. \kw{bool} or \kw{string}) are defined in
FGA.

A type literal can be one of 3 things: a struct literal, an interface literal,
or an array literal. A struct type literal consists of the keyword \struct,
followed by a sequence of fields inside curly braces. Each field is a pair of a
field identifier and a type name, separated by a space. A type name can be
either an aforementioned declared type name or the predeclared type name
\kw{int}. The rules specify integer literals as valid type names. However, these
are not allowed in user programs and are only used internally by the reduction
and typing rules.

An interface type literal consists of the keyword \interface, followed by a
sequence of method specifications inside curly braces. Each method specification
consists of a method name (an identifier) followed by the method signature. A
method signature consists of a sequence of parameters within parenthesis,
followed by a type name denoting the return type of the method. Each parameter
is a pair of a variable name (an identifier) and a type name, separated by a
space.

An array type literal consists of an integer literal within square brackets,
denoting the length of the array followed by a type name, denoting the type of
the array elements. This restricts array type literals to a single dimension,
i.e. \texttt{[2][3]int} is not allowed. However, in practice, this is not a
concern, as one can define a nested array in two steps, i.e. declaring the inner
array as a type and using that declared type name as the element type of the
outer array.

A method declaration consists of the keyword \func, followed by the receiver in
parentheses, followed by a method specification, followed by the method body.
The method receiver is just a single parameter, where its type name refers to a
declared value type name. A value type name is a type name defined in terms of
either a struct type literal or an array type literal. The method body consists
of the keyword \return, followed by an expression, all enclosed within curly
braces.

An array set method declaration is a special syntactical term in FGA, which is a
valid method in Go. The receiver type refers to a declared array type name,
which is a type name defined in terms of an array literal. The method can have
any name (identifier), and its parameters can be any identifier as long as all
the parameters are named uniquely. However, there are stricter restrictions on
the rest of the construct. The return type must match the receiver type. The
first parameter must have a type of \kw{int}, while the second parameter may
(syntactically) be any type name. The method body is constructed from the
parameter names, where $x$ refers to the array, $x_1$ to the array index, and
$x_2$ to the new value to be assigned to that index. In figure
\ref{fig:fg-syntax}, identical metavariables should be understood as identical
syntactical terms in the rule defining array set methods, e.g. where a concrete
term appears for $x$ in one part of the rule, any other occurrence of the
metavariable $x$ must be instantiated with the same concrete term. The same does
not apply to other rules in the figure (e.g. array index consists of two
potentially distinct subexpressions). This is the only instance where,
syntactically, two statements are allowed within the method body, as there is no
way in Go to perform an array index assignment and return the array in a single
expression.

The expressions that are supported are standard and are described in
appendix \ref{sec:fg-appx-syntax-expr}.

\documentclass[acmsmall,screen]{acmart}
\usepackage{xcolor}

\begin{document}

\newcommand{\squeeze}{\hspace{-1.5em}}
\newcommand{\gap}{\hspace{0.75em}}
\newcommand{\Strut}{\vphantom{\ov{f}}}
\newcommand{\calD}{\mathcal{D}}
\newcommand{\calS}{\mathcal{S}}
\newcommand{\ov}{\overline}
\newcommand{\at}{\mathrel{/}}
\newcommand{\kw}[1]{\texttt{\bf #1}}
\newcommand{\id}[1]{\texttt{#1}}
\newcommand{\meta}{\mathit}
\newcommand{\ok}{~\meta{ok}}
\newcommand{\val}{~\meta{value}}
\newcommand{\fields}{\meta{fields}}
\newcommand{\methods}{\meta{methods}}
\newcommand{\vtype}{\meta{type}}
\newcommand{\mbody}{\meta{body}}
\newcommand{\tdecls}{\meta{tdecls}}
\newcommand{\mdecls}{\meta{mdecls}}
\newcommand{\bounds}{\meta{bounds}}
\newcommand{\indexbounds}{\meta{indexBounds}}
\newcommand{\len}{\meta{lenType}}
\newcommand{\instance}{\meta{instance}}
\newcommand{\notref}{\meta{notReferenced}}
\newcommand{\isconst}{\meta{isConst}}
\newcommand{\isarraysetmethod}{\meta{isArraySetMethod}}
\newcommand{\flatten}{\meta{flatten}}
\newcommand{\unique}{\meta{unique}}
\newcommand{\distinct}{\meta{distinct}}
\newcommand{\elementtype}{\meta{elementType}}
\newcommand{\lentype}{\meta{lenType}}
\newcommand{\typeparams}{\meta{typeParams}}
\newcommand{\type}{\kw{type}}
\newcommand{\struct}{\kw{struct}}
\newcommand{\interface}{\kw{interface}}
\newcommand{\func}{\kw{func}}
\newcommand{\return}{\kw{return}}
\newcommand{\package}{\kw{package}}
\newcommand{\main}{\kw{main}}
\newcommand{\const}{\kw{const}}

\newcommand{\un}{\id{\textunderscore}}
\newcommand{\prog}{\rhd}
\newcommand{\br}[1]{\id{\{}#1\id{\}}}
\newcommand{\lst}[1]{[#1]}
\newcommand{\set}[1]{\{#1\}}
\newcommand{\an}[1]{\langle #1 \rangle}
\newcommand{\imp}{\mathbin{\id{<:}}}
\newcommand{\notimp}{\mathbin{\not\!\!\imp}}
\newcommand{\becomes}{\longrightarrow}
\newcommand{\by}{\mathbin{:=}}
\newcommand{\gray}[1]{{\color{gray}#1}}
\newcommand{\black}[1]{{\color{black}#1}}
\newcommand{\comma}{,\,}
\newcommand{\stoup}{;\,}
\newcommand{\Hole}{\Box}
\newcommand{\trep}{\ensuremath{\mathsf{Rep}}}

\newcommand{\sem}[1]{\llbracket #1 \rrbracket}
\newcommand{\bodies}{\meta{bodies}}
\newcommand{\fix}{\kw{fix}}
\newcommand{\lam}[1]{\lambda #1.\,}

\newcommand{\yields}{\blacktriangleright}
\newcommand{\extensionkw}{closure}
\newcommand{\Sclo}{\textit{S-\extensionkw}}
\newcommand{\Iclo}{\textit{I-\extensionkw}}
\newcommand{\Fclo}{\textit{F-\extensionkw}}
\newcommand{\Mclo}{\textit{M-\extensionkw}}
\newcommand{\Pclo}{\textit{P-\extensionkw}}
\newcommand{\Eclo}{\textit{E-\extensionkw}}

\newcommand{\SExtensionD}[2]{\Sclo(#1)}
\newcommand{\IExtensionD}[2]{\Iclo_{#2}(#1)}
\newcommand{\FExtensionD}[2]{\Fclo(#1)}
\newcommand{\MExtensionD}[2]{\Mclo_{#2}(#1)}
\newcommand{\EExtensionD}[2]{\Eclo(#1)}
\newcommand{\PExtensionD}[2]{\Pclo(#1)}

\newcommand{\derivrule}[3][]{
    \inferrule*[right=#1]
    {#2}
    {#3}
}
\newcommand{\axiomrule}[2][]{
    \inferrule*[right=#1]
    {~}
    {#2}
}

\newcommand{\register}[1]{
    \expandafter\newcommand\csname #1\endcsname{
        \operatorname{#1}
    }
}

\newcommand{\alias}[2]{
    \expandafter\newcommand\csname #1\endcsname{
        \operatorname{#2}
    }
}

\newcommand{\aliasparam}[1]{
    \expandafter\newcommand\csname #1Param\endcsname{
        \operatorname{#1}
    }
}

\newcommand{\ws} {
    % create some whitespace
    \begin{equation*}
    \end{equation*}
}

\newcommand{\golisting}[1] {
    \noindent\begin{minipage}{\linewidth}
        \lstinputlisting[language=Go, tabsize=4, firstline=3]{#1}
    \end{minipage}
}


\begin{figure}
    \gray{
        \begin{minipage}[t]{\textwidth}
            \begin{tabular}[t]{ll}
                Field name                 & $f$                                                       \\
                Method name                & $m$                                                       \\
                Variable name              & $x$                                                       \\
                Structure type name        & $t_S, u_S$                                                \\
                Interface type name        & $t_I, u_I$                                                \\
                \black{Array type name}    & \black{$t_A, u_A$}                                        \\
                \black{Declared type name} & \black{$t_D, u_D$} ::= $t_S \mid t_I~$\black{$\mid t_A$}  \\
                Type name                  & $t, u$ ::= \black{$t_D \mid \kw{int} \mid$ \fbox{$n$}}    \\
                \black{Value type name}    & \black{$t_V, u_V$ ::= $t_S \mid t_A$}                     \\
                Method signature           & $M$ ::= $(\ov{x~t})~t$                                    \\
                Method specification       & $S$ ::= $mM$                                              \\
                Type Literal               & $T$ ::=                                                   \\
                \quad Structure            & \quad $\struct~\br{\ov{f~t}}$                             \\
                \quad Interface            & \quad $\interface~\br{\ov{S}}$                            \\
                \quad \black{Array}        & \quad\black{$[n]t$}                                       \\
                Declaration                & $D$ ::=                                                   \\
                \quad Type declaration     & \quad $\type~\black{t_D}~T$                               \\
                \quad Method declaration   & \quad $\func~(x~$\black{$t_V$}$)~mM~\br{\return~e}$       \\
                \quad Array set method     &                                                           \\
                \quad declaration          & \quad \black{$\func~(x~t_A) ~m(x_1~\kw{int},~x_2~t) ~t_A~
                \br{ x[x_1] = x_2;~\return~x }$}                                                       \\
                Program                    & $P$ ::= $\package~\main;~\ov{D}~\func~\main()~\br{\un=e}$
            \end{tabular}
        \end{minipage}
        \hspace{-0.5\textwidth}
        \begin{minipage}[t]{0.4\textwidth}
            \begin{tabular}[t]{ll}
                Expression                     & $d, e$ ::=                     \\
                \quad \black{Integer literal } & \quad\black{$n$}               \\
                \quad Variable                 & \quad $x$                      \\
                \quad Method call              & \quad $e.m(\ov{e})$            \\
                \quad \black{Value literal}    & \quad $\black{t_V}\br{\ov{e}}$ \\
                \quad Select                   & \quad $e.f$                    \\
                \quad \black{Array index}      & \quad\black{$e$[$e$]}
            \end{tabular}
        \end{minipage}
    }
    \caption{FG syntax with arrays}
    \label{fig:fg-syntax}
\end{figure}
\end{document}


\subsection{FGA Reduction}

Figure \ref{fig:fg-reduction} describes the small-step operational semantics of
FGA, with auxiliary functions defined in figure \ref{fig:fg-reduction-aux}. A
value is a term that cannot be reduced further, i.e. is the final result of
computation. In FGA, a value is either an integer literal or a value literal
(struct or array), whose elements are all values themselves. An expression $d$
reduces to expression $e$ in a single step (denoted as $d \to e$), if one of the
reduction rule templates can be instantiated with the metaexpression $d \to e$
by pattern matching.

The auxiliary predicate $\isarraysetmethod$ returns true if and only if the
method given by the type name and method name exists in the sequence of
declarations and is syntactically an array set method, as defined in figure
\ref{fig:fg-syntax}. The rule R-Array-Set says that given a method call
expression, where the receiver is an array literal value, the first argument is
an integer literal, and the second argument is a value, evaluates to an array
value literal, with the same elements but for the $n$\textsuperscript{th} index
which is replaced with the value in the second argument of the method call. This
rule can be applied if and only if the method in question is an array set method
and the integer literal $n$ is within the bounds of the array.

The remaining reduction rules are standard and are described in
appendix \ref{sec:fg-appx-reduction}. An example of the reduction
rules being applied to a very simple FGA program can be found in appendix
\ref{sec:fg-derivation-example}.

\begin{figure}
    \begin{mathpar}
        \gray{
            \inferrule
            {(\type~t_S~\struct\br{\ov{f~t}}) \in \ov{D}}
            {\fields(t_S) = \ov{f~t}}

            \inferrule
            {(\func~(x~\black{t_V})~m(\ov{x~t})~t~\br{\return~e}) \in \ov{D}}
            {\mbody(\black{t_V}.m) = (x:\black{t_V},\ov{x:t}).e}
        }

        \inferrule
        {
        (\type~t_A~ [n]t) \in \ov{D}
        }
        { \{ i \in \mathbb{Z} \mid 0 \le i < n \} = \indexbounds(t_A)}

        \inferrule
        {(\func~(x~t_A) ~m(x_1~\kw{int},~x_2~t) ~t_A~
            \br{ x[x_1] = x_2;~\return~x }) \in \ov{D}}
        {\isarraysetmethod(t_A.m)}
    \end{mathpar}
    \caption{FGA auxiliary functions for reduction rules}
    \label{fig:fg-reduction-aux}
\end{figure}


\begin{figure}
    \begin{center}
        Value \qquad $v$ ::= $t_V\br{\ov{v}} \mid n$
    \end{center}

    \begin{center}
        \gray{
            \begin{minipage}{0.5\textwidth}
                \begin{tabular}{ll}
                    Evaluation context          & $E$ ::=                      \\
                    \quad Hole                  & \quad $\Hole$                \\
                    \quad Method call receiver  & \quad $E.m(\ov{e})$          \\
                    \quad Method call arguments & \quad $v.m(\ov{v},E,\ov{e})$
                \end{tabular}
            \end{minipage}
            \begin{minipage}{0.4\textwidth}
                \begin{tabular}{ll}
                    ~                                                                      \\
                    \quad Value literal          & \quad $\black{t_V}\br{\ov{v},E,\ov{e}}$ \\
                    \quad Select                 & \quad $E.f$                             \\
                    \quad \black{Index receiver} & \quad \black{$E[e]$}                    \\
                    \quad \black{Index argument} & \quad \black{$t_A\br{\ov{v}}[E]$}
                \end{tabular}
            \end{minipage}
        }
    \end{center}

    Reduction \hfill \fbox{$d \becomes e$}
    \begin{mathpar}

        \gray{
            \inferrule[r-field]
            { (\ov{f~t}) = \fields(t_S) }
            { t_S\br{\ov{v}}.f_i \becomes v_i }
        }

        \inferrule[r-index]
        {
            n \in \indexbounds(t_A)
        }
        { t_A\br{\ov{v}}[n] \becomes v_n }

        \gray{
            \inferrule[r-call]
            { (x : \black{t_V},\ov{x: t}).e = \mbody(\vtype(v).m) }
            { v.m(\ov{v}) \becomes e[x \by v, \ov{x \by v}] }
        }

        % potential ambiguity with regular r-call?
        % no, because isarraysetmethod(t_A.m) and body(t_A.m) are mutually exclusive
        \inferrule[r-array-set]
        {
            n \in \indexbounds(t_A) \\
            \isarraysetmethod(t_A.m) \\
        }
        { t_A\br{\ov{v}}.m(n, v) \becomes t_A\br{\ov{v}}[n := v]}

        \gray{
            \inferrule[r-context]
            { d \becomes e }
            { E[d] \becomes E[e] }
        }

    \end{mathpar}
    \caption{FGA reduction rules}
    \label{fig:fg-reduction}
\end{figure}


\subsection{FGA Typing}

Figures \ref{fig:fg-typing-1} and \ref{fig:fg-typing-2} describe the typing
rules in FGA, with auxiliary functions defined in figure
\ref{fig:fg-typing-aux}. A term is considered well-typed (or well-formed) when
there exists a typing rule that the term can pattern match against, followed by
the \emph{ok} symbol. The $\imp$ symbol denotes a subtyping or ``implements''
relation between two types. The metaexpression $\Gamma \vdash e : t$ denotes a
3-tuple relation where an expression $e$ is of type $t$ in the environment
$\Gamma$, which maps variables $x$ to types $t$.

The first set of rules defines the subtyping relation between types. A type is
said to be a subtype of another type in FGA when it \emph{implements} the latter
type. Any type implements itself (rules $\imp_V$ $\imp_{int}$, $\imp_n$, and
implicitly $\imp_I$). The $\methods$ helper function returns the set of method
specifications (later referred to as a ``method set'') defined on a given type
(including array set methods). A type implements an interface if the type's
method set is a superset of the interface's method set. By this definition,
interfaces can also implement other interfaces (and themselves).

A type, identified by its name, is considered well-formed when it's either the
predeclared \kw{int} type (rule T-Int-Type), or it is the type name of one of
the type declarations in the program (rule T-Named). An array set method is
well-formed if the 2nd parameter's type is a subtype of the receiver type's
element type and the receiver type is itself a well-formed array type (rule
T-Func-Arrayset).

An array literal expression is well-formed and of the array type that was
instantiated, if the array type is well-formed, the number of elements matches
the size of the array (note that unlike in Go, zero values are not defined in
FGA, as such, all elements of an array must be explicitly instantiated upfront),
and all elements' types are subtypes of the array's element type (rule
T-Array-Literal).

For reasons that will be discussed when type parameters are introduced, integer
literals are of their own types, e.g. the expression $1$ is of type $1$ (rule
T-Int-Literal). Integer literal types are subtypes of the \kw{int} type, e.g. $1
    \imp \kw{int}$ (rule $\imp_{int-n}$). Go does not make such a distinction (i.e.
each integer literal being its own type), however, integer literals (constants)
in Go may be ``untyped'', which can be used as subtypes of other numerical types
(e.g. \kw{int}, \kw{byte}, \kw{int16} etc.). In FGA, integer literals are
treated similarly to untyped integer constants in Go \autocite{spec}.

Because of the distinction between constant (literal) and non-constant integer
types, there are two typing rules for an array index expression. In both cases,
the array expression must be well-typed and of an array type, and the array
index expression's type is of the element type of the array. When the index
expression is of the non-constant \kw{int} type, no other checks are performed
(statically) (rule T-Array-Index). However, when the index expression is of an
integer literal type, an index bounds check is also statically performed via the
type system (rule T-Array-Index-Literal).

The remainder of the rules are described in appendix \ref{sec:fg-appx-typing}.
An example of the typing rules being applied to a very simple FGA program can be
found in appendix \ref{sec:fg-typing-derivation-example}.

\documentclass[acmsmall,screen]{acmart}
\usepackage{mathpartir}

\begin{document}

\newcommand{\squeeze}{\hspace{-1.5em}}
\newcommand{\gap}{\hspace{0.75em}}
\newcommand{\Strut}{\vphantom{\ov{f}}}
\newcommand{\calD}{\mathcal{D}}
\newcommand{\calS}{\mathcal{S}}
\newcommand{\ov}{\overline}
\newcommand{\at}{\mathrel{/}}
\newcommand{\kw}[1]{\texttt{\bf #1}}
\newcommand{\id}[1]{\texttt{#1}}
\newcommand{\meta}{\mathit}
\newcommand{\ok}{~\meta{ok}}
\newcommand{\val}{~\meta{value}}
\newcommand{\fields}{\meta{fields}}
\newcommand{\methods}{\meta{methods}}
\newcommand{\vtype}{\meta{type}}
\newcommand{\mbody}{\meta{body}}
\newcommand{\tdecls}{\meta{tdecls}}
\newcommand{\mdecls}{\meta{mdecls}}
\newcommand{\bounds}{\meta{bounds}}
\newcommand{\indexbounds}{\meta{indexBounds}}
\newcommand{\len}{\meta{lenType}}
\newcommand{\instance}{\meta{instance}}
\newcommand{\notref}{\meta{notReferenced}}
\newcommand{\isconst}{\meta{isConst}}
\newcommand{\isarraysetmethod}{\meta{isArraySetMethod}}
\newcommand{\flatten}{\meta{flatten}}
\newcommand{\unique}{\meta{unique}}
\newcommand{\distinct}{\meta{distinct}}
\newcommand{\elementtype}{\meta{elementType}}
\newcommand{\lentype}{\meta{lenType}}
\newcommand{\typeparams}{\meta{typeParams}}
\newcommand{\type}{\kw{type}}
\newcommand{\struct}{\kw{struct}}
\newcommand{\interface}{\kw{interface}}
\newcommand{\func}{\kw{func}}
\newcommand{\return}{\kw{return}}
\newcommand{\package}{\kw{package}}
\newcommand{\main}{\kw{main}}
\newcommand{\const}{\kw{const}}

\newcommand{\un}{\id{\textunderscore}}
\newcommand{\prog}{\rhd}
\newcommand{\br}[1]{\id{\{}#1\id{\}}}
\newcommand{\lst}[1]{[#1]}
\newcommand{\set}[1]{\{#1\}}
\newcommand{\an}[1]{\langle #1 \rangle}
\newcommand{\imp}{\mathbin{\id{<:}}}
\newcommand{\notimp}{\mathbin{\not\!\!\imp}}
\newcommand{\becomes}{\longrightarrow}
\newcommand{\by}{\mathbin{:=}}
\newcommand{\gray}[1]{{\color{gray}#1}}
\newcommand{\black}[1]{{\color{black}#1}}
\newcommand{\comma}{,\,}
\newcommand{\stoup}{;\,}
\newcommand{\Hole}{\Box}
\newcommand{\trep}{\ensuremath{\mathsf{Rep}}}

\newcommand{\sem}[1]{\llbracket #1 \rrbracket}
\newcommand{\bodies}{\meta{bodies}}
\newcommand{\fix}{\kw{fix}}
\newcommand{\lam}[1]{\lambda #1.\,}

\newcommand{\yields}{\blacktriangleright}
\newcommand{\extensionkw}{closure}
\newcommand{\Sclo}{\textit{S-\extensionkw}}
\newcommand{\Iclo}{\textit{I-\extensionkw}}
\newcommand{\Fclo}{\textit{F-\extensionkw}}
\newcommand{\Mclo}{\textit{M-\extensionkw}}
\newcommand{\Pclo}{\textit{P-\extensionkw}}
\newcommand{\Eclo}{\textit{E-\extensionkw}}

\newcommand{\SExtensionD}[2]{\Sclo(#1)}
\newcommand{\IExtensionD}[2]{\Iclo_{#2}(#1)}
\newcommand{\FExtensionD}[2]{\Fclo(#1)}
\newcommand{\MExtensionD}[2]{\Mclo_{#2}(#1)}
\newcommand{\EExtensionD}[2]{\Eclo(#1)}
\newcommand{\PExtensionD}[2]{\Pclo(#1)}

\newcommand{\derivrule}[3][]{
    \inferrule*[right=#1]
    {#2}
    {#3}
}
\newcommand{\axiomrule}[2][]{
    \inferrule*[right=#1]
    {~}
    {#2}
}

\newcommand{\register}[1]{
    \expandafter\newcommand\csname #1\endcsname{
        \operatorname{#1}
    }
}

\newcommand{\alias}[2]{
    \expandafter\newcommand\csname #1\endcsname{
        \operatorname{#2}
    }
}

\newcommand{\aliasparam}[1]{
    \expandafter\newcommand\csname #1Param\endcsname{
        \operatorname{#1}
    }
}

\newcommand{\ws} {
    % create some whitespace
    \begin{equation*}
    \end{equation*}
}

\newcommand{\golisting}[1] {
    \noindent\begin{minipage}{\linewidth}
        \lstinputlisting[language=Go, tabsize=4, firstline=3]{#1}
    \end{minipage}
}


\begin{figure}
    \begin{mathpar}
        \inferrule
        {(\type~t_A~ [n]t) \in \ov{D}}
        {t = \elementtype(t_A)}

        \inferrule
        {(\type~t_A~ [n]t) \in \ov{D}}
        {n = \len(t_A)}

        \inferrule
        {~}
        {\methods(\kw{int}) = \{\}}

        \inferrule
        {~}
        {\methods(n) = \{\}}

        \gray{
            \inferrule
            {}
            {\methods(\black{t_V}) = \{
                mM \mid
                (\func~(x~\black{t_V})~mM~\br{\return~e}) \in \ov{D}
                \}\\
                \black{\cup~\{
                    m(x_1~\kw{int},~x_2~t) ~t_V~ \mid
                    (\func~(x~t_V) ~m(x_1~\kw{int},~x_2~t) ~t_V~
                    \br{ x[x_1] = x_2;~\return~x }) \in \ov{D}
                    \}}}
        }

        \gray{
            \inferrule
            {\type~t_I~\interface\br{\ov{S}} \in \ov{D}}
            {\methods(t_I) = \ov{S}}

            \inferrule
            {(\type~t_S~\struct\br{\ov{f~t}}) \in \ov{D}}
            {\fields(t_S) = \ov{f~t}}

            \inferrule
            {\text{$mM_1, mM_2 \in \ov{S}$ implies $M_1 = M_2$}}
            {\unique(\ov{S})}
        }

        \gray{
            \inferrule
            {}
            {\tdecls(\ov{D}) = \lst{\black{t_D} \mid (\type~\black{t_D}~T) \in \ov{D}}}

            \inferrule
            {}
            {\mdecls(\ov{D}) = \lst{\black{t_V}.m \mid
                    (\func~(x~\black{t_V})~mM~\br{\return~e}) \in \ov{D}}}
        }
    \end{mathpar}
    \caption{FG auxiliary functions for array typing}
\end{figure}

\end{document}

\begin{figure}
    Implements, well-formed type
    \hfill \fbox{$t \imp u$} \qquad \fbox{$t \ok$}
    \begin{mathpar}

        \inferrule[$\imp_V$]
        {~}
        {t_V \imp t_V}

        \inferrule[$\imp_{int}$]
        {~}
        {\kw{int} \imp \kw{int}}

        \inferrule[$\imp_{n}$]
        {~}
        {n \imp n}

        \inferrule[$\imp_{int-n}$]
        {~}
        { n \imp \kw{int} }

        \gray{
            \inferrule[$\imp_I$]
            {
                \methods(t) \supseteq \methods(t_I)
            }
            { t \imp t_I }
        }

        \inferrule[t-int-type]
        {~}
        {\kw{int} \ok}

        \gray{
            \inferrule[t-named]
            {
                (\type~t~T) \in \ov{D}
            }
            { t \ok }
        }

    \end{mathpar}

    Well-formed method specifications and type literals
    \hfill \fbox {$S \ok$} \qquad \fbox{$T \ok$}
    \begin{mathpar}

        \inferrule[t-array]
        {
            n \ge 0\\
            t \ok
        }
        {[n]t \ok}

        \gray{
            \inferrule[t-specification]
            {
                \distinct(\ov{x}) \\
                \ov{t \ok} \\
                t \ok
            }
            { m(\ov{x~t})~t \ok }

            \inferrule[t-struct]
            {
                \distinct(\ov{f}) \\
                \ov{t \ok}
            }
            { \struct~\br{\ov{f~t}} \ok }

            \inferrule[t-interface]
            {
                \unique(\ov{S}) \\
                \ov{S \ok}
            }
            { \interface~\br{\ov{S}} \ok }
        }

    \end{mathpar}

    Well-formed declarations \hfill \fbox{$D \ok$}
    \begin{mathpar}

        \gray{
            \inferrule[t-type]
            {
                T \ok \\
                \black{\notref(t,~T)}
            }
            { \type~t~T \ok }

            \inferrule[t-func]
            {
                \distinct(x, \ov{x}) \\\\
                \black{t_V} \ok \\
                \black{m(\ov{x~t})~u \ok} \\
                x : \black{t_V} \comma \ov{x : t} \vdash e : t \\
                t \imp u
            }
            { \func~(x~\black{t_V})~m(\ov{x~t})~u~\br{\return~e} \ok }
        }

        \inferrule[t-func-arrayset]
        {
            u = \elementtype(t_A) \\
            t <: u \\
            t_A \ok
        }
        {
            \func~(x~t_A) ~m(x_1~\kw{int},~x_2~t) ~t_A~
            \br{ x[x_1] = x_2;~\return~x }
        }

    \end{mathpar}

    \caption{FGA typing rules (1 of 2)}
    \label{fig:fg-typing-1}
\end{figure}

\begin{figure}

    Expressions \hfill \fbox{$\Gamma \vdash e : t$}
    \begin{mathpar}

        \gray{
            \inferrule[t-var]
            { (x : t) \in \Gamma }
            { \Gamma \vdash x : t }

            \inferrule[t-call]
            {
                \Gamma \vdash e : t \\
                \Gamma \vdash \ov{e : t} \\
                (m(\ov{x~u})~u) \in \methods(t) \\
                \ov{t \imp u}
            }
            { \Gamma \vdash e.m(\ov{e}) : u }
        }

        \gray{
            \inferrule[t-struct-literal]
            {
                t_S \ok \\
                \Gamma \vdash \ov{e : t} \\
                (\ov{f~u}) = \fields(t_S) \\
                \ov{t \imp u}
            }
            { \Gamma \vdash t_S\br{\ov{e}} : t_S }

            \inferrule[t-field]
            {
                \Gamma \vdash e : t_S \\
                (\ov{f~u}) = \fields(t_S)
            }
            { \Gamma \vdash e.f_i : u_i }
        }

        \inferrule[t-array-literal]
        {
            t_A \ok \\
            |\ov{e}| = \len(t_A)\\
            \Gamma \vdash \ov{e : t} \\
            u = \elementtype(t_A) \\
            \ov{t <: u}
        }
        { \Gamma \vdash t_A\br{\ov{e}} : t_A }

        \inferrule[t-int-literal]
        {~}
        { \Gamma \vdash n : n }

        \inferrule[t-array-index]
        {
        \Gamma \vdash e_1 : t_A \\
        \Gamma \vdash e_2 : \kw{int} \\
        t = \elementtype(t_A)
        }
        { \Gamma \vdash e_1[e_2] : t }

        \inferrule[t-array-index-literal]
        {
        \Gamma \vdash e_1 : t_A \\
        \Gamma \vdash e_2 : n \\
        0 \le n < \len(t_A) \\
        t = \elementtype(t_A)
        }
        { \Gamma \vdash e_1[e_2] : t }

    \end{mathpar}

    Programs \hfill \fbox{$P \ok$}
    \begin{mathpar}
        \gray{
            \inferrule[t-prog]
            {
                \distinct(\tdecls(\ov{D})\black{, \kw{int}}) \\
                \distinct(\mdecls(\ov{D})) \\
                \ov{D \ok} \\
                \emptyset \vdash e : t
            }
            { \package~\main;~\ov{D}~\func~\main()~\br{\un=e} \ok }
        }
    \end{mathpar}

    \caption{FGA typing rules (2 of 2)}
    \label{fig:fg-typing-2}
\end{figure}


\subsection{FGA Properties}

An array index expression or an array-set method call expression panics if they
contain an array type $t_A$, and an array index $n$, where $n \notin
    \indexbounds(t_A)$. An expression $e$ panics if $e = E[d]$, where $E$ is any
evaluation context, and $d$ is an expression that panics.

The progress and preservation properties covered in \emph{Featherweight Go}
apply to FGA and are defined as follows \autocite{fg}:

\begin{theorem}[Preservation]
    If\/ $\emptyset \vdash d : u$ and $d \becomes e$
    then\/ $\emptyset \vdash e : t$ for some $t$
    with\/ $t \imp u$.
\end{theorem}

\begin{theorem}[Progress]
    If\/ $\emptyset \vdash d:u$ then
    either\/ $d$ is a value,
    $d \becomes e$ for some $e$,
    or\/ $d$ panics.
\end{theorem}

% The typing environment is the emtpy set because we always reduce in the empty
% environment, i.e. in the top-level program expression
