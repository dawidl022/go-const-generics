\section{Introduction}

Generics have been introduced in the Go programming language following the
theoretical work by \cite{fg} in \emph{Featherweight Go}. The \emph{Type
Parameters Proposal} lists several generic programming constructs not supported
by the initial implementation of generics (as of Go 1.18). Among them is "no
parameterization on non-type values such as constants."
\autocite{genericsProposal} The most notable use case of such type parameters
would be for arrays. In Go, the size of an array is part of its type
\autocite{spec}. As such, if the programmer wishes to write a function that
operates on arrays or a data type that contains arrays, it is necessary to
hard-code the size of the operated/contained array. This imposes a limitation on
what abstraction may be introduced where arrays are concerned. Extending the
generic type system in Go to support constants as type parameters aims to
resolve this issue.

\subsection{Background}

Arrays are a primitive data structure found in many programming languages.
However, various languages treat arrays differently...
% TODO talk about first-class vs non first-class arrays, static and only dynamic
% arrays, and whether the size is part of the arrays type or not, and which
% languages are similar to what Go does. Mention that Go has slices for most
% practical use cases. Mention however, the use cases where arrays can come
% handy in Go.

% TODO explain how this problem is tackled in other languages

% C: Arrays are just pointers to the first element (no abstraction)
% "The C programming language"; https://stefansf.de/post/arrays-are-second-class-citizens/
% Python, Ruby, JS: Only dynamic arrays - abstraction on language level
% Java: Array size is static, but not part of its type
% Go, Rust (+ other langs??): Array size is part of type

\subsection{Problem Statement}

% TODO something along the lines of "How can the type system be extended
% to support generically sized arrays in an efficient way, compatible with the
% existing implementation of generics in Go.

\subsection{Aim}

% TODO this will be a formal rules for arrays, their generic counterparts,
% based on Featherweight Go.

\subsection{Objectives}

% Individual deliverable - formal rules + verifying design with interpreter +
% monomorphiser

\subsection{Research Questions}

How have other languages where the size of an array is part of its type tackled
this issue, if at all?

\subsection{Report Structure}
