\section{Introduction}

Generics have been introduced in the Go programming language following the
theoretical work by \cite{fg} in \emph{Featherweight Go}. The \emph{Type
Parameters Proposal} lists several generic programming constructs not supported
by the initial implementation of generics (as of Go 1.18). Among them is "no
parameterization on non-type values such as constants."
\autocite{genericsProposal} The most notable use case of such type parameters
would be for arrays. In Go, the size of an array is part of its type
\autocite{spec}. As such, if the programmer wishes to write a function that
operates on arrays or a data type that contains arrays, it is necessary to
hard-code the size of the operated/contained array. This imposes a limitation on
what abstraction may be introduced where arrays are concerned. Extending the
generic type system in Go to support constants as type parameters aims to
resolve this issue.
