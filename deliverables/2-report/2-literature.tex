\section{Background Research}

\subsection{Generic array sizes in other languages}

Languages where the size of an array is not part of its type (e.g. Java, C\#),
automatically abstract over arrays of all sizes, as there is no way of
expressing that a variable holds an array of a specific size.

\subsubsection{Const generics in Rust}

\subsection{Programming language theory}

% TODO review TAPL here

\subsection{Featherweight Go}

For many years, the biggest criticism against the Go language was the lack of
generics \autocites{survey2021}{survey2020}{survey2019}. The Go team recognised
the importance of solving this problem ``right'' and consequently reached out to
the world of academics for a collaboration, the result of which was a paper
named \emph{Featherweight Go} \autocite{fg}. The work was inspired by
\emph{Featherweight Java}, an effort two decades prior aimed at formalising Java
and its generic extension \autocite{fj}.
The common theme in the two papers is the reduction of the programming language
into a small core subset, making it easier to prove properties about the
language and, subsequently, any proposed extensions.
% TODO why is it important to proove properties of a language?
Both papers also extended their language subsets into variants with generics
(parametric polymorphism), and showed how the generic variant can be translated
into the non-generic base language.

% TODO talk about contracts and then how Featherweight Go simplified it and
% convinced the team
