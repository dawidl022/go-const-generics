\section{Arrays by example}
\label{ch:examples}

Most of the time, Go programmers will use slices over arrays due to their
dynamic nature. However, when the size of a collection of elements can be
determined at compile time, there are certain benefits to using arrays. This
section intends to identify use cases where using arrays in Go programs may be
more beneficial than using slices, and what are the benefits and the performance
implications.

\subsection{Gentle introduction}

The first function we'll examine creates a reversed copy of an array/slice.
Below is a side-by-side view of the two functions: the first operating on
arrays, and the second operating on slices.

\golisting{../../projects/go-generic-array-sizes/benchmarks/benchmarks/reversed/arrays.go}
\golisting{../../projects/go-generic-array-sizes/benchmarks/benchmarks/reversed/slices.go}

The slice function is almost identical to the array function, except that we
need to explicitly allocate a new slice for the result. Arrays are value types,
so simply passing an array into a function (and similarly returning an array
from a function) creates a copy. For relatively small arrays (8MB in size or
less), the memory for the array copy is allocated on the stack. For slices,
however, even for the smallest slices (of length 1 or more), the
\texttt{reversed} function allocates the new slice on the heap. This operation
is more expensive, and so, as the benchmarks show, the array variant of the
\texttt{reversed} functions performs on average around 50\% faster for arrays of
size 8MB or less. For very small arrays (64 \texttt{int}s or less), the
performance benefits or arrays for this operation are even more apparent. Once
the array reaches 16MB or more, the memory for the copy gets allocated onto the
heap and becomes slower than using slices. The benchmarks are limited to 256MB,
since at array sizes of 512MB the compiler rejects the program since it could
potentially use up more than 1GB of stack space (we need to multiply 512MB by
two since we are creating a copy of the array). So if collections of such large
sizes are necessary, slices are the only option. For smaller sizes, however,
arrays perform better.

The benchmarks were run on a single core, using \texttt{GOMAXPROCS=1}, and the
results of the functions were written to global variables, in an attempt to
prevent the compiler optimiser from eliminating the benchmarked code
\autocite{benchPits}.

\begin{figure}
	\begin{tikzpicture}
    
    \pgfplotsset{set layers}
    
    \begin{loglogaxis}[
            scale only axis,
            height=12cm,
            xlabel={Number of elements},
            ylabel={Runtime (ns/op)},
            legend pos=north west,
            legend entries={
                BenchmarkReversedArray,BenchmarkReversedSlice,
            },
            xmin=2,
            xmax=33554432, axis y line*=left,
            log basis x={2},
            log basis y={2},
            xmajorgrids=true,
            ymajorgrids=true,
            grid style=dashed,
        ]

        
        \addplot table {../../projects/go-generic-array-sizes/benchmarks/runner/outputs/reversed/BenchmarkReversedArray.dat};
        
        \addplot table {../../projects/go-generic-array-sizes/benchmarks/runner/outputs/reversed/BenchmarkReversedSlice.dat};
        
    \end{loglogaxis}
    
    \begin{semilogxaxis}[
            scale only axis,
            xmin=2,
            xmax=33554432,
            height=12cm,
            axis y line*=right,
            axis x line=none,
            ylabel=Relative speedup,
            log basis x={2},
            extra y ticks = 0,
            extra y tick labels = ,
            extra y tick style  = { grid = major },
        ]
        \addplot [gray] table {../../projects/go-generic-array-sizes/benchmarks/runner/outputs/reversed/ArraySlice/speedup.dat};
    \end{semilogxaxis}
    
\end{tikzpicture}

% GOOS: darwin
% GOARCH: arm64

	\caption{Comparison of \texttt{reversed} function benchmarks captured on
		Apple M1 Pro}
\end{figure}

The literal \texttt{N} in the array type refers to a constant, defined elsewhere
in the program. The function signature only accepts arrays of length \texttt{N},
despite the function body being generic enough that it could work on arrays of
any length. To illustrate the point, the next example will inline \texttt{N}
with an integer literal. The function also happens to only accept \texttt{int}
arrays, but this can easily be fixed using generic type parameters, introduced
in Go 1.18 (unfortunately, at the cost of some performance):

% TODO benchmark reversing in place

% TODO suggest benchmarks for more use-cases as part of further work, mention
% challenges of benchmarking in Go due to compiler optimisations of contrived
% usages. It may make more sense to benchmark components in context of real
% system, or be able to read Go's assembly code to see for oneself's whether
% the compiler optimised away their benchmarked code, or why certain code is
% faster than other code

\golisting{./examples/reversed/generic_type.go}

The implication of this is that for each different array length we want
to use the \texttt{reversed} operation on, we would have to write a new function
with the exact same code, or use a code generation tool to do this for us.

As of Go 1.18, there is a workaround that partially solves the above problem.
Interface types are now defined in terms of the more general notion of type
sets, as opposed to method sets pre Go 1.18 \autocites{spec}{specPre1.18}.
``General interfaces'' were introduced, that can only be (as of Go 1.21) used as
type parameter constraints, and among its features is the ability to specify a
union of types. With this, we can define an interface in terms of the union of
differently sized arrays that we wish to use with our \texttt{reversed} function
\autocite{goArrayProposal}:

\golisting{./examples/reversed/union_type.go}

This approach still has limitations. Apart from the obvious burden of having to
update the \texttt{array} interface every time we use an array of a new size,
this model breaks down as soon as we wish to expose such a function as part of a
public API. There is no way of knowing ahead of time what array sizes a user may
wish to use, and enumerating them all is infeasible. Ideally, we'd want a way to
abstract over arrays of any size.

\subsection{Array semantics}

We've looked at an example where using arrays is faster than slices. In general,
it is difficult to reliably build array-based data structures that offer better
performance than slices, since all the operations need to be handwritten and
optimised. One rule of thumb is that heap allocations are expensive, and it is
easier to avoid heap allocations (e.g. when making copies) for arrays than for
slices. Figures \ref{lab:arr-semant} and \ref{lab:slice-semant} compare the
semantics of arrays versus slices, and an accompanying description can be found
in appendix \ref{ch:arr-semantics-appx}.

\begin{figure}
	\golisting{./examples/semantics/arrays.go}
	\caption{Array semantics allow for trivial deep-copying and comparison}
	\label{lab:arr-semant}
\end{figure}

\begin{figure}
	\golisting{./examples/semantics/slices.go}
	\caption{Slices hold references to underlying data, and cannot be compared}
	\label{lab:slice-semant}
\end{figure}

\subsubsection{Limitations}

The drawback of defining array-based data structures and operations on them is
that the array sizes must be fixed at compile time. By making array sizes
generic, we can parameterize array-based data structures over many sizes known
at compile time, making them much more versatile. The following subsection
outlines some examples of array-based data structures, and then we proceed to
describe how numerical type parameters can resolve this limitation and propose a
design for them.

\subsection{Data structure examples}

Array-based data structures can enjoy the benefits of being easily copyable and
comparable, even as part of nested data structures. The figures present two
popular data types: a ``resizable'' array with a fixed maximum capacity, and a
double-ended queue with an underlying fixed-size ring buffer (circular
array)\footnote{The ``wasted'' slot strategy was used to differentiate between
	``empty'' and ``full'' states \autocite{ringBuffer}}.

The two example data structures can be fully implemented in FGA, with minor
adjustments. Most notably, FGA does not support mutation via pointers (or
pointers at all), so functions that ``mutate'' the data structure return updated
copies in FGA instead. Additionally, user-raised panics (via calls to the Go
\texttt{panic}) function are not supported, so instead some fallback (such as
returning the zero-value, or performing a no-op) is chosen instead.

\begin{figure}
	\noindent\begin{minipage}[t]{.45\linewidth}
		\lstinputlisting[language=Go, tabsize=4, firstline=3]
		{../../projects/go-generic-array-sizes/benchmarks/benchmarks/semidynamicarray/array.go}
	\end{minipage}
	\hfill
	\noindent\begin{minipage}[t]{.45\linewidth}
		\lstinputlisting[language=Go, tabsize=4, firstline=3, lastline=35]
		{../../projects/go-generic-array-sizes/interpreters/fg/examples/semidynamicarray/array.go}
	\end{minipage}
	\caption{ ``Resizeable'' array with a fixed underlying buffer. Left-hand side
		shows idiomatic Go implementation, while right-hand side shows FGA
		compatible implementation. Full code definitions (including
		\texttt{Nat}) can be found in \hyperref[sec:fg-resizable-array-code]{the
			appendix}.}
\end{figure}

\begin{figure}
	\golisting{../../projects/go-generic-array-sizes/benchmarks/benchmarks/ringbuffer/deque.go}
	\caption{Double-ended queue (deque) with a fixed underlying ring buffer implemented in idiomatic Go}
\end{figure}

\begin{figure}
	\lstinputlisting[language=Go, tabsize=4, firstline=3, lastline=48, basicstyle=\small\ttfamily, emptylines=0]
	{../../projects/go-generic-array-sizes/interpreters/fg/examples/ringbuffer/deque.go}
	\caption{Double-ended queue (deque) with a fixed underlying ring buffer
		implemented in FGA. Full code definitions can be
		found in appendix \ref{sec:fg-deque-code}.}
\end{figure}
