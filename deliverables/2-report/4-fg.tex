\section{Featherweight Go}

\emph{Featherweight Go} is a small, functional, Turing-complete subset of the Go
programming language, introduced by \citeauthor{fg} for the purpose of showing
how generics can be added to the language (\citeyear{fg}). This section will
extend \emph{Featherweight Go} with arrays, as found in Go. In a similar
fashion, only a subset of array features are included to keep things manageable.
In particular, slices are excluded. Since FG is a subset of Go, many constructs
and syntax that are allowed in Go, are disallowed in FG to keep the rules
simpler.

A couple of notes on formal notation: a bar above a term or group of terms
denotes a sequence or a rule to be applied to each element in the sequence. A
sequence may contain 0 or more instances of the terms. In actual programs,
various delimiters are required between terms in a sequence. Depending on the
construct, this is either a comma or a semicolon (interchangeable with a
newline), but these details are omitted from the formal rules. The metavariables
$e$ and $\ov{e}$ are considered distinct, i.e. unless stated explicitly, it is
not automatically implied that $e \in \ov{e}$. A box around a syntactical term
means it cannot appear in a normal user program, but can be used internally
during reduction. Rules or rule fragments appearing in grey have been taken
directly from the original \emph{Featherweight Go} \autocite{fg} without any
modification. The rules (or rule fragments) in black show the changes introduced
when extending the rules to include arrays.

\subsection{FG Syntax}

An FG program is defined in a single file. The program starts with the package
name, and since we limit the language to a single file, the package must be
named $\main$, allowing the program to be compiled into an executable in regular
Go. It is then followed by a sequence of 0 or more declarations, and finally the
$\main$ function is required at the end. The $\main$ function consists of a
single expression, and to make it compatible with Go, the expression is assigned
to the blank identifier, denoted by an underscore. This makes it so that the Go
compiler does not reject the program because of an unused expression or
variable.

Two main types of declarations exist: a type declaration and a method
declaration. The latter is subdivided further into a regular method declaration
and an ``array set'' method declaration. The reason for having a special
syntactical form for methods that set a particular element of an array (more
precisely of a copy of an array, due to the value semantics of arrays), is that
FG is a functional subset of Go, and does not support variable assignment in the
general sense. By encapsulating the operation of creating an array copy with a
certain element taking on a new value, FG is able to maintain its functional
property. It also makes the rules simpler since there is no need to introduce
general assignment, yet the ``array set'' syntax is fully compatible with Go and
will behave as expected.

A type declaration is defined as the keyword \type, followed by a declared type
name, followed by a type literal. A ``declared'' type name can be any identifier
as defined by \autocite{spec}, except for the predeclared \kw{int}. In Go, it is
actually legal to redefine predeclared type names, such as the built-in
\kw{int}, however to keep things simple FG does not permit this. That is to say,
the only predeclared type name in FG is \kw{int}, for values of integer type.
No other primitive types (e.g. \kw{bool} or \kw{string}) are defined in FG.

A type literal can be one of 3 things: a struct literal, an interface literal,
or an array literal. A struct type literal consists of the keyword \struct,
followed by a sequence of fields inside curly braces. Each field is a pair of a
field identifier and a type name, separated by a space. A type name can be
either an aforementioned declared type name, or the predeclared type name
\kw{int}. The rules specify integer literals as valid type names. However, these
are not allowed in user programs, and are only used internally by the reduction
and typing rules.

An interface type literal consists of the keyword \interface, followed by a
sequence of method specifications inside curly braces. Each method specification
consists of a method name (an identifier) followed by the method signature. A
method signature consists of a sequence of parameters within parenthesis,
followed by a type name denoting the return type of the method. Each parameter
is a pair of a variable name (an identifier) and a type name, separated by a
space.

An array type literal consists of an integer literal within square brackets,
denoting the length of the array followed by a type name, denoting the type of
the array elements. This restricts array type literals to a single dimension,
i.e. \texttt{[2][3]int} is not allowed. However, in practice this is not a
concern, as one can define a nested array in two steps, i.e. declaring the inner
array as a type and using that declared type name as the element type of the
outer array.

A method declaration consists of the keyword \func, followed by the receiver in
parentheses, followed by a method specification, followed by the method body.
The method receiver is just a single parameter, where its type name refers to a
declared value type name. A value type name is a type name defined in terms of
either a struct type literal, or an array type literal. The method body consists
of the keyword \return, followed by an expression, all enclosed within curly
braces.

An array set method declaration is a special syntactical term in FG, which is a
valid method in Go. The receiver type refers to a declared array type name,
which is a type name defined in terms of an array literal. The method can have
any name (identifier), and its parameters can be any identifier as long as all
the parameters are named uniquely. However, there are stricter restrictions on
the rest of the construct. The return type must match the receiver type. The
first parameter must have a type of \kw{int}, while the second parameter may
(syntactically) be any type name. The method body is constructed from the
parameter names, where $x$ refers to the array, $x_1$ to the array index, and
$x_2$ to the new value to be assigned to that index. In figure
\ref{fig:fg-syntax}, identical metavariables should be understood as identical
syntactical terms in the rule defining array set methods, e.g. where a concrete
term appears for $x$ in one part of the rule, any other occurrence of the
metavariable $x$ must be instantiated with the same concrete term. The same does
not apply to other rules in the figure (e.g. array index consists of two
potentially distinct subexpressions). This is the only instance where,
syntactically, two statements are allowed within the method body, as there is no
way in Go to perform an array index assignment and return the array in a single
expression.

Finally, expressions may take on a number of forms. The most basic expressions
are integer literals (e.g. 0, 1, 2, 10, -1 etc.). Variables, which in the
context of FG happen to be parameter names, are also basic expressions.

Complex expressions may involve 0 or more subexpressions. The first kind are
method calls, which are recursively defined as an expression, followed by a dot,
followed by a method name (an identifier), followed by a sequence of expressions
enclosed in parentheses. Another recursively defined kind of expression is a
value literal, which consists of a value type name, followed by a sequence of
expressions enclosed in curly brackets. Value literals are used to instantiate
structs or arrays. The select expression consists of an expression, followed by
a dot and the field name (an identifier), which is used to select a field from a
struct. An array index consists of an expression, followed by another expression
enclosed in square brackets. The array index expression is used to retrieve an
element of an array (the first subexpression) at a specified index (the second
subexpression).

\documentclass[acmsmall,screen]{acmart}
\usepackage{xcolor}

\begin{document}

\newcommand{\squeeze}{\hspace{-1.5em}}
\newcommand{\gap}{\hspace{0.75em}}
\newcommand{\Strut}{\vphantom{\ov{f}}}
\newcommand{\calD}{\mathcal{D}}
\newcommand{\calS}{\mathcal{S}}
\newcommand{\ov}{\overline}
\newcommand{\at}{\mathrel{/}}
\newcommand{\kw}[1]{\texttt{\bf #1}}
\newcommand{\id}[1]{\texttt{#1}}
\newcommand{\meta}{\mathit}
\newcommand{\ok}{~\meta{ok}}
\newcommand{\val}{~\meta{value}}
\newcommand{\fields}{\meta{fields}}
\newcommand{\methods}{\meta{methods}}
\newcommand{\vtype}{\meta{type}}
\newcommand{\mbody}{\meta{body}}
\newcommand{\tdecls}{\meta{tdecls}}
\newcommand{\mdecls}{\meta{mdecls}}
\newcommand{\bounds}{\meta{bounds}}
\newcommand{\indexbounds}{\meta{indexBounds}}
\newcommand{\len}{\meta{lenType}}
\newcommand{\instance}{\meta{instance}}
\newcommand{\notref}{\meta{notReferenced}}
\newcommand{\isconst}{\meta{isConst}}
\newcommand{\isarraysetmethod}{\meta{isArraySetMethod}}
\newcommand{\flatten}{\meta{flatten}}
\newcommand{\unique}{\meta{unique}}
\newcommand{\distinct}{\meta{distinct}}
\newcommand{\elementtype}{\meta{elementType}}
\newcommand{\lentype}{\meta{lenType}}
\newcommand{\typeparams}{\meta{typeParams}}
\newcommand{\type}{\kw{type}}
\newcommand{\struct}{\kw{struct}}
\newcommand{\interface}{\kw{interface}}
\newcommand{\func}{\kw{func}}
\newcommand{\return}{\kw{return}}
\newcommand{\package}{\kw{package}}
\newcommand{\main}{\kw{main}}
\newcommand{\const}{\kw{const}}

\newcommand{\un}{\id{\textunderscore}}
\newcommand{\prog}{\rhd}
\newcommand{\br}[1]{\id{\{}#1\id{\}}}
\newcommand{\lst}[1]{[#1]}
\newcommand{\set}[1]{\{#1\}}
\newcommand{\an}[1]{\langle #1 \rangle}
\newcommand{\imp}{\mathbin{\id{<:}}}
\newcommand{\notimp}{\mathbin{\not\!\!\imp}}
\newcommand{\becomes}{\longrightarrow}
\newcommand{\by}{\mathbin{:=}}
\newcommand{\gray}[1]{{\color{gray}#1}}
\newcommand{\black}[1]{{\color{black}#1}}
\newcommand{\comma}{,\,}
\newcommand{\stoup}{;\,}
\newcommand{\Hole}{\Box}
\newcommand{\trep}{\ensuremath{\mathsf{Rep}}}

\newcommand{\sem}[1]{\llbracket #1 \rrbracket}
\newcommand{\bodies}{\meta{bodies}}
\newcommand{\fix}{\kw{fix}}
\newcommand{\lam}[1]{\lambda #1.\,}

\newcommand{\yields}{\blacktriangleright}
\newcommand{\extensionkw}{closure}
\newcommand{\Sclo}{\textit{S-\extensionkw}}
\newcommand{\Iclo}{\textit{I-\extensionkw}}
\newcommand{\Fclo}{\textit{F-\extensionkw}}
\newcommand{\Mclo}{\textit{M-\extensionkw}}
\newcommand{\Pclo}{\textit{P-\extensionkw}}
\newcommand{\Eclo}{\textit{E-\extensionkw}}

\newcommand{\SExtensionD}[2]{\Sclo(#1)}
\newcommand{\IExtensionD}[2]{\Iclo_{#2}(#1)}
\newcommand{\FExtensionD}[2]{\Fclo(#1)}
\newcommand{\MExtensionD}[2]{\Mclo_{#2}(#1)}
\newcommand{\EExtensionD}[2]{\Eclo(#1)}
\newcommand{\PExtensionD}[2]{\Pclo(#1)}

\newcommand{\derivrule}[3][]{
    \inferrule*[right=#1]
    {#2}
    {#3}
}
\newcommand{\axiomrule}[2][]{
    \inferrule*[right=#1]
    {~}
    {#2}
}

\newcommand{\register}[1]{
    \expandafter\newcommand\csname #1\endcsname{
        \operatorname{#1}
    }
}

\newcommand{\alias}[2]{
    \expandafter\newcommand\csname #1\endcsname{
        \operatorname{#2}
    }
}

\newcommand{\aliasparam}[1]{
    \expandafter\newcommand\csname #1Param\endcsname{
        \operatorname{#1}
    }
}

\newcommand{\ws} {
    % create some whitespace
    \begin{equation*}
    \end{equation*}
}

\newcommand{\golisting}[1] {
    \noindent\begin{minipage}{\linewidth}
        \lstinputlisting[language=Go, tabsize=4, firstline=3]{#1}
    \end{minipage}
}


\begin{figure}
    \gray{
        \begin{minipage}[t]{\textwidth}
            \begin{tabular}[t]{ll}
                Field name                 & $f$                                                       \\
                Method name                & $m$                                                       \\
                Variable name              & $x$                                                       \\
                Structure type name        & $t_S, u_S$                                                \\
                Interface type name        & $t_I, u_I$                                                \\
                \black{Array type name}    & \black{$t_A, u_A$}                                        \\
                \black{Declared type name} & \black{$t_D, u_D$} ::= $t_S \mid t_I~$\black{$\mid t_A$}  \\
                Type name                  & $t, u$ ::= \black{$t_D \mid \kw{int} \mid$ \fbox{$n$}}    \\
                \black{Value type name}    & \black{$t_V, u_V$ ::= $t_S \mid t_A$}                     \\
                Method signature           & $M$ ::= $(\ov{x~t})~t$                                    \\
                Method specification       & $S$ ::= $mM$                                              \\
                Type Literal               & $T$ ::=                                                   \\
                \quad Structure            & \quad $\struct~\br{\ov{f~t}}$                             \\
                \quad Interface            & \quad $\interface~\br{\ov{S}}$                            \\
                \quad \black{Array}        & \quad\black{$[n]t$}                                       \\
                Declaration                & $D$ ::=                                                   \\
                \quad Type declaration     & \quad $\type~\black{t_D}~T$                               \\
                \quad Method declaration   & \quad $\func~(x~$\black{$t_V$}$)~mM~\br{\return~e}$       \\
                \quad Array set method     &                                                           \\
                \quad declaration          & \quad \black{$\func~(x~t_A) ~m(x_1~\kw{int},~x_2~t) ~t_A~
                \br{ x[x_1] = x_2;~\return~x }$}                                                       \\
                Program                    & $P$ ::= $\package~\main;~\ov{D}~\func~\main()~\br{\un=e}$
            \end{tabular}
        \end{minipage}
        \hspace{-0.5\textwidth}
        \begin{minipage}[t]{0.4\textwidth}
            \begin{tabular}[t]{ll}
                Expression                     & $d, e$ ::=                     \\
                \quad \black{Integer literal } & \quad\black{$n$}               \\
                \quad Variable                 & \quad $x$                      \\
                \quad Method call              & \quad $e.m(\ov{e})$            \\
                \quad \black{Value literal}    & \quad $\black{t_V}\br{\ov{e}}$ \\
                \quad Select                   & \quad $e.f$                    \\
                \quad \black{Array index}      & \quad\black{$e$[$e$]}
            \end{tabular}
        \end{minipage}
    }
    \caption{FG syntax with arrays}
    \label{fig:fg-syntax}
\end{figure}
\end{document}


\subsection{FG Reduction}

Figure \ref{fig:fg-reduction} describes the small-step operational semantics of
FG, with auxiliary functions defined in figure \ref{fig:fg-reduction-aux}. A
value is a term that cannot be reduced further, i.e. is the final result of
computation. In FG, a value is either an integer literal, or a value literal
(struct or array), whose elements are all values themselves. An expression $d$
reduces to expression $e$ in a single step (denoted as $d \to e$), if one of the
reduction rule templates can be instantiated with the metaexpression $d \to e$
by pattern matching.

The auxiliary function $\fields$ looks up the struct type name given as an
argument in the sequence of declarations, and returns the sequence of fields in
the definition of the struct type. The rule R-Field says that a select
expression on a struct literal \emph{value} evaluates to the field value
corresponding to the same position in the struct literal as the field name is in
the struct type declaration.

The auxiliary function $\indexbounds$ returns a set of valid index indices of an
array type, i.e. the set of indices that are within bounds of the array. The
rule R-Index says an array index expression on an array literal \emph{value}
with an integer literal $n$ as the index reduces to the element of the array at
index $n$, if and only if $n$ is within bounds of the array.

The auxiliary function $\mbody$ looks up the method declaration given by the
type name and method name, and returns the expression in the body of that method
with a template for the receiver and parameters. The rule R-Call says that a
method call expression where the receiver is a value and the arguments are also
values, reduces to the $\mbody$ of the method, with the actual parameters
(receiver and arguments) substituted for the formal parameters in the $\mbody$
template.

The auxiliary predicate $\isarraysetmethod$ returns true if and only if the
method given by the type name and method name exists in the sequence of
declarations and is syntactically an array set method, as defined in figure
\ref{fig:fg-syntax}. The rule R-Array-Set says that given a method call
expression, where the receiver is an array literal value, the first argument is
an integer literal, and the second argument is a value, evaluates to an array
value literal, with the same elements but for the $n$\textsuperscript{th} index
which is replaced with the value in the second argument of the method call. This
rule can be applied if and only of the method in question is an array set method
and the integer literal $n$ is within bounds of the array.

Apart from the 4 computation rules described above, there is also a congruence
rule defined in terms of evaluation contexts, which says that if $d$ evaluates
to $e$, then $d$ in the context of $E$ evaluates to $e$ in the same context $E$.
The evaluation context defines the order of evaluation \autocite{evalContexts},
when there are multiple subexpressions in a single expression, and where at
least one of the subexpressions is not a value.

By the evaluation context rules, a method call must first reduce its receiver to
a value, and subsequently reduce its arguments to values, one by one. A value
literal has its elements reduced to values, one by one. A select expression must
first reduce its receiver to a value. An array index must first reduce its
receiver to an array literal value, then its index expression to a value.

% TODO explicitly capture panicking in reduction rules, in case indexBounds is
% violated

\begin{figure}
    \begin{mathpar}
        \gray{
            \inferrule
            {(\type~t_S~\struct\br{\ov{f~t}}) \in \ov{D}}
            {\fields(t_S) = \ov{f~t}}

            \inferrule
            {(\func~(x~\black{t_V})~m(\ov{x~t})~t~\br{\return~e}) \in \ov{D}}
            {\mbody(\black{t_V}.m) = (x:\black{t_V},\ov{x:t}).e}
        }

        \inferrule
        {
        (\type~t_A~ [n]t) \in \ov{D}
        }
        { \{ i \in \mathbb{Z} \mid 0 \le i < n \} = \indexbounds(t_A)}

        \inferrule
        {(\func~(x~t_A) ~m(x_1~\kw{int},~x_2~t) ~t_A~
            \br{ x[x_1] = x_2;~\return~x }) \in \ov{D}}
        {\isarraysetmethod(t_A.m)}
    \end{mathpar}
    \caption{FGA auxiliary functions for reduction rules}
    \label{fig:fg-reduction-aux}
\end{figure}


\begin{figure}
    \begin{center}
        Value \qquad $v$ ::= $t_V\br{\ov{v}} \mid n$
    \end{center}

    \begin{center}
        \gray{
            \begin{minipage}{0.5\textwidth}
                \begin{tabular}{ll}
                    Evaluation context          & $E$ ::=                      \\
                    \quad Hole                  & \quad $\Hole$                \\
                    \quad Method call receiver  & \quad $E.m(\ov{e})$          \\
                    \quad Method call arguments & \quad $v.m(\ov{v},E,\ov{e})$
                \end{tabular}
            \end{minipage}
            \begin{minipage}{0.4\textwidth}
                \begin{tabular}{ll}
                    ~                                                                      \\
                    \quad Value literal          & \quad $\black{t_V}\br{\ov{v},E,\ov{e}}$ \\
                    \quad Select                 & \quad $E.f$                             \\
                    \quad \black{Index receiver} & \quad \black{$E[e]$}                    \\
                    \quad \black{Index argument} & \quad \black{$t_A\br{\ov{v}}[E]$}
                \end{tabular}
            \end{minipage}
        }
    \end{center}

    Reduction \hfill \fbox{$d \becomes e$}
    \begin{mathpar}

        \gray{
            \inferrule[r-field]
            { (\ov{f~t}) = \fields(t_S) }
            { t_S\br{\ov{v}}.f_i \becomes v_i }
        }

        \inferrule[r-index]
        {
            n \in \indexbounds(t_A)
        }
        { t_A\br{\ov{v}}[n] \becomes v_n }

        \gray{
            \inferrule[r-call]
            { (x : \black{t_V},\ov{x: t}).e = \mbody(\vtype(v).m) }
            { v.m(\ov{v}) \becomes e[x \by v, \ov{x \by v}] }
        }

        % potential ambiguity with regular r-call?
        % no, because isarraysetmethod(t_A.m) and body(t_A.m) are mutually exclusive
        \inferrule[r-array-set]
        {
            n \in \indexbounds(t_A) \\
            \isarraysetmethod(t_A.m) \\
        }
        { t_A\br{\ov{v}}.m(n, v) \becomes t_A\br{\ov{v}}[n := v]}

        \gray{
            \inferrule[r-context]
            { d \becomes e }
            { E[d] \becomes E[e] }
        }

    \end{mathpar}
    \caption{FGA reduction rules}
    \label{fig:fg-reduction}
\end{figure}


\subsection{FG Typing}

\documentclass[acmsmall,screen]{acmart}
\usepackage{mathpartir}

\begin{document}

\newcommand{\squeeze}{\hspace{-1.5em}}
\newcommand{\gap}{\hspace{0.75em}}
\newcommand{\Strut}{\vphantom{\ov{f}}}
\newcommand{\calD}{\mathcal{D}}
\newcommand{\calS}{\mathcal{S}}
\newcommand{\ov}{\overline}
\newcommand{\at}{\mathrel{/}}
\newcommand{\kw}[1]{\texttt{\bf #1}}
\newcommand{\id}[1]{\texttt{#1}}
\newcommand{\meta}{\mathit}
\newcommand{\ok}{~\meta{ok}}
\newcommand{\val}{~\meta{value}}
\newcommand{\fields}{\meta{fields}}
\newcommand{\methods}{\meta{methods}}
\newcommand{\vtype}{\meta{type}}
\newcommand{\mbody}{\meta{body}}
\newcommand{\tdecls}{\meta{tdecls}}
\newcommand{\mdecls}{\meta{mdecls}}
\newcommand{\bounds}{\meta{bounds}}
\newcommand{\indexbounds}{\meta{indexBounds}}
\newcommand{\len}{\meta{lenType}}
\newcommand{\instance}{\meta{instance}}
\newcommand{\notref}{\meta{notReferenced}}
\newcommand{\isconst}{\meta{isConst}}
\newcommand{\isarraysetmethod}{\meta{isArraySetMethod}}
\newcommand{\flatten}{\meta{flatten}}
\newcommand{\unique}{\meta{unique}}
\newcommand{\distinct}{\meta{distinct}}
\newcommand{\elementtype}{\meta{elementType}}
\newcommand{\lentype}{\meta{lenType}}
\newcommand{\typeparams}{\meta{typeParams}}
\newcommand{\type}{\kw{type}}
\newcommand{\struct}{\kw{struct}}
\newcommand{\interface}{\kw{interface}}
\newcommand{\func}{\kw{func}}
\newcommand{\return}{\kw{return}}
\newcommand{\package}{\kw{package}}
\newcommand{\main}{\kw{main}}
\newcommand{\const}{\kw{const}}

\newcommand{\un}{\id{\textunderscore}}
\newcommand{\prog}{\rhd}
\newcommand{\br}[1]{\id{\{}#1\id{\}}}
\newcommand{\lst}[1]{[#1]}
\newcommand{\set}[1]{\{#1\}}
\newcommand{\an}[1]{\langle #1 \rangle}
\newcommand{\imp}{\mathbin{\id{<:}}}
\newcommand{\notimp}{\mathbin{\not\!\!\imp}}
\newcommand{\becomes}{\longrightarrow}
\newcommand{\by}{\mathbin{:=}}
\newcommand{\gray}[1]{{\color{gray}#1}}
\newcommand{\black}[1]{{\color{black}#1}}
\newcommand{\comma}{,\,}
\newcommand{\stoup}{;\,}
\newcommand{\Hole}{\Box}
\newcommand{\trep}{\ensuremath{\mathsf{Rep}}}

\newcommand{\sem}[1]{\llbracket #1 \rrbracket}
\newcommand{\bodies}{\meta{bodies}}
\newcommand{\fix}{\kw{fix}}
\newcommand{\lam}[1]{\lambda #1.\,}

\newcommand{\yields}{\blacktriangleright}
\newcommand{\extensionkw}{closure}
\newcommand{\Sclo}{\textit{S-\extensionkw}}
\newcommand{\Iclo}{\textit{I-\extensionkw}}
\newcommand{\Fclo}{\textit{F-\extensionkw}}
\newcommand{\Mclo}{\textit{M-\extensionkw}}
\newcommand{\Pclo}{\textit{P-\extensionkw}}
\newcommand{\Eclo}{\textit{E-\extensionkw}}

\newcommand{\SExtensionD}[2]{\Sclo(#1)}
\newcommand{\IExtensionD}[2]{\Iclo_{#2}(#1)}
\newcommand{\FExtensionD}[2]{\Fclo(#1)}
\newcommand{\MExtensionD}[2]{\Mclo_{#2}(#1)}
\newcommand{\EExtensionD}[2]{\Eclo(#1)}
\newcommand{\PExtensionD}[2]{\Pclo(#1)}

\newcommand{\derivrule}[3][]{
    \inferrule*[right=#1]
    {#2}
    {#3}
}
\newcommand{\axiomrule}[2][]{
    \inferrule*[right=#1]
    {~}
    {#2}
}

\newcommand{\register}[1]{
    \expandafter\newcommand\csname #1\endcsname{
        \operatorname{#1}
    }
}

\newcommand{\alias}[2]{
    \expandafter\newcommand\csname #1\endcsname{
        \operatorname{#2}
    }
}

\newcommand{\aliasparam}[1]{
    \expandafter\newcommand\csname #1Param\endcsname{
        \operatorname{#1}
    }
}

\newcommand{\ws} {
    % create some whitespace
    \begin{equation*}
    \end{equation*}
}

\newcommand{\golisting}[1] {
    \noindent\begin{minipage}{\linewidth}
        \lstinputlisting[language=Go, tabsize=4, firstline=3]{#1}
    \end{minipage}
}


\begin{figure}
    \begin{mathpar}
        \inferrule
        {(\type~t_A~ [n]t) \in \ov{D}}
        {t = \elementtype(t_A)}

        \inferrule
        {(\type~t_A~ [n]t) \in \ov{D}}
        {n = \len(t_A)}

        \inferrule
        {~}
        {\methods(\kw{int}) = \{\}}

        \inferrule
        {~}
        {\methods(n) = \{\}}

        \gray{
            \inferrule
            {}
            {\methods(\black{t_V}) = \{
                mM \mid
                (\func~(x~\black{t_V})~mM~\br{\return~e}) \in \ov{D}
                \}\\
                \black{\cup~\{
                    m(x_1~\kw{int},~x_2~t) ~t_V~ \mid
                    (\func~(x~t_V) ~m(x_1~\kw{int},~x_2~t) ~t_V~
                    \br{ x[x_1] = x_2;~\return~x }) \in \ov{D}
                    \}}}
        }

        \gray{
            \inferrule
            {\type~t_I~\interface\br{\ov{S}} \in \ov{D}}
            {\methods(t_I) = \ov{S}}

            \inferrule
            {(\type~t_S~\struct\br{\ov{f~t}}) \in \ov{D}}
            {\fields(t_S) = \ov{f~t}}

            \inferrule
            {\text{$mM_1, mM_2 \in \ov{S}$ implies $M_1 = M_2$}}
            {\unique(\ov{S})}
        }

        \gray{
            \inferrule
            {}
            {\tdecls(\ov{D}) = \lst{\black{t_D} \mid (\type~\black{t_D}~T) \in \ov{D}}}

            \inferrule
            {}
            {\mdecls(\ov{D}) = \lst{\black{t_V}.m \mid
                    (\func~(x~\black{t_V})~mM~\br{\return~e}) \in \ov{D}}}
        }
    \end{mathpar}
    \caption{FG auxiliary functions for array typing}
\end{figure}

\end{document}

\begin{figure}
    Implements, well-formed type
    \hfill \fbox{$t \imp u$} \qquad \fbox{$t \ok$}
    \begin{mathpar}

        \inferrule[$\imp_V$]
        {~}
        {t_V \imp t_V}

        \inferrule[$\imp_{int}$]
        {~}
        {\kw{int} \imp \kw{int}}

        \inferrule[$\imp_{n}$]
        {~}
        {n \imp n}

        \inferrule[$\imp_{int-n}$]
        {~}
        { n \imp \kw{int} }

        \gray{
            \inferrule[$\imp_I$]
            {
                \methods(t) \supseteq \methods(t_I)
            }
            { t \imp t_I }
        }

        \inferrule[t-int-type]
        {~}
        {\kw{int} \ok}

        \gray{
            \inferrule[t-named]
            {
                (\type~t~T) \in \ov{D}
            }
            { t \ok }
        }

    \end{mathpar}

    Well-formed method specifications and type literals
    \hfill \fbox {$S \ok$} \qquad \fbox{$T \ok$}
    \begin{mathpar}

        \inferrule[t-array]
        {
            n \ge 0\\
            t \ok
        }
        {[n]t \ok}

        \gray{
            \inferrule[t-specification]
            {
                \distinct(\ov{x}) \\
                \ov{t \ok} \\
                t \ok
            }
            { m(\ov{x~t})~t \ok }

            \inferrule[t-struct]
            {
                \distinct(\ov{f}) \\
                \ov{t \ok}
            }
            { \struct~\br{\ov{f~t}} \ok }

            \inferrule[t-interface]
            {
                \unique(\ov{S}) \\
                \ov{S \ok}
            }
            { \interface~\br{\ov{S}} \ok }
        }

    \end{mathpar}

    Well-formed declarations \hfill \fbox{$D \ok$}
    \begin{mathpar}

        \gray{
            \inferrule[t-type]
            {
                T \ok \\
                \black{\notref(t,~T)}
            }
            { \type~t~T \ok }

            \inferrule[t-func]
            {
                \distinct(x, \ov{x}) \\\\
                \black{t_V} \ok \\
                \black{m(\ov{x~t})~u \ok} \\
                x : \black{t_V} \comma \ov{x : t} \vdash e : t \\
                t \imp u
            }
            { \func~(x~\black{t_V})~m(\ov{x~t})~u~\br{\return~e} \ok }
        }

        \inferrule[t-func-arrayset]
        {
            u = \elementtype(t_A) \\
            t <: u \\
            t_A \ok
        }
        {
            \func~(x~t_A) ~m(x_1~\kw{int},~x_2~t) ~t_A~
            \br{ x[x_1] = x_2;~\return~x }
        }

    \end{mathpar}

    \caption{FGA typing rules (1 of 2)}
    \label{fig:fg-typing-1}
\end{figure}

\begin{figure}

    Expressions \hfill \fbox{$\Gamma \vdash e : t$}
    \begin{mathpar}

        \gray{
            \inferrule[t-var]
            { (x : t) \in \Gamma }
            { \Gamma \vdash x : t }

            \inferrule[t-call]
            {
                \Gamma \vdash e : t \\
                \Gamma \vdash \ov{e : t} \\
                (m(\ov{x~u})~u) \in \methods(t) \\
                \ov{t \imp u}
            }
            { \Gamma \vdash e.m(\ov{e}) : u }
        }

        \gray{
            \inferrule[t-struct-literal]
            {
                t_S \ok \\
                \Gamma \vdash \ov{e : t} \\
                (\ov{f~u}) = \fields(t_S) \\
                \ov{t \imp u}
            }
            { \Gamma \vdash t_S\br{\ov{e}} : t_S }

            \inferrule[t-field]
            {
                \Gamma \vdash e : t_S \\
                (\ov{f~u}) = \fields(t_S)
            }
            { \Gamma \vdash e.f_i : u_i }
        }

        \inferrule[t-array-literal]
        {
            t_A \ok \\
            |\ov{e}| = \len(t_A)\\
            \Gamma \vdash \ov{e : t} \\
            u = \elementtype(t_A) \\
            \ov{t <: u}
        }
        { \Gamma \vdash t_A\br{\ov{e}} : t_A }

        \inferrule[t-int-literal]
        {~}
        { \Gamma \vdash n : n }

        \inferrule[t-array-index]
        {
        \Gamma \vdash e_1 : t_A \\
        \Gamma \vdash e_2 : \kw{int} \\
        t = \elementtype(t_A)
        }
        { \Gamma \vdash e_1[e_2] : t }

        \inferrule[t-array-index-literal]
        {
        \Gamma \vdash e_1 : t_A \\
        \Gamma \vdash e_2 : n \\
        0 \le n < \len(t_A) \\
        t = \elementtype(t_A)
        }
        { \Gamma \vdash e_1[e_2] : t }

    \end{mathpar}

    Programs \hfill \fbox{$P \ok$}
    \begin{mathpar}
        \gray{
            \inferrule[t-prog]
            {
                \distinct(\tdecls(\ov{D})\black{, \kw{int}}) \\
                \distinct(\mdecls(\ov{D})) \\
                \ov{D \ok} \\
                \emptyset \vdash e : t
            }
            { \package~\main;~\ov{D}~\func~\main()~\br{\un=e} \ok }
        }
    \end{mathpar}

    \caption{FGA typing rules (2 of 2)}
    \label{fig:fg-typing-2}
\end{figure}


% TODO progress + preservation formally stated
