\section{Featherweight Go}

\emph{Featherweight Go} is a small, functional, Turing-complete subset of the Go
programming language, introduced by \citeauthor{fg} for the purpose of showing
how generics can be added to the language (\citeyear{fg}). This section will
extend \emph{Featherweight Go} with arrays, as found in Go. In a similar
fashion, only a subset of array features are included to keep things manageable.
In particular, slices are excluded.

A couple of notes on formal notation: a bar above a term or group of terms
denotes a sequence or a rule to be applied to each element in the sequence. A
sequence may contain 0 or more instances of the terms. In actual programs,
various delimiters are required between terms in a sequence. Depending on the
construct, this is either a comma or a semicolon (interchangeable with a
newline), but these details are omitted from the formal rules. A box around a
syntactical term means it cannot appear in a normal user program, but can be
used internally during reduction. Rules or rule fragments appearing in grey have
been taken directly from the original \emph{Featherweight Go} \autocite{fg}
without any modification. The rules in black show the changes introduced when
extending the rules to include arrays.

\subsection{FG Syntax}

\documentclass[acmsmall,screen]{acmart}
\usepackage{xcolor}

\begin{document}

\newcommand{\squeeze}{\hspace{-1.5em}}
\newcommand{\gap}{\hspace{0.75em}}
\newcommand{\Strut}{\vphantom{\ov{f}}}
\newcommand{\calD}{\mathcal{D}}
\newcommand{\calS}{\mathcal{S}}
\newcommand{\ov}{\overline}
\newcommand{\at}{\mathrel{/}}
\newcommand{\kw}[1]{\texttt{\bf #1}}
\newcommand{\id}[1]{\texttt{#1}}
\newcommand{\meta}{\mathit}
\newcommand{\ok}{~\meta{ok}}
\newcommand{\val}{~\meta{value}}
\newcommand{\fields}{\meta{fields}}
\newcommand{\methods}{\meta{methods}}
\newcommand{\vtype}{\meta{type}}
\newcommand{\mbody}{\meta{body}}
\newcommand{\tdecls}{\meta{tdecls}}
\newcommand{\mdecls}{\meta{mdecls}}
\newcommand{\bounds}{\meta{bounds}}
\newcommand{\indexbounds}{\meta{indexBounds}}
\newcommand{\len}{\meta{lenType}}
\newcommand{\instance}{\meta{instance}}
\newcommand{\notref}{\meta{notReferenced}}
\newcommand{\isconst}{\meta{isConst}}
\newcommand{\isarraysetmethod}{\meta{isArraySetMethod}}
\newcommand{\flatten}{\meta{flatten}}
\newcommand{\unique}{\meta{unique}}
\newcommand{\distinct}{\meta{distinct}}
\newcommand{\elementtype}{\meta{elementType}}
\newcommand{\lentype}{\meta{lenType}}
\newcommand{\typeparams}{\meta{typeParams}}
\newcommand{\type}{\kw{type}}
\newcommand{\struct}{\kw{struct}}
\newcommand{\interface}{\kw{interface}}
\newcommand{\func}{\kw{func}}
\newcommand{\return}{\kw{return}}
\newcommand{\package}{\kw{package}}
\newcommand{\main}{\kw{main}}
\newcommand{\const}{\kw{const}}

\newcommand{\un}{\id{\textunderscore}}
\newcommand{\prog}{\rhd}
\newcommand{\br}[1]{\id{\{}#1\id{\}}}
\newcommand{\lst}[1]{[#1]}
\newcommand{\set}[1]{\{#1\}}
\newcommand{\an}[1]{\langle #1 \rangle}
\newcommand{\imp}{\mathbin{\id{<:}}}
\newcommand{\notimp}{\mathbin{\not\!\!\imp}}
\newcommand{\becomes}{\longrightarrow}
\newcommand{\by}{\mathbin{:=}}
\newcommand{\gray}[1]{{\color{gray}#1}}
\newcommand{\black}[1]{{\color{black}#1}}
\newcommand{\comma}{,\,}
\newcommand{\stoup}{;\,}
\newcommand{\Hole}{\Box}
\newcommand{\trep}{\ensuremath{\mathsf{Rep}}}

\newcommand{\sem}[1]{\llbracket #1 \rrbracket}
\newcommand{\bodies}{\meta{bodies}}
\newcommand{\fix}{\kw{fix}}
\newcommand{\lam}[1]{\lambda #1.\,}

\newcommand{\yields}{\blacktriangleright}
\newcommand{\extensionkw}{closure}
\newcommand{\Sclo}{\textit{S-\extensionkw}}
\newcommand{\Iclo}{\textit{I-\extensionkw}}
\newcommand{\Fclo}{\textit{F-\extensionkw}}
\newcommand{\Mclo}{\textit{M-\extensionkw}}
\newcommand{\Pclo}{\textit{P-\extensionkw}}
\newcommand{\Eclo}{\textit{E-\extensionkw}}

\newcommand{\SExtensionD}[2]{\Sclo(#1)}
\newcommand{\IExtensionD}[2]{\Iclo_{#2}(#1)}
\newcommand{\FExtensionD}[2]{\Fclo(#1)}
\newcommand{\MExtensionD}[2]{\Mclo_{#2}(#1)}
\newcommand{\EExtensionD}[2]{\Eclo(#1)}
\newcommand{\PExtensionD}[2]{\Pclo(#1)}

\newcommand{\derivrule}[3][]{
    \inferrule*[right=#1]
    {#2}
    {#3}
}
\newcommand{\axiomrule}[2][]{
    \inferrule*[right=#1]
    {~}
    {#2}
}

\newcommand{\register}[1]{
    \expandafter\newcommand\csname #1\endcsname{
        \operatorname{#1}
    }
}

\newcommand{\alias}[2]{
    \expandafter\newcommand\csname #1\endcsname{
        \operatorname{#2}
    }
}

\newcommand{\aliasparam}[1]{
    \expandafter\newcommand\csname #1Param\endcsname{
        \operatorname{#1}
    }
}

\newcommand{\ws} {
    % create some whitespace
    \begin{equation*}
    \end{equation*}
}

\newcommand{\golisting}[1] {
    \noindent\begin{minipage}{\linewidth}
        \lstinputlisting[language=Go, tabsize=4, firstline=3]{#1}
    \end{minipage}
}


\begin{figure}
    \gray{
        \begin{minipage}[t]{\textwidth}
            \begin{tabular}[t]{ll}
                Field name                 & $f$                                                       \\
                Method name                & $m$                                                       \\
                Variable name              & $x$                                                       \\
                Structure type name        & $t_S, u_S$                                                \\
                Interface type name        & $t_I, u_I$                                                \\
                \black{Array type name}    & \black{$t_A, u_A$}                                        \\
                \black{Declared type name} & \black{$t_D, u_D$} ::= $t_S \mid t_I~$\black{$\mid t_A$}  \\
                Type name                  & $t, u$ ::= \black{$t_D \mid \kw{int} \mid$ \fbox{$n$}}    \\
                \black{Value type name}    & \black{$t_V, u_V$ ::= $t_S \mid t_A$}                     \\
                Method signature           & $M$ ::= $(\ov{x~t})~t$                                    \\
                Method specification       & $S$ ::= $mM$                                              \\
                Type Literal               & $T$ ::=                                                   \\
                \quad Structure            & \quad $\struct~\br{\ov{f~t}}$                             \\
                \quad Interface            & \quad $\interface~\br{\ov{S}}$                            \\
                \quad \black{Array}        & \quad\black{$[n]t$}                                       \\
                Declaration                & $D$ ::=                                                   \\
                \quad Type declaration     & \quad $\type~\black{t_D}~T$                               \\
                \quad Method declaration   & \quad $\func~(x~$\black{$t_V$}$)~mM~\br{\return~e}$       \\
                \quad Array set method     &                                                           \\
                \quad declaration          & \quad \black{$\func~(x~t_A) ~m(x_1~\kw{int},~x_2~t) ~t_A~
                \br{ x[x_1] = x_2;~\return~x }$}                                                       \\
                Program                    & $P$ ::= $\package~\main;~\ov{D}~\func~\main()~\br{\un=e}$
            \end{tabular}
        \end{minipage}
        \hspace{-0.5\textwidth}
        \begin{minipage}[t]{0.4\textwidth}
            \begin{tabular}[t]{ll}
                Expression                     & $d, e$ ::=                     \\
                \quad \black{Integer literal } & \quad\black{$n$}               \\
                \quad Variable                 & \quad $x$                      \\
                \quad Method call              & \quad $e.m(\ov{e})$            \\
                \quad \black{Value literal}    & \quad $\black{t_V}\br{\ov{e}}$ \\
                \quad Select                   & \quad $e.f$                    \\
                \quad \black{Array index}      & \quad\black{$e$[$e$]}
            \end{tabular}
        \end{minipage}
    }
    \caption{FG syntax with arrays}
    \label{fig:fg-syntax}
\end{figure}
\end{document}


\subsection{FG Reduction}

\begin{figure}
    \begin{mathpar}
        \gray{
            \inferrule
            {(\type~t_S~\struct\br{\ov{f~t}}) \in \ov{D}}
            {\fields(t_S) = \ov{f~t}}

            \inferrule
            {(\func~(x~\black{t_V})~m(\ov{x~t})~t~\br{\return~e}) \in \ov{D}}
            {\mbody(\black{t_V}.m) = (x:\black{t_V},\ov{x:t}).e}
        }

        \inferrule
        {
        (\type~t_A~ [n]t) \in \ov{D}
        }
        { \{ i \in \mathbb{Z} \mid 0 \le i < n \} = \indexbounds(t_A)}

        \inferrule
        {(\func~(x~t_A) ~m(x_1~\kw{int},~x_2~t) ~t_A~
            \br{ x[x_1] = x_2;~\return~x }) \in \ov{D}}
        {\isarraysetmethod(t_A.m)}
    \end{mathpar}
    \caption{FGA auxiliary functions for reduction rules}
    \label{fig:fg-reduction-aux}
\end{figure}


\begin{figure}
    \begin{center}
        Value \qquad $v$ ::= $t_V\br{\ov{v}} \mid n$
    \end{center}

    \begin{center}
        \gray{
            \begin{minipage}{0.5\textwidth}
                \begin{tabular}{ll}
                    Evaluation context          & $E$ ::=                      \\
                    \quad Hole                  & \quad $\Hole$                \\
                    \quad Method call receiver  & \quad $E.m(\ov{e})$          \\
                    \quad Method call arguments & \quad $v.m(\ov{v},E,\ov{e})$
                \end{tabular}
            \end{minipage}
            \begin{minipage}{0.4\textwidth}
                \begin{tabular}{ll}
                    ~                                                                      \\
                    \quad Value literal          & \quad $\black{t_V}\br{\ov{v},E,\ov{e}}$ \\
                    \quad Select                 & \quad $E.f$                             \\
                    \quad \black{Index receiver} & \quad \black{$E[e]$}                    \\
                    \quad \black{Index argument} & \quad \black{$t_A\br{\ov{v}}[E]$}
                \end{tabular}
            \end{minipage}
        }
    \end{center}

    Reduction \hfill \fbox{$d \becomes e$}
    \begin{mathpar}

        \gray{
            \inferrule[r-field]
            { (\ov{f~t}) = \fields(t_S) }
            { t_S\br{\ov{v}}.f_i \becomes v_i }
        }

        \inferrule[r-index]
        {
            n \in \indexbounds(t_A)
        }
        { t_A\br{\ov{v}}[n] \becomes v_n }

        \gray{
            \inferrule[r-call]
            { (x : \black{t_V},\ov{x: t}).e = \mbody(\vtype(v).m) }
            { v.m(\ov{v}) \becomes e[x \by v, \ov{x \by v}] }
        }

        % potential ambiguity with regular r-call?
        % no, because isarraysetmethod(t_A.m) and body(t_A.m) are mutually exclusive
        \inferrule[r-array-set]
        {
            n \in \indexbounds(t_A) \\
            \isarraysetmethod(t_A.m) \\
        }
        { t_A\br{\ov{v}}.m(n, v) \becomes t_A\br{\ov{v}}[n := v]}

        \gray{
            \inferrule[r-context]
            { d \becomes e }
            { E[d] \becomes E[e] }
        }

    \end{mathpar}
    \caption{FGA reduction rules}
    \label{fig:fg-reduction}
\end{figure}


\subsection{FG Typing}

\documentclass[acmsmall,screen]{acmart}
\usepackage{mathpartir}

\begin{document}

\newcommand{\squeeze}{\hspace{-1.5em}}
\newcommand{\gap}{\hspace{0.75em}}
\newcommand{\Strut}{\vphantom{\ov{f}}}
\newcommand{\calD}{\mathcal{D}}
\newcommand{\calS}{\mathcal{S}}
\newcommand{\ov}{\overline}
\newcommand{\at}{\mathrel{/}}
\newcommand{\kw}[1]{\texttt{\bf #1}}
\newcommand{\id}[1]{\texttt{#1}}
\newcommand{\meta}{\mathit}
\newcommand{\ok}{~\meta{ok}}
\newcommand{\val}{~\meta{value}}
\newcommand{\fields}{\meta{fields}}
\newcommand{\methods}{\meta{methods}}
\newcommand{\vtype}{\meta{type}}
\newcommand{\mbody}{\meta{body}}
\newcommand{\tdecls}{\meta{tdecls}}
\newcommand{\mdecls}{\meta{mdecls}}
\newcommand{\bounds}{\meta{bounds}}
\newcommand{\indexbounds}{\meta{indexBounds}}
\newcommand{\len}{\meta{lenType}}
\newcommand{\instance}{\meta{instance}}
\newcommand{\notref}{\meta{notReferenced}}
\newcommand{\isconst}{\meta{isConst}}
\newcommand{\isarraysetmethod}{\meta{isArraySetMethod}}
\newcommand{\flatten}{\meta{flatten}}
\newcommand{\unique}{\meta{unique}}
\newcommand{\distinct}{\meta{distinct}}
\newcommand{\elementtype}{\meta{elementType}}
\newcommand{\lentype}{\meta{lenType}}
\newcommand{\typeparams}{\meta{typeParams}}
\newcommand{\type}{\kw{type}}
\newcommand{\struct}{\kw{struct}}
\newcommand{\interface}{\kw{interface}}
\newcommand{\func}{\kw{func}}
\newcommand{\return}{\kw{return}}
\newcommand{\package}{\kw{package}}
\newcommand{\main}{\kw{main}}
\newcommand{\const}{\kw{const}}

\newcommand{\un}{\id{\textunderscore}}
\newcommand{\prog}{\rhd}
\newcommand{\br}[1]{\id{\{}#1\id{\}}}
\newcommand{\lst}[1]{[#1]}
\newcommand{\set}[1]{\{#1\}}
\newcommand{\an}[1]{\langle #1 \rangle}
\newcommand{\imp}{\mathbin{\id{<:}}}
\newcommand{\notimp}{\mathbin{\not\!\!\imp}}
\newcommand{\becomes}{\longrightarrow}
\newcommand{\by}{\mathbin{:=}}
\newcommand{\gray}[1]{{\color{gray}#1}}
\newcommand{\black}[1]{{\color{black}#1}}
\newcommand{\comma}{,\,}
\newcommand{\stoup}{;\,}
\newcommand{\Hole}{\Box}
\newcommand{\trep}{\ensuremath{\mathsf{Rep}}}

\newcommand{\sem}[1]{\llbracket #1 \rrbracket}
\newcommand{\bodies}{\meta{bodies}}
\newcommand{\fix}{\kw{fix}}
\newcommand{\lam}[1]{\lambda #1.\,}

\newcommand{\yields}{\blacktriangleright}
\newcommand{\extensionkw}{closure}
\newcommand{\Sclo}{\textit{S-\extensionkw}}
\newcommand{\Iclo}{\textit{I-\extensionkw}}
\newcommand{\Fclo}{\textit{F-\extensionkw}}
\newcommand{\Mclo}{\textit{M-\extensionkw}}
\newcommand{\Pclo}{\textit{P-\extensionkw}}
\newcommand{\Eclo}{\textit{E-\extensionkw}}

\newcommand{\SExtensionD}[2]{\Sclo(#1)}
\newcommand{\IExtensionD}[2]{\Iclo_{#2}(#1)}
\newcommand{\FExtensionD}[2]{\Fclo(#1)}
\newcommand{\MExtensionD}[2]{\Mclo_{#2}(#1)}
\newcommand{\EExtensionD}[2]{\Eclo(#1)}
\newcommand{\PExtensionD}[2]{\Pclo(#1)}

\newcommand{\derivrule}[3][]{
    \inferrule*[right=#1]
    {#2}
    {#3}
}
\newcommand{\axiomrule}[2][]{
    \inferrule*[right=#1]
    {~}
    {#2}
}

\newcommand{\register}[1]{
    \expandafter\newcommand\csname #1\endcsname{
        \operatorname{#1}
    }
}

\newcommand{\alias}[2]{
    \expandafter\newcommand\csname #1\endcsname{
        \operatorname{#2}
    }
}

\newcommand{\aliasparam}[1]{
    \expandafter\newcommand\csname #1Param\endcsname{
        \operatorname{#1}
    }
}

\newcommand{\ws} {
    % create some whitespace
    \begin{equation*}
    \end{equation*}
}

\newcommand{\golisting}[1] {
    \noindent\begin{minipage}{\linewidth}
        \lstinputlisting[language=Go, tabsize=4, firstline=3]{#1}
    \end{minipage}
}


\begin{figure}
    \begin{mathpar}
        \inferrule
        {(\type~t_A~ [n]t) \in \ov{D}}
        {t = \elementtype(t_A)}

        \inferrule
        {(\type~t_A~ [n]t) \in \ov{D}}
        {n = \len(t_A)}

        \inferrule
        {~}
        {\methods(\kw{int}) = \{\}}

        \inferrule
        {~}
        {\methods(n) = \{\}}

        \gray{
            \inferrule
            {}
            {\methods(\black{t_V}) = \{
                mM \mid
                (\func~(x~\black{t_V})~mM~\br{\return~e}) \in \ov{D}
                \}\\
                \black{\cup~\{
                    m(x_1~\kw{int},~x_2~t) ~t_V~ \mid
                    (\func~(x~t_V) ~m(x_1~\kw{int},~x_2~t) ~t_V~
                    \br{ x[x_1] = x_2;~\return~x }) \in \ov{D}
                    \}}}
        }

        \gray{
            \inferrule
            {\type~t_I~\interface\br{\ov{S}} \in \ov{D}}
            {\methods(t_I) = \ov{S}}

            \inferrule
            {(\type~t_S~\struct\br{\ov{f~t}}) \in \ov{D}}
            {\fields(t_S) = \ov{f~t}}

            \inferrule
            {\text{$mM_1, mM_2 \in \ov{S}$ implies $M_1 = M_2$}}
            {\unique(\ov{S})}
        }

        \gray{
            \inferrule
            {}
            {\tdecls(\ov{D}) = \lst{\black{t_D} \mid (\type~\black{t_D}~T) \in \ov{D}}}

            \inferrule
            {}
            {\mdecls(\ov{D}) = \lst{\black{t_V}.m \mid
                    (\func~(x~\black{t_V})~mM~\br{\return~e}) \in \ov{D}}}
        }
    \end{mathpar}
    \caption{FG auxiliary functions for array typing}
\end{figure}

\end{document}

\begin{figure}
    Implements, well-formed type
    \hfill \fbox{$t \imp u$} \qquad \fbox{$t \ok$}
    \begin{mathpar}

        \inferrule[$\imp_V$]
        {~}
        {t_V \imp t_V}

        \inferrule[$\imp_{int}$]
        {~}
        {\kw{int} \imp \kw{int}}

        \inferrule[$\imp_{n}$]
        {~}
        {n \imp n}

        \inferrule[$\imp_{int-n}$]
        {~}
        { n \imp \kw{int} }

        \gray{
            \inferrule[$\imp_I$]
            {
                \methods(t) \supseteq \methods(t_I)
            }
            { t \imp t_I }
        }

        \inferrule[t-int-type]
        {~}
        {\kw{int} \ok}

        \gray{
            \inferrule[t-named]
            {
                (\type~t~T) \in \ov{D}
            }
            { t \ok }
        }

    \end{mathpar}

    Well-formed method specifications and type literals
    \hfill \fbox {$S \ok$} \qquad \fbox{$T \ok$}
    \begin{mathpar}

        \inferrule[t-array]
        {
            n \ge 0\\
            t \ok
        }
        {[n]t \ok}

        \gray{
            \inferrule[t-specification]
            {
                \distinct(\ov{x}) \\
                \ov{t \ok} \\
                t \ok
            }
            { m(\ov{x~t})~t \ok }

            \inferrule[t-struct]
            {
                \distinct(\ov{f}) \\
                \ov{t \ok}
            }
            { \struct~\br{\ov{f~t}} \ok }

            \inferrule[t-interface]
            {
                \unique(\ov{S}) \\
                \ov{S \ok}
            }
            { \interface~\br{\ov{S}} \ok }
        }

    \end{mathpar}

    Well-formed declarations \hfill \fbox{$D \ok$}
    \begin{mathpar}

        \gray{
            \inferrule[t-type]
            {
                T \ok \\
                \black{\notref(t,~T)}
            }
            { \type~t~T \ok }

            \inferrule[t-func]
            {
                \distinct(x, \ov{x}) \\\\
                \black{t_V} \ok \\
                \black{m(\ov{x~t})~u \ok} \\
                x : \black{t_V} \comma \ov{x : t} \vdash e : t \\
                t \imp u
            }
            { \func~(x~\black{t_V})~m(\ov{x~t})~u~\br{\return~e} \ok }
        }

        \inferrule[t-func-arrayset]
        {
            u = \elementtype(t_A) \\
            t <: u \\
            t_A \ok
        }
        {
            \func~(x~t_A) ~m(x_1~\kw{int},~x_2~t) ~t_A~
            \br{ x[x_1] = x_2;~\return~x }
        }

    \end{mathpar}

    \caption{FGA typing rules (1 of 2)}
    \label{fig:fg-typing-1}
\end{figure}

\begin{figure}

    Expressions \hfill \fbox{$\Gamma \vdash e : t$}
    \begin{mathpar}

        \gray{
            \inferrule[t-var]
            { (x : t) \in \Gamma }
            { \Gamma \vdash x : t }

            \inferrule[t-call]
            {
                \Gamma \vdash e : t \\
                \Gamma \vdash \ov{e : t} \\
                (m(\ov{x~u})~u) \in \methods(t) \\
                \ov{t \imp u}
            }
            { \Gamma \vdash e.m(\ov{e}) : u }
        }

        \gray{
            \inferrule[t-struct-literal]
            {
                t_S \ok \\
                \Gamma \vdash \ov{e : t} \\
                (\ov{f~u}) = \fields(t_S) \\
                \ov{t \imp u}
            }
            { \Gamma \vdash t_S\br{\ov{e}} : t_S }

            \inferrule[t-field]
            {
                \Gamma \vdash e : t_S \\
                (\ov{f~u}) = \fields(t_S)
            }
            { \Gamma \vdash e.f_i : u_i }
        }

        \inferrule[t-array-literal]
        {
            t_A \ok \\
            |\ov{e}| = \len(t_A)\\
            \Gamma \vdash \ov{e : t} \\
            u = \elementtype(t_A) \\
            \ov{t <: u}
        }
        { \Gamma \vdash t_A\br{\ov{e}} : t_A }

        \inferrule[t-int-literal]
        {~}
        { \Gamma \vdash n : n }

        \inferrule[t-array-index]
        {
        \Gamma \vdash e_1 : t_A \\
        \Gamma \vdash e_2 : \kw{int} \\
        t = \elementtype(t_A)
        }
        { \Gamma \vdash e_1[e_2] : t }

        \inferrule[t-array-index-literal]
        {
        \Gamma \vdash e_1 : t_A \\
        \Gamma \vdash e_2 : n \\
        0 \le n < \len(t_A) \\
        t = \elementtype(t_A)
        }
        { \Gamma \vdash e_1[e_2] : t }

    \end{mathpar}

    Programs \hfill \fbox{$P \ok$}
    \begin{mathpar}
        \gray{
            \inferrule[t-prog]
            {
                \distinct(\tdecls(\ov{D})\black{, \kw{int}}) \\
                \distinct(\mdecls(\ov{D})) \\
                \ov{D \ok} \\
                \emptyset \vdash e : t
            }
            { \package~\main;~\ov{D}~\func~\main()~\br{\un=e} \ok }
        }
    \end{mathpar}

    \caption{FGA typing rules (2 of 2)}
    \label{fig:fg-typing-2}
\end{figure}


% TODO progress + preservation formally stated
