\section{Code examples}

\subsection{Generic programming in C using macros}
\label{sec:generic-c}

% TODO consider using newly introduced _Generic keyword (C11)
% https://codereview.stackexchange.com/questions/274860/concept-of-implementing-generic-types-in-c-using-macros

\lstinputlisting[language=C]{examples/macro-generics.c}

\subsection{Full FGA implementation of resizeable arrays}
\label{sec:fg-resizable-array-code}

In this particular example, the capacity of the resizeable array is hard-coded
to 5. This can of course be any value, but must be hard-coded without
generically sized arrays. Because FGA (unlike FGGA) does not support generics at
all, some structs are monomorphised ``by hand'' to work with all the types
required for the example, e.g. \texttt{Func} and \texttt{FuncA}.

One may notice that this implementation is much more verbose than the idiomatic
Go example. This is due to a design decision to keep FGA simple and not
introduce ``unnecessary'' constructs such as boolean and subtraction (and
therefore negative integers). Church-like encoding is used for booleans and
conditional logic \autocite{lambdaCalculus}, whereas natural numbers (that can
be incremented and decremented for the length of the \texttt{Array}) use
\texttt{succ}, \texttt{pred} and \texttt{isZero} similarly to how they're found
in \textcite{tapl}'s calculus of booleans and numbers. Functions have been
encoded in a similar way as in \emph{Featherweight Go} \autocite{fg} --- structs
are created to hold arguments, which implement the \texttt{Func} interface with
a single \texttt{call} method that takes no arguments. This way we can create
arbitrary closures (callbacks) for conditional selection.

\clearpage
\lstinputlisting[language=Go, tabsize=4, firstline=3]{../../projects/go-generic-array-sizes/interpreters/fg/examples/semidynamicarray/array.go}
