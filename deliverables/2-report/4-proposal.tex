\section{Generically sized arrays}

\subsection{Type-set interfaces and the \kw{const} interface}

We've already looked at type set based interfaces introduced in Go 1.18. If we
treat array sizes as types, then we now have a conceptual set of types 0, 1, 2,
3, etc. Similarly, we can conceptually define the \kw{const} type as the type
set of all array sizes:

\golisting{./examples/reversed/const_type.go}

Where the \texttt{...} means the pattern repeats for all the non-negative
integers. In practice, \kw{const} would be another ``special'' predeclared
interface type, just like the existing \texttt{comparable} that ``denotes the
set of all non-interface types that are strictly comparable'' \autocite{spec}.
Just like \texttt{comparable}, \kw{const} would be an interface that can only be
used as a type parameter constraint, and not e.g. as the type of a variable,
function parameter or return type.
This \texttt{comparable} interface is not defined in terms of regular Go code,
but rather exists on the level of the language itself. \kw{const} would follow
the same pattern. The choice of the identifier \kw{const} is to ensure backwards
compatibility with existing programs, as this keyword is currently not allowed
to be used as a type name. Just like a union type set, \kw{const} can be
instantiated with one of the elements of the union, i.e. a non-negative integer
literal. This restricts numerical type arguments to strictly compile-time
constant integers. Such a type parameter could then be used as the size of an
array. With such an extension, we can express the \texttt{reversed} function from
above as follows:

\golisting{./examples/reversed/const_generic_type.go}

Note how once again, the body of the function remains unchanged. The only
difference is that \texttt{N} is now a type parameter bound by the
\texttt{const} interface. The above function can operate on any array of any
size and any element type.

The rest of this work looks at the theory and implementation of the \kw{const}
type into the existing Go language. We will examine a language called
Featherweight Generic Go With Arrays (FGGA), which is a subset of Go, modulo the
addition of numerical type parameters. Since FGGA only considers ``classic''
(method set) interfaces, \kw{const} will not be an interface type in FGGA, but
rather in its own category. This category can be though of as non-method set
interfaces, since the two have the same restrictions, i.e. they can only be used
as type parameter bounds.

\subsection{Comparison with existing proposal}

Shortly after the Type Parameters proposal was published
\autocite{genericsProposal}, a proposal to extend generics to array sizes was
published \autocite{goArrayProposal}. It would allow type set interfaces of the
following form:

\golisting{./examples/werner/array_interface.go}

In this design, the parameterisation of the array size in implicit using
\texttt{...}, and does not appear in the list of type parameters, meaning the
numerical type parameter cannot be referenced directly. Instead, the proposal
author suggests applying the feature as follows:

\golisting{./examples/werner/matrix.go}

The example can in fact be simplified slightly, by inlining the type set
directly in into the type parameter constraint (allowed as of Go 1.18):

\golisting{./examples/werner/matrix_inline.go}

It could then be instantiated as follows:

\golisting{./examples/werner/matrix_variable.go}

Compare this with the proposal presented in this work. Using the \kw{const}
interface, the analogous code would be as follows:

\golisting{./examples/werner/my_matrix.go}

This approach mandates much less boilerplate than the current proposal, as the
type user is not forced to create ``dummy'' type arguments, and the type author
is not forced to retrieve an implicit numerical parameter through the
\texttt{len} function. Explicit numerical type parameters would make generic
arrays a first-class feature of Go, consistent with the rest of the language.
All the existing compound data types in Go can already be fully type
parameterised (slices: \texttt{[]T}, maps: \texttt{map[K]V} and channels:
\texttt{chan T}), except for arrays, so this work would bridge that gap
(\texttt{[N]T}), without making the feature feel like a workaround. In addition,
explicit numerical type parameters make the code more readable, as the
programmer can immediately see when a type is parameterised on integers.

Not only would arrays become first class, but so would numerical type
parameters. Currently, arrays are the only type which accept a numerical type
parameter, to parameterize the length of an array type. The \kw{const} interface
would allow any type or function to accept a constant integer (or another
\kw{const} bounded type parameter) as a type argument.

The benefit of Werner's proposal, is that is uses existing syntax:
\texttt{[...]T} can already be used to denote an array's type when constructing
an array literal:

\golisting{./examples/werner/array_literal.go}

where \texttt{myArray} has an inferred type of \texttt{[3]int}. It's worth
noting however, that this syntax is used for type inference, as opposed to
denote the type of a value, similar to how in some cases type arguments can be
omitted, where the compiler is able to infer what they are.

% TODO describe pros and cons of existing proposal
% cons: 
%       difficulty of implementing type arguments using expressions (e.g. len([...]T) + 1)

Another shortcoming of the implicit \texttt{[...]} syntax to parameterise array
sizes, is that it becomes unclear whether two type parameters of the same
constraint \texttt{[...]T} have the same length. If yes, then how do we express
two type parameters of the different lengths? If no, then how do we express that
two type parameters must have the same length? How about when we use the
shorthand syntax to collapse the bounds of multiple type parameters? Explicit
numerical type parameters make this differentiation trivial, enhancing the
readability of the code.

\begin{figure}
    \golisting{./examples/werner/array_pair.go}
    \golisting{./examples/werner/array_pair_collapsed.go}
    \caption{Examples of ambiguity when using \texttt{[...]T} to constrain
        multiple type parameters}
\end{figure}

The proposal mentions making \texttt{len} applicable to array \emph{types} in
addition to array values (as seen in Werner's examples presented above),
however, as pointed out in the GitHub
issue\footnote{\url{https://github.com/golang/go/issues/44253\#issuecomment-820999513}},
this is uneccessary as the desired behaviour can already be achieved by applying
\texttt{len} to an instantiated value of an array type parameter.

\subsection{The \texttt{len} function}

Go's built-in \texttt{len} function is special in the sense that depending on
the context, it may or may not be computed at compile time \autocite{spec}. If
it is computed at compile time, we can assign the result to a \kw{const}
variable, or use it in any other place that requires a compile-time constant
non-negative integer value, such as for the length of an array.

An example of a compile-time evaluation of \texttt{len} is when it is applied to
an array value literal, and an example of a run-time evaluation of \texttt{len}
is when it is applied to a slice value. The question arises, how should
\texttt{len} treat generic arrays (specifically, ones that are parameterised on
size)?

Interestingly, there are already cases, where a value of array type does not
have compile-time \texttt{len} semantics. In particular, the spec says
\texttt{len} does not yield a constant for array type expressions that ``contain
channel receives or (non-constant) function calls'' \autocite{spec}. While this
is indeed the case as of Go 1.21, assigning the result of such a
``non-constant'' expression to a variable, and then calling \texttt{len} on that
variable, \emph{does} yield a constant. As such, this restriction appears to be
a compiler limitation, as opposed to being inherent to the language.

If we take Rust's approach of prohibiting the use of expressions that include
constant type parameters (except when the expression is a lone type parameter)
as type arguments, which include applying \texttt{len} to a generically sized
array, then it becomes clear why \texttt{len} of a generically sized array
should yield a non-constant integer (as indeed is the case in Rust). Without
this restriction, the compiler would have to keep track of which constant was
derived from an expression containing a type-parameter, to prevent the later
usage of such a constant (or another const expression containing it) as a type
argument.

% TODO include code examples demonstrating the problem

In fact, if we view the generally sized arrays as a countably infinite union of
different array sizes (similar to the finite union we can already represent in
code), then we can keep things as they are currently in Go. I.e. a generic type
bound by a union of arrays has a non-constant \texttt{len}:

\golisting{./examples/werner/const_len_of_interface.go}

This is likely to avoid constraining the compiler implementation. With a full
monomorphisation approach, the \kw{const} of the example above is not an issue,
as each instantiation of the function has its own local \texttt{const n} with a
distinct value. However, in the current GC Shape Stenciling approach used by the
Go compiler as of Go 1.18 \autocite{generics1.18}, it is possible to construct a
union of array types that are have the same GC shape, yet the length of the
array type differs. In such cases, \kw{const}s within a generic function would
be difficult or even impossible to handle correctly. A simpler explanation, is
that type parameters simply always have non-constant lengths (given \texttt{len}
is defined on all types of the type set), regardless of whether or not values
of their constraint type could yield a constant length, \autocite{spec}.

If we permit non-trivial constant expressions containing type parameters as type
arguments, then the model is sound when \texttt{len} evaluates to a constant. As
a side remark, why would we need to even call \texttt{len} on a generically
sized array, since the numerical type parameter used to size it is in scope? The
same could be asked for the existing semantics of \texttt{len} on arrays: why
call \texttt{len} if the array size could be made a constant, and referenced in
multiple places? One justification comes to mind: when dealing with multiple
arrays of various sizes, it may be easier to use the wrong type parameter (or
constant in the non-generic case) than we intend to, whereas calling
\texttt{len} is guaranteed to yield the actual length of the array.

% TODO think of a scenario of using len as opposed to the type parameter?

% TODO why is len of a generic array not a constant in Rust?
% probably because it cannot be used as a type parameter

\subsection{Allowed \texttt{const} type arguments}

% Maybe we allow complex expressions for non-recursive structures? Go already
% does recursion detection in the compiler

% Evaluate both conservative (Rust) and permissive (Werner) approaches

% Discuss code bloat concerns

\subsection{Allowed operations on generic arrays}

% casting length to a value other than int is forbidden, as it must statically
% type check with any instantiation (of any size, including MaxInt), same
% applies to casting type parameter

% if len and type param evaluate to constant values, than that constant may only
% be cast to int (avoids SAT problem:
% https://github.com/golang/go/issues/44253#issuecomment-821047754)

% indexing with constant is forbidden, as it must statically type with any
% instantiation (of any size, including 0)
