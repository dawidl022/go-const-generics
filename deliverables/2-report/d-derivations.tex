\section{Formal derivation examples}

\subsection{Featherweight Go with Arrays reduction}
\label{sec:fg-derivation-example}

Below are derivation trees reducing an example FGA expression down to a value.
Each reduction step in the example program has its own derivation tree.

\documentclass{article}
\usepackage{amsmath}
\usepackage{amssymb}
\usepackage{ebproof}
\usepackage[left=0.5cm, right=0.5cm]{geometry}

\begin{document}

\newcommand{\squeeze}{\hspace{-1.5em}}
\newcommand{\gap}{\hspace{0.75em}}
\newcommand{\Strut}{\vphantom{\ov{f}}}
\newcommand{\calD}{\mathcal{D}}
\newcommand{\calS}{\mathcal{S}}
\newcommand{\ov}{\overline}
\newcommand{\at}{\mathrel{/}}
\newcommand{\kw}[1]{\texttt{\bf #1}}
\newcommand{\id}[1]{\texttt{#1}}
\newcommand{\meta}{\mathit}
\newcommand{\ok}{~\meta{ok}}
\newcommand{\val}{~\meta{value}}
\newcommand{\fields}{\meta{fields}}
\newcommand{\methods}{\meta{methods}}
\newcommand{\vtype}{\meta{type}}
\newcommand{\mbody}{\meta{body}}
\newcommand{\tdecls}{\meta{tdecls}}
\newcommand{\mdecls}{\meta{mdecls}}
\newcommand{\bounds}{\meta{bounds}}
\newcommand{\indexbounds}{\meta{indexBounds}}
\newcommand{\len}{\meta{lenType}}
\newcommand{\instance}{\meta{instance}}
\newcommand{\notref}{\meta{notReferenced}}
\newcommand{\isconst}{\meta{isConst}}
\newcommand{\isarraysetmethod}{\meta{isArraySetMethod}}
\newcommand{\flatten}{\meta{flatten}}
\newcommand{\unique}{\meta{unique}}
\newcommand{\distinct}{\meta{distinct}}
\newcommand{\elementtype}{\meta{elementType}}
\newcommand{\lentype}{\meta{lenType}}
\newcommand{\typeparams}{\meta{typeParams}}
\newcommand{\type}{\kw{type}}
\newcommand{\struct}{\kw{struct}}
\newcommand{\interface}{\kw{interface}}
\newcommand{\func}{\kw{func}}
\newcommand{\return}{\kw{return}}
\newcommand{\package}{\kw{package}}
\newcommand{\main}{\kw{main}}
\newcommand{\const}{\kw{const}}

\newcommand{\un}{\id{\textunderscore}}
\newcommand{\prog}{\rhd}
\newcommand{\br}[1]{\id{\{}#1\id{\}}}
\newcommand{\lst}[1]{[#1]}
\newcommand{\set}[1]{\{#1\}}
\newcommand{\an}[1]{\langle #1 \rangle}
\newcommand{\imp}{\mathbin{\id{<:}}}
\newcommand{\notimp}{\mathbin{\not\!\!\imp}}
\newcommand{\becomes}{\longrightarrow}
\newcommand{\by}{\mathbin{:=}}
\newcommand{\gray}[1]{{\color{gray}#1}}
\newcommand{\black}[1]{{\color{black}#1}}
\newcommand{\comma}{,\,}
\newcommand{\stoup}{;\,}
\newcommand{\Hole}{\Box}
\newcommand{\trep}{\ensuremath{\mathsf{Rep}}}

\newcommand{\sem}[1]{\llbracket #1 \rrbracket}
\newcommand{\bodies}{\meta{bodies}}
\newcommand{\fix}{\kw{fix}}
\newcommand{\lam}[1]{\lambda #1.\,}

\newcommand{\yields}{\blacktriangleright}
\newcommand{\extensionkw}{closure}
\newcommand{\Sclo}{\textit{S-\extensionkw}}
\newcommand{\Iclo}{\textit{I-\extensionkw}}
\newcommand{\Fclo}{\textit{F-\extensionkw}}
\newcommand{\Mclo}{\textit{M-\extensionkw}}
\newcommand{\Pclo}{\textit{P-\extensionkw}}
\newcommand{\Eclo}{\textit{E-\extensionkw}}

\newcommand{\SExtensionD}[2]{\Sclo(#1)}
\newcommand{\IExtensionD}[2]{\Iclo_{#2}(#1)}
\newcommand{\FExtensionD}[2]{\Fclo(#1)}
\newcommand{\MExtensionD}[2]{\Mclo_{#2}(#1)}
\newcommand{\EExtensionD}[2]{\Eclo(#1)}
\newcommand{\PExtensionD}[2]{\Pclo(#1)}

\newcommand{\derivrule}[3][]{
    \inferrule*[right=#1]
    {#2}
    {#3}
}
\newcommand{\axiomrule}[2][]{
    \inferrule*[right=#1]
    {~}
    {#2}
}

\newcommand{\register}[1]{
    \expandafter\newcommand\csname #1\endcsname{
        \operatorname{#1}
    }
}

\newcommand{\alias}[2]{
    \expandafter\newcommand\csname #1\endcsname{
        \operatorname{#2}
    }
}

\newcommand{\aliasparam}[1]{
    \expandafter\newcommand\csname #1Param\endcsname{
        \operatorname{#1}
    }
}

\newcommand{\ws} {
    % create some whitespace
    \begin{equation*}
    \end{equation*}
}

\newcommand{\golisting}[1] {
    \noindent\begin{minipage}{\linewidth}
        \lstinputlisting[language=Go, tabsize=4, firstline=3]{#1}
    \end{minipage}
}


\alias{AnyArrayTwo}{AnyArray2}
\register{any}
\register{this}
\register{First}
\register{Set}
\aliasparam{i}
\aliasparam{v}

\begin{align*}
    D_0    & = \type~\any~\interface\br{}                                       \\
    D_1    & = \type~\AnyArrayTwo~[2]\any                                       \\
    D_2    & = \func (\this~\AnyArrayTwo)~\First()~\any~\br{ \return~\this[0] } \\
    D_3    & = \func (\this~\AnyArrayTwo)~\Set(
    \iParam~\kw{int}, \vParam~\any
    )~\AnyArrayTwo~\br{
        \this[\iParam] = \vParam;~\return~\this
    }                                                                           \\
    \ov{D} & = (D_0, D_1, D_2, D_3)
\end{align*}

\begin{prooftree}
    \infer0{
        D_1 \in \ov{D}
    }
    \infer1{
        \{0, 1\} = \indexbounds(\AnyArrayTwo)
    }
    \infer1{
        0 \in \indexbounds(\AnyArrayTwo)
    }
    \infer0{
        D_3 \in \ov{D}
    }
    \infer1{
        \isarraysetmethod(\AnyArrayTwo.\Set)
    }
    \infer2[R-Array-Set]{
        \AnyArrayTwo\br{1, 2}.\Set(0, 3)
        \to
        \AnyArrayTwo\br{3, 2}
    }
    \infer1[R-Context]{
        \AnyArrayTwo\br{1, 2}.\Set(0, 3).\First()
        \to
        \AnyArrayTwo\br{3, 2}.\First()
    }
\end{prooftree}

\wss

\begin{prooftree}
    \infer0{
        D_2 \in \ov{D}
    }
    \infer1{
        (\this: \AnyArrayTwo).\this[0] = \mbody(\AnyArrayTwo.\First)
    }
    \infer1[R-Call]{
        \AnyArrayTwo\br{3, 2}.\First()
        \to
        \AnyArrayTwo\br{3, 2}[0]
    }
\end{prooftree}

\wss

\begin{prooftree}
    \infer0{
        D_1 \in \ov{D}
    }
    \infer1{
        \{0, 1\} = \indexbounds(\AnyArrayTwo)
    }
    \infer1{
        0 \in \indexbounds(\AnyArrayTwo)
    }
    \infer1[R-Index]{
        \AnyArrayTwo\br{3, 2}[0]
        \to
        3
    }
\end{prooftree}

\end{document}


\subsection{Featherweight Go with Arrays type checking}
\label{sec:fg-typing-derivation-example}

Below are derivation trees type checking two simple FGA programs, using a common set
of declarations. Since each type checking derivation tree would be too large to
fit on a single page, the trees have been split into multiple smaller subtrees.

\alias{AnyArrayTwo}{AnyArray2}
\register{any}
\register{this}
\register{First}
\register{Length}

\begin{align*}
    D_0    & = \type~\any~\interface\br{}                                       \\
    D_1    & = \type~\AnyArrayTwo~[2]\any                                       \\
    D_2    & = \func (\this~\AnyArrayTwo)~\First()~\any~\br{ \return~\this[0] } \\
    D_3    & = \func (\this~\AnyArrayTwo)~\Length()~\kw{int}~\br{ \return~2 }   \\
    \ov{D} & = (D_0, D_1, D_2, D_3)                                             \\
    e_1    & = \AnyArrayTwo\br{1, 2}.\First()                                   \\
    e_2    & = \AnyArrayTwo\br{1, 2}.\Length()                                  \\
    AA2    & = \AnyArrayTwo
\end{align*}

\ws

\begin{prooftree}
    \infer0[T-Interface]{
        \interface\br{} \ok
    }
    \infer1[T-Type]{
        D_0 \ok
    }
\end{prooftree}
\wss

\begin{prooftree}
    \infer0{
        2 \ge 0
    }
    \infer0{
        D_0 \in \ov{D}
    }
    \infer1[T-Named]{
        \any \ok
    }
    \infer2[T-Array]{
        [2]\any \ok
    }
    \infer1[T-Type]{
        D_1 \ok
    }
\end{prooftree}
\wss
\begin{prooftree}
    \infer0{
        (\this : AA2) \in (\this : AA2)
    }
    \infer1[T-Var]{
        \this : AA2 \vdash \this : AA2
    }
    \infer0[T-Int-Literal]{
        \this : AA2 \vdash  0 : 0
    }
    \infer0{
        0 \le 0 < 2
    }
    \infer1{
        0 \le 0 < \len(AA2)
    }
    \infer0{
        D_1 \in \ov{D}
    }
    \infer1{
        \any = \elementtype(AA2)
    }
    \infer4[T-Array-Index]{
        \this : AA2 \vdash \this[0] : \any
    }
\end{prooftree}
\wss
\begin{prooftree}
    \infer0{
        \methods(\any) \supseteq  \methods(\any)
    }
    \infer1[I]{
        \any \imp \any
    }
\end{prooftree}
\wss
\begin{prooftree}
    \infer0{
        \distinct(\this)
    }
    \infer0{
        D_1 \in \ov{D}
    }
    \infer1[T-Named]{
        \AnyArrayTwo \ok
    }
    \infer0{
        D_0 \in \ov{D}
    }
    \infer1[T-named]{
        \any \ok
    }
    \hypo{
        \this : \AnyArrayTwo \vdash \this[0] : \any
    }
    \hypo{
        \any \imp \any
    }
    \infer5[T-Func]{
        D_2 \ok
    }
\end{prooftree}
\wss
\begin{prooftree}
    \infer0{
        \distinct(\this)
    }
    \infer0{
        D_1 \in \ov{D}
    }
    \infer1[T-Named]{
        \AnyArrayTwo \ok
    }
    \infer0[T-Int-Type]{
        \kw{int} \ok
    }
    \infer0[T-Int-Literal]{
        \this : \AnyArrayTwo \vdash 2 : 2
    }
    \infer0[Int-N]{
        2 \imp \kw{int}
    }
    \infer5[T-Func]{
        D_3 \ok
    }
\end{prooftree}
\wss
\begin{prooftree}
    \infer0{
        \methods(1) \supseteq  \methods(\any)
    }
    \infer1[I]{
        1 \imp \any
    }
\end{prooftree}
\wss
\begin{prooftree}
    \infer0{
        \methods(2) \supseteq  \methods(\any)
    }
    \infer1[I]{
        2 \imp \any
    }
\end{prooftree}
\wss
\begin{prooftree}
    \infer0{
        D_1 \in \ov{D}
    }
    \infer1[T-Named]{
        \AnyArrayTwo \ok
    }
    \infer0[T-Int-Literal]{
        \emptyset \vdash 1: 1, 2: 2
    }
    \infer0{
        D_1 \in \ov{D}
    }
    \infer1{
        \any = \elementtype(\AnyArrayTwo)
    }
    \hypo{
        1 \imp \any
    }
    \hypo{
        2 \imp \any
    }
    \infer5[T-Array-Literal]{
        \emptyset \vdash \AnyArrayTwo\br{1, 2} : \AnyArrayTwo
    }
\end{prooftree}
\wss
\begin{prooftree}
    \hypo{
        \emptyset \vdash \AnyArrayTwo\br{1, 2} : \AnyArrayTwo
    }
    \infer0{
        \methods(\AnyArrayTwo) = \{ (\First()~\any), (\Length()~\kw{int}) \}
    }
    \infer1{
        (\First()~\any) \in \methods(\AnyArrayTwo)
    }
    \infer2[T-Call]{
        \emptyset \vdash e_1 : \any
    }
\end{prooftree}
\wss
\begin{prooftree}
    \infer0{
        \distinct(\tdecls(\ov{D}), \kw{int})
    }
    \infer0{
        \distinct(\mdecls(\ov{D}))
    }
    \hypo{
        D_0 \ok
    }
    \hypo{
        D_1 \ok
    }
    \hypo{
        D_2 \ok
    }
    \hypo{
        D_3 \ok
    }
    \infer4{
        \ov{D \ok}
    }
    \hypo{
        \emptyset \vdash e_1 : \any
    }
    \infer4[T-Prog]{ \package~\main;~\ov{D}~\func~\main()~\br{\un=e_1} \ok }
\end{prooftree}

\wss
\begin{prooftree}
    \hypo{
        \emptyset \vdash \AnyArrayTwo\br{1, 2} : \AnyArrayTwo
    }
    \infer0{
        \methods(\AnyArrayTwo) = \{ (\First()~\any), (\Length()~\kw{int}) \}
    }
    \infer1{
        (\Length()~\kw{int}) \in \methods(\AnyArrayTwo)
    }
    \infer2[T-Call]{
        \emptyset \vdash e_2 : \kw{int}
    }
\end{prooftree}

\wss
\begin{prooftree}
    \infer0{
        \distinct(\tdecls(\ov{D}))
    }
    \infer0{
        \distinct(\mdecls(\ov{D}))
    }
    \hypo{
        D_0 \ok
    }
    \hypo{
        D_1 \ok
    }
    \hypo{
        D_2 \ok
    }
    \hypo{
        D_3 \ok
    }
    \infer4{
        \ov{D \ok}
    }
    \hypo{
        \emptyset \vdash e_2 : \kw{int}
    }
    \infer4[T-Prog]{ \package~\main;~\ov{D}~\func~\main()~\br{\un=e_2} \ok }
\end{prooftree}


\subsection{Featherweight Generic Go with Arrays reduction}
\label{sec:fgg-derivation-example}

Below are derivation trees reducing an example FGGA expression down to a value.

\documentclass{article}
\usepackage{amsmath}
\usepackage{amssymb}
\usepackage{ebproof}
\usepackage[left=0.5cm, right=0.5cm]{geometry}

\begin{document}

\newcommand{\squeeze}{\hspace{-1.5em}}
\newcommand{\gap}{\hspace{0.75em}}
\newcommand{\Strut}{\vphantom{\ov{f}}}
\newcommand{\calD}{\mathcal{D}}
\newcommand{\calS}{\mathcal{S}}
\newcommand{\ov}{\overline}
\newcommand{\at}{\mathrel{/}}
\newcommand{\kw}[1]{\texttt{\bf #1}}
\newcommand{\id}[1]{\texttt{#1}}
\newcommand{\meta}{\mathit}
\newcommand{\ok}{~\meta{ok}}
\newcommand{\val}{~\meta{value}}
\newcommand{\fields}{\meta{fields}}
\newcommand{\methods}{\meta{methods}}
\newcommand{\vtype}{\meta{type}}
\newcommand{\mbody}{\meta{body}}
\newcommand{\tdecls}{\meta{tdecls}}
\newcommand{\mdecls}{\meta{mdecls}}
\newcommand{\bounds}{\meta{bounds}}
\newcommand{\indexbounds}{\meta{indexBounds}}
\newcommand{\len}{\meta{lenType}}
\newcommand{\instance}{\meta{instance}}
\newcommand{\notref}{\meta{notReferenced}}
\newcommand{\isconst}{\meta{isConst}}
\newcommand{\isarraysetmethod}{\meta{isArraySetMethod}}
\newcommand{\flatten}{\meta{flatten}}
\newcommand{\unique}{\meta{unique}}
\newcommand{\distinct}{\meta{distinct}}
\newcommand{\elementtype}{\meta{elementType}}
\newcommand{\lentype}{\meta{lenType}}
\newcommand{\typeparams}{\meta{typeParams}}
\newcommand{\type}{\kw{type}}
\newcommand{\struct}{\kw{struct}}
\newcommand{\interface}{\kw{interface}}
\newcommand{\func}{\kw{func}}
\newcommand{\return}{\kw{return}}
\newcommand{\package}{\kw{package}}
\newcommand{\main}{\kw{main}}
\newcommand{\const}{\kw{const}}

\newcommand{\un}{\id{\textunderscore}}
\newcommand{\prog}{\rhd}
\newcommand{\br}[1]{\id{\{}#1\id{\}}}
\newcommand{\lst}[1]{[#1]}
\newcommand{\set}[1]{\{#1\}}
\newcommand{\an}[1]{\langle #1 \rangle}
\newcommand{\imp}{\mathbin{\id{<:}}}
\newcommand{\notimp}{\mathbin{\not\!\!\imp}}
\newcommand{\becomes}{\longrightarrow}
\newcommand{\by}{\mathbin{:=}}
\newcommand{\gray}[1]{{\color{gray}#1}}
\newcommand{\black}[1]{{\color{black}#1}}
\newcommand{\comma}{,\,}
\newcommand{\stoup}{;\,}
\newcommand{\Hole}{\Box}
\newcommand{\trep}{\ensuremath{\mathsf{Rep}}}

\newcommand{\sem}[1]{\llbracket #1 \rrbracket}
\newcommand{\bodies}{\meta{bodies}}
\newcommand{\fix}{\kw{fix}}
\newcommand{\lam}[1]{\lambda #1.\,}

\newcommand{\yields}{\blacktriangleright}
\newcommand{\extensionkw}{closure}
\newcommand{\Sclo}{\textit{S-\extensionkw}}
\newcommand{\Iclo}{\textit{I-\extensionkw}}
\newcommand{\Fclo}{\textit{F-\extensionkw}}
\newcommand{\Mclo}{\textit{M-\extensionkw}}
\newcommand{\Pclo}{\textit{P-\extensionkw}}
\newcommand{\Eclo}{\textit{E-\extensionkw}}

\newcommand{\SExtensionD}[2]{\Sclo(#1)}
\newcommand{\IExtensionD}[2]{\Iclo_{#2}(#1)}
\newcommand{\FExtensionD}[2]{\Fclo(#1)}
\newcommand{\MExtensionD}[2]{\Mclo_{#2}(#1)}
\newcommand{\EExtensionD}[2]{\Eclo(#1)}
\newcommand{\PExtensionD}[2]{\Pclo(#1)}

\newcommand{\derivrule}[3][]{
    \inferrule*[right=#1]
    {#2}
    {#3}
}
\newcommand{\axiomrule}[2][]{
    \inferrule*[right=#1]
    {~}
    {#2}
}

\newcommand{\register}[1]{
    \expandafter\newcommand\csname #1\endcsname{
        \operatorname{#1}
    }
}

\newcommand{\alias}[2]{
    \expandafter\newcommand\csname #1\endcsname{
        \operatorname{#2}
    }
}

\newcommand{\aliasparam}[1]{
    \expandafter\newcommand\csname #1Param\endcsname{
        \operatorname{#1}
    }
}

\newcommand{\ws} {
    % create some whitespace
    \begin{equation*}
    \end{equation*}
}

\newcommand{\golisting}[1] {
    \noindent\begin{minipage}{\linewidth}
        \lstinputlisting[language=Go, tabsize=4, firstline=3]{#1}
    \end{minipage}
}


\alias{AnyArrayTwo}{AnyArray2}
\register{any}
\register{this}
\register{First}
\register{Set}
\aliasparam{i}
\aliasparam{v}
\aliasparam{N}
\aliasparam{T}
\register{Array}
\register{Get}
\register{Length}

\begin{align*}
    D_0    & = \type~\any~\interface\br{}                                                                       \\
    D_1    & = \type~\Array[\NParam~\const, \TParam~\any]~[\NParam]\TParam                                      \\
    D_2    & = \func (\this~\Array[\NParam, \TParam])~\Get(\iParam~\kw{int})~\any~\br{ \return~\this[\iParam] } \\
    D_3    & = \func (\this~\Array[\NParam, \TParam])~\Set(
    \iParam~\kw{int}, \vParam~\TParam
    )~\AnyArrayTwo~\br{
        \this[\iParam] = \vParam;~\return~\this
    }                                                                                                           \\
    \ov{D} & = (D_0, D_1, D_2, D_3)
\end{align*}

\ws

\begin{prooftree}
    \infer0{
        D_1 \in \ov{D}
    }
    \infer1{
        \{0, 1\} = \indexbounds(\Array[2, \kw{int}])
    }
    \infer1{
        0 \in \indexbounds(\Array[2, \kw{int}])
    }
    \infer0{
        D_3 \in \ov{D}
    }
    \infer1{
        \isarraysetmethod(\Array.\Set)
    }
    \infer2[R-Array-Set]{
        \Array[2, \kw{int}]\br{1, 2}.\Set(0, 3)
        \to
        \Array[2, \kw{int}]\br{3, 2}
    }
    \infer1[R-Context]{
        \Array[2, \kw{int}]\br{1, 2}.\Set(0, 3).\Get(0)
        \to
        \Array[2, \kw{int}]\br{3, 2}.\Get(0)
    }
\end{prooftree}

\wss

\begin{prooftree}
    \infer0{
        D_2 \in \ov{D}
    }
    \infer1{
        (\this: \Array[2, \kw{int}], i: \kw{int}).\this[i] = \mbody(\Array[2, \kw{int}].\First)
    }
    \infer1[R-Call]{
        \Array[2, \kw{int}]\br{3, 2}.\Get(0)
        \to
        \Array[2, \kw{int}]\br{3, 2}[0]
    }
\end{prooftree}

\wss

\begin{prooftree}
    \infer0{
        D_1 \in \ov{D}
    }
    \infer1{
        \{0, 1\} = \indexbounds(\Array[2, \kw{int}])
    }
    \infer1{
        0 \in \indexbounds(\Array[2, \kw{int}])
    }
    \infer1[R-Index]{
        \Array[2, \kw{int}]\br{3, 2}[0]
        \to
        3
    }
\end{prooftree}

\end{document}


\subsection{Featherweight Generic Go with Arrays type checking}
\label{sec:fgg-typing-derivation-example}

Below is a derivation tree type checking a simple FGGA program.

% TODO consider type checking second program, or removing e2 from the list of
% expressions

\begin{align*}
    D_0       & = \type~\any~\interface\br{}                                                                       \\
    D_1       & = \type~\Array[\NParam~\const, \TParam~\any]~[\NParam]\TParam                                      \\
    D_2       & = \func (\this~\Array[\NParam, \TParam])~\Get(\iParam~\kw{int})~\any~\br{ \return~\this[\iParam] } \\
    \ov{D}    & = (D_0, D_1, D_2)                                                                                  \\
    e_1       & = \Array[2, \kw{int}]\br{1, 2}.\Get(0)                                                             \\
    e_2       & = \Array[2, \kw{int}]\br{1, 2}.\Length()                                                           \\
    \ov{\Phi} & = (\NParam~\const,\TParam~\any)
\end{align*}


\begin{mathpar}

    \derivrule[T-Type]{
        \axiomrule[T-Interface]{
            \emptyset \vdash \interface\br{} \ok
        }
        \\
        \axiomrule{
            \notref(\any, \interface\br{})
        }
    }{
        D_0 \ok
    }

    \derivrule[T-Formal]{
        \axiomrule{\distinct(\NParam, \TParam)} \\
        \axiomrule[T-Const]{\ov{\Phi} \vdash \const \ok} \\
        \derivrule[T-Named]{
            \axiomrule{D_0 \in \ov{D}}
        }{
            \ov{\Phi} \vdash \any \ok
        }
    }{
        \ov{\Phi} \ok
    }

    \derivrule{
        \derivrule[Const-Param]{
            \axiomrule{
                (\NParam : \const) \in \ov{\Phi}
            }
        }{
            \ov{\Phi} \vdash \NParam \imp \const
        }
    }{
        \isconst_{\ov{\Phi}}(\NParam)
    }

    \derivrule[T-Param]{
        \axiomrule{
            (\NParam : \kw{const}) \in \ov{\Phi}
        }
    }{
        \ov{\Phi} \vdash \NParam \ok
    }

    \derivrule[T-Array]{
        \ov{\Phi} \vdash \NParam \ok
        \\
        \isconst_{\ov{\Phi}}(\NParam)
        \\
        \derivrule[T-Param]{
            \axiomrule{
                (\TParam : \any) \in \ov{\Phi}
            }
        }{
            \ov{\Phi} \vdash \TParam \ok
        }
        \\
        \axiomrule{
            \neg \isconst_{\ov{\Phi}}(\TParam)
        }
    }{
        \ov{\Phi} \vdash [\NParam]\TParam \ok
    }

    \axiomrule{
        \notref_\alpha(\Array, \kw{const})
    }

    \derivrule{
        \axiomrule{
            \Array \neq \any
        }
    }{
        \notref_\alpha(\Array, \any)
    }

    \derivrule{
        \axiomrule{
            \notref(\Array, \TParam)
        }
    }{
        \notref(\Array, [\NParam]\TParam)
    }

    \derivrule[T-Type]
    {
        \ov{\Phi} \ok
        \\
        \notref_\alpha(\Array, \kw{const})
        \\
        \notref_\alpha(\Array, \any)
        \\
        \ov{\Phi} \vdash [\NParam]\TParam \ok
        \\
        \notref(\Array, [\NParam]\TParam)
    }
    {
        D_1 \ok
    }

    \axiomrule{
        \distinct(\this)
    }

    \axiomrule{
        D_1 \in \ov{D}
    }

    \axiomrule{
        \neg \isconst_{\ov{\Phi}}(\kw{int})
    }

    \axiomrule[T-Int-Type]{
        \ov{\Phi} \vdash \kw{int} \ok
    }

    \derivrule[T-Param]{
        (\TParam : \any) \in \ov{\Phi}
    }{
        \ov{\Phi} \vdash \TParam \ok
    }

    \derivrule
    {
        D_1 \in \ov{D}
    }
    {
        \ov{\Phi} = \typeparams(\Array)
    }

\end{mathpar}
\begin{mathpar}

    \derivrule{
        \eta_0 = (\NParam \by \NParam, \TParam \by \TParam)
    }{
        \eta_0 = (\NParam~\const \by \NParam, \TParam~\any \by \TParam)
    }

    \derivrule{
        D_1 \in \ov{D}
    }{
        \TParam = \elementtype(\Array)[\eta_0]
    }

    \derivrule[T-Var]
    {
        (\this : \Array[\NParam, \TParam]) \in (\this: \Array[\NParam, \TParam], \iParam : \kw{int})
    }
    {
        \ov{\Phi}; \this: \Array[\NParam, \TParam], \iParam : \kw{int} \vdash
        \this : \Array[\NParam, \TParam]
    }

    \derivrule[T-Var]
    {
        (\iParam : \kw{int}) \in (\this: \Array[\NParam, \TParam], \iParam : \kw{int})
    }
    {
        \ov{\Phi}; \this: \Array[\NParam, \TParam], \iParam : \kw{int} \vdash
        \iParam : \kw{int}
    }

    \derivrule[T-Array-Index]
    {
        \ov{\Phi} = \typeparams(\Array)                                 \\
        \eta_0 = (\NParam~\const \by \NParam, \TParam~\any \by \TParam) \\
        \TParam = \elementtype(\Array)[\eta_0]                          \\
        \ov{\Phi}; \this: \Array[\NParam, \TParam], \iParam : \kw{int} \vdash
        \this : \Array[\NParam, \TParam]                                \\
        \ov{\Phi}; \this: \Array[\NParam, \TParam], \iParam : \kw{int} \vdash
        \iParam : \kw{int}
    }
    {
        \ov{\Phi}; \this: \Array[\NParam, \TParam] \vdash \this[\iParam] : \TParam
    }

    \axiomrule[$\imp_{\text{Param}}$]{
        \ov{\Phi} \vdash \TParam \imp \TParam
    }

    \axiomrule{
        \neg \isconst_{\ov{\Phi}}(T)
    }

    \derivrule[T-Func]{
        \distinct(\this)                                                           \\
        D_1 \in \ov{D}                                                             \\
        \neg \isconst_{\ov{\Phi}}(\kw{int})                                        \\
        \ov{\Phi} \vdash \kw{int} \ok                                              \\
        \ov{\Phi} \vdash \TParam \ok                                               \\
        \ov{\Phi}; \this: \Array[\NParam, \TParam] \vdash \this[\iParam] : \TParam \\
        \ov{\Phi} \vdash \TParam \imp \TParam                                      \\
        \neg \isconst_{\ov{\Phi}}(T)
    }{
        D_2 \ok
    }

    \derivrule[$\imp_{const-n}$]{
        \axiomrule{ 2 \ge 0 }
    }{
        \emptyset \vdash 2 \imp \const
    }

    \derivrule[$\imp_I$]{
        \axiomrule{
            \methods_\emptyset(\kw{int}) \supseteq \methods_\emptyset(\any)
        }
    }{
        \emptyset \vdash \kw{int} \imp \any
    }

    \derivrule
    {
        \derivrule{
            \eta_1 = (\NParam \by 2,
            \TParam \by \kw{int})
        }{
            \eta_1 = (\NParam~\const \by 2,
            \TParam~\any \by \kw{int})
        }
        \\
        \emptyset \vdash 2 \imp \const
        \\
        \emptyset \vdash \kw{int} \imp \any
    }
    {
        \eta_1 = (\NParam~\const \by_\emptyset 2,~
        \TParam~\any \by_\emptyset~\kw{int})
    }

    \derivrule
    {
        (\Get(\iParam~\kw{int})~\kw{int}) \in \{(\Get(\iParam~\kw{int})~\TParam)[\eta_1], (\Length()~\kw{int})[\eta_1] \}
    }
    {
        (\Get(\iParam~\kw{int})~\kw{int}) \in \methods_\emptyset(\Array[2, \kw{int}])
    }

    \derivrule[T-Named]
    {
        \derivrule[T-N-Type]
        { \axiomrule{ 2 \ge 0 } }
        { \emptyset \vdash 2 \ok }
        \\
        \derivrule[T-Int-Type]
        {~}
        { \emptyset \vdash \kw{int} \ok }
        \\
        \axiomrule
        { D_1 \in \ov{D} }
        \\
        \eta_1
    }
    {
        \emptyset \vdash \Array[2, \kw{int}] \ok
    }


\end{mathpar}
\begin{mathpar}


    \derivrule
    {
        \axiomrule{D_1 \in \ov{D}}
    }
    { \ov{\Phi} = \typeparams(\Array) }

    \derivrule
    {
        \eta_2 = (\NParam \by 2,~
        \TParam \by~\kw{int})
    }
    {
        \eta_2 = (\NParam~\const\by2,~
        \TParam~\any \by~\kw{int})
    }

    \derivrule
    { \kw{int} = \TParam[\eta_2] }
    { \kw{int} = \elementtype(\Array)[\eta_2] }

    \derivrule[T-Int-Literal]
    { ~ }
    { \emptyset; \emptyset \vdash 1: 1, 2: 2 }

    \derivrule[$\imp_{int-n}$]
    { ~ }
    { \emptyset \vdash 1 \imp \kw{int} }

    \derivrule[$\imp_{int-n}$]
    { ~ }
    { \emptyset \vdash 2 \imp \kw{int} }

    \derivrule[T-Array-Literal]{
        \emptyset \vdash \Array[2, \kw{int}] \ok \\
        \ov{\Phi} = \typeparams(\Array)          \\
        \eta_2 = (\NParam \by 2,
        \TParam \by \kw{int})                    \\
        \kw{int} = \elementtype(\Array)[\eta_2]  \\
        \emptyset; \emptyset \vdash 1: 1, 2: 2   \\
        \emptyset \vdash 1 \imp \kw{int}         \\
        \emptyset \vdash 2 \imp \kw{int}
    }{
        \emptyset; \emptyset \vdash \Array[2, \kw{int}]\br{1, 2} : \Array[2, \kw{int}]
    }

    \axiomrule[T-Int-Literal]{
        \emptyset; \emptyset \vdash 0 : 0
    }

    \axiomrule[Int-N]{
        \emptyset \vdash 0 \imp \kw{int}
    }

\end{mathpar}
\begin{mathpar}

    \derivrule[T-Call]{
        (\Get(\iParam~\kw{int})~\kw{int}) \in \methods_\emptyset(\Array[2, \kw{int}])
        \\
        \emptyset; \emptyset \vdash \Array[2, \kw{int}]\br{1, 2} : \Array[2, \kw{int}]
        \\
        \emptyset; \emptyset \vdash 0 : 0
        \\
        \emptyset \vdash 0 \imp \kw{int}
    }{
        \emptyset; \emptyset \vdash e_1 : \kw{int}
    }

    \derivrule{
        D_0 \ok
        \\
        D_1 \ok
        \\
        D_2 \ok
        \\
        D_3 \ok
    }{
        \ov{D \ok}
    }

    \derivrule[T-Prog]{
        \axiomrule{
            \distinct(\tdecls(\ov{D}), \kw{int})
        }
        \\
        \axiomrule{
            \distinct(\mdecls(\ov{D}))
        }
        \\
        \ov{D \ok}
        \\
        \emptyset; \emptyset \vdash e_1 : \kw{int}
    }{
        \package~\main;~\ov{D}~\func~\main()~\br{\un=e_1} \ok
    }
\end{mathpar}


% TODO revisit all derivation trees once FGA and FGGA rules are stable
