\section{Conclusion}

The main contributions this project achieved:

\begin{itemize}
      \item Formalisation of arrays in a subset of Go (Featherweight Go with
            Arrays).
      \item Formalising an extension of FGA with regular and numerical type
            parameters that can be used to create arrays of generic sizes.
      \item Implementing interpreters for the above two languages to dynamically
            test their safety properties. The implementation underwent rigorous
            testing with hundreds of small example programs.
      \item Formalising a translation (monomorphisation) of FGGA to regular Go.
      \item Implementing the formalised monomorphiser.
      \item Submitting a formal language feature proposal and discussing the
            addition with the Go community, including the core Go team.
      \item Reporting confirmed bugs in actual Go compiler as a result of the
            rigorous testing.
\end{itemize}

\subsection{Further work}

\begin{itemize}
      \item Formal proofs of progress and preservation properties, similar to the
            original \emph{Featherweight Go} work \autocite{fg}.
      \item Address Go team's feedback on proposal \autocite{myProposal} and
            come to a consensus with the Go community about the future of the
            new feature. This may include investigating the original vision for
            generics in Go e.g. from the list of omissions in the accepted
            proposal \autocite{genericsProposal} and the older contracts
            proposal \autocite{contractsProposal}.
      \item Go compiler implementation, which includes integrating the design
            proposed in this work with Go's ``GCShape stenciling with
            Dictionaries'' approach for generics \autocite{generics1.18}.
      \item Formalise a more expressive technique for genericising over arrays,
            e.g. by using refinements types to specify the range of numerical
            type parameters accepted by a type.
\end{itemize}
