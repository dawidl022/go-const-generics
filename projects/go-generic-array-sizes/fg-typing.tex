\documentclass[acmsmall,screen]{acmart}
\usepackage{mathpartir}

\begin{document}

\newcommand{\squeeze}{\hspace{-1.5em}}
\newcommand{\gap}{\hspace{0.75em}}
\newcommand{\Strut}{\vphantom{\ov{f}}}
\newcommand{\calD}{\mathcal{D}}
\newcommand{\calS}{\mathcal{S}}
\newcommand{\ov}{\overline}
\newcommand{\at}{\mathrel{/}}
\newcommand{\kw}[1]{\texttt{\bf #1}}
\newcommand{\id}[1]{\texttt{#1}}
\newcommand{\meta}{\mathit}
\newcommand{\ok}{~\meta{ok}}
\newcommand{\val}{~\meta{value}}
\newcommand{\fields}{\meta{fields}}
\newcommand{\methods}{\meta{methods}}
\newcommand{\vtype}{\meta{type}}
\newcommand{\mbody}{\meta{body}}
\newcommand{\tdecls}{\meta{tdecls}}
\newcommand{\mdecls}{\meta{mdecls}}
\newcommand{\bounds}{\meta{bounds}}
\newcommand{\indexbounds}{\meta{indexBounds}}
\newcommand{\len}{\meta{lenType}}
\newcommand{\instance}{\meta{instance}}
\newcommand{\notref}{\meta{notReferenced}}
\newcommand{\isconst}{\meta{isConst}}
\newcommand{\isarraysetmethod}{\meta{isArraySetMethod}}
\newcommand{\flatten}{\meta{flatten}}
\newcommand{\unique}{\meta{unique}}
\newcommand{\distinct}{\meta{distinct}}
\newcommand{\elementtype}{\meta{elementType}}
\newcommand{\lentype}{\meta{lenType}}
\newcommand{\typeparams}{\meta{typeParams}}
\newcommand{\type}{\kw{type}}
\newcommand{\struct}{\kw{struct}}
\newcommand{\interface}{\kw{interface}}
\newcommand{\func}{\kw{func}}
\newcommand{\return}{\kw{return}}
\newcommand{\package}{\kw{package}}
\newcommand{\main}{\kw{main}}
\newcommand{\const}{\kw{const}}

\newcommand{\un}{\id{\textunderscore}}
\newcommand{\prog}{\rhd}
\newcommand{\br}[1]{\id{\{}#1\id{\}}}
\newcommand{\lst}[1]{[#1]}
\newcommand{\set}[1]{\{#1\}}
\newcommand{\an}[1]{\langle #1 \rangle}
\newcommand{\imp}{\mathbin{\id{<:}}}
\newcommand{\notimp}{\mathbin{\not\!\!\imp}}
\newcommand{\becomes}{\longrightarrow}
\newcommand{\by}{\mathbin{:=}}
\newcommand{\gray}[1]{{\color{gray}#1}}
\newcommand{\black}[1]{{\color{black}#1}}
\newcommand{\comma}{,\,}
\newcommand{\stoup}{;\,}
\newcommand{\Hole}{\Box}
\newcommand{\trep}{\ensuremath{\mathsf{Rep}}}

\newcommand{\sem}[1]{\llbracket #1 \rrbracket}
\newcommand{\bodies}{\meta{bodies}}
\newcommand{\fix}{\kw{fix}}
\newcommand{\lam}[1]{\lambda #1.\,}

\newcommand{\yields}{\blacktriangleright}
\newcommand{\extensionkw}{closure}
\newcommand{\Sclo}{\textit{S-\extensionkw}}
\newcommand{\Iclo}{\textit{I-\extensionkw}}
\newcommand{\Fclo}{\textit{F-\extensionkw}}
\newcommand{\Mclo}{\textit{M-\extensionkw}}
\newcommand{\Pclo}{\textit{P-\extensionkw}}
\newcommand{\Eclo}{\textit{E-\extensionkw}}

\newcommand{\SExtensionD}[2]{\Sclo(#1)}
\newcommand{\IExtensionD}[2]{\Iclo_{#2}(#1)}
\newcommand{\FExtensionD}[2]{\Fclo(#1)}
\newcommand{\MExtensionD}[2]{\Mclo_{#2}(#1)}
\newcommand{\EExtensionD}[2]{\Eclo(#1)}
\newcommand{\PExtensionD}[2]{\Pclo(#1)}

\newcommand{\derivrule}[3][]{
    \inferrule*[right=#1]
    {#2}
    {#3}
}
\newcommand{\axiomrule}[2][]{
    \inferrule*[right=#1]
    {~}
    {#2}
}

\newcommand{\register}[1]{
    \expandafter\newcommand\csname #1\endcsname{
        \operatorname{#1}
    }
}

\newcommand{\alias}[2]{
    \expandafter\newcommand\csname #1\endcsname{
        \operatorname{#2}
    }
}

\newcommand{\aliasparam}[1]{
    \expandafter\newcommand\csname #1Param\endcsname{
        \operatorname{#1}
    }
}

\newcommand{\ws} {
    % create some whitespace
    \begin{equation*}
    \end{equation*}
}

\newcommand{\golisting}[1] {
    \noindent\begin{minipage}{\linewidth}
        \lstinputlisting[language=Go, tabsize=4, firstline=3]{#1}
    \end{minipage}
}



\begin{figure}
    \begin{mathpar}
        \inferrule
        {
        (\type~t_A~ [n]u) \in \ov{D} \\
        t <: u
        }
        {t = \elementtype(t_A)}

        \inferrule
        {n \ge 0}
        {\nonnegative(n)}
    \end{mathpar}
    \caption{FG auxiliary functions for arrays}
\end{figure}


\begin{figure}
    Implements
    \hfill \fbox{$t \imp u$}
    \begin{mathpar}

        \inferrule[<:$_V$]
        {~}
        {t_V <: t_V}

    \end{mathpar}

    Well-formed type literals
    \hfill \fbox{$T \ok$}
    \begin{mathpar}

        \inferrule[t-array]
        {
            \nonnegative(n)\\
            t \ok
        }
        {[n]t \ok}

    \end{mathpar}

    Well-formed declarations \hfill \fbox{$D \ok$}
    \begin{mathpar}

        \inferrule[t-func-arrayset]
        {
            t = \elementtype(t_A) \\
            t_A \ok
        }
        {
            \func~(x~t_A) ~m(x_1~\kw{int},~x_2~t) ~t_A~
            \br{ x[x_1] = x_2;~\return~x }
        }

    \end{mathpar}

    Expressions \hfill \fbox{$\Gamma \vdash e : t$}
    \begin{mathpar}

        \inferrule[t-array-literal]
        {
            t_A \ok \\
            \Gamma \vdash \ov{e : t} \\
            \ov{t = \elementtype(t_A)}
        }
        { \Gamma \vdash t_A\br{\ov{e}} : t_A }

        \inferrule[int-literal]
        {~}
        { \Gamma \vdash n : \kw{int} }

        \inferrule[t-array-index]
        {
        t_A \ok \\
        \Gamma \vdash e_1 : t_A,~e_2 : \kw{int} \\
        t = \elementtype(t_A)
        }
        { \Gamma \vdash e_1[e_2] : t }

        \inferrule[t-array-length]
        {
            \Gamma \vdash e: t_A \\
        }
        { \Gamma \vdash len(e) : \kw{int} }

    \end{mathpar}

    \caption{FG typing for arrays}
\end{figure}
\end{document}
