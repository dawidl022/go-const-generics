\documentclass[acmsmall,screen]{acmart}
\usepackage{mathpartir}

\begin{document}

\newcommand{\squeeze}{\hspace{-1.5em}}
\newcommand{\gap}{\hspace{0.75em}}
\newcommand{\Strut}{\vphantom{\ov{f}}}
\newcommand{\calD}{\mathcal{D}}
\newcommand{\calS}{\mathcal{S}}
\newcommand{\ov}{\overline}
\newcommand{\at}{\mathrel{/}}
\newcommand{\kw}[1]{\texttt{\bf #1}}
\newcommand{\id}[1]{\texttt{#1}}
\newcommand{\meta}{\mathit}
\newcommand{\ok}{~\meta{ok}}
\newcommand{\val}{~\meta{value}}
\newcommand{\fields}{\meta{fields}}
\newcommand{\methods}{\meta{methods}}
\newcommand{\vtype}{\meta{type}}
\newcommand{\mbody}{\meta{body}}
\newcommand{\tdecls}{\meta{tdecls}}
\newcommand{\mdecls}{\meta{mdecls}}
\newcommand{\bounds}{\meta{bounds}}
\newcommand{\indexbounds}{\meta{indexBounds}}
\newcommand{\len}{\meta{lenType}}
\newcommand{\instance}{\meta{instance}}
\newcommand{\notref}{\meta{notReferenced}}
\newcommand{\isconst}{\meta{isConst}}
\newcommand{\isarraysetmethod}{\meta{isArraySetMethod}}
\newcommand{\flatten}{\meta{flatten}}
\newcommand{\unique}{\meta{unique}}
\newcommand{\distinct}{\meta{distinct}}
\newcommand{\elementtype}{\meta{elementType}}
\newcommand{\lentype}{\meta{lenType}}
\newcommand{\typeparams}{\meta{typeParams}}
\newcommand{\type}{\kw{type}}
\newcommand{\struct}{\kw{struct}}
\newcommand{\interface}{\kw{interface}}
\newcommand{\func}{\kw{func}}
\newcommand{\return}{\kw{return}}
\newcommand{\package}{\kw{package}}
\newcommand{\main}{\kw{main}}
\newcommand{\const}{\kw{const}}

\newcommand{\un}{\id{\textunderscore}}
\newcommand{\prog}{\rhd}
\newcommand{\br}[1]{\id{\{}#1\id{\}}}
\newcommand{\lst}[1]{[#1]}
\newcommand{\set}[1]{\{#1\}}
\newcommand{\an}[1]{\langle #1 \rangle}
\newcommand{\imp}{\mathbin{\id{<:}}}
\newcommand{\notimp}{\mathbin{\not\!\!\imp}}
\newcommand{\becomes}{\longrightarrow}
\newcommand{\by}{\mathbin{:=}}
\newcommand{\gray}[1]{{\color{gray}#1}}
\newcommand{\black}[1]{{\color{black}#1}}
\newcommand{\comma}{,\,}
\newcommand{\stoup}{;\,}
\newcommand{\Hole}{\Box}
\newcommand{\trep}{\ensuremath{\mathsf{Rep}}}

\newcommand{\sem}[1]{\llbracket #1 \rrbracket}
\newcommand{\bodies}{\meta{bodies}}
\newcommand{\fix}{\kw{fix}}
\newcommand{\lam}[1]{\lambda #1.\,}

\newcommand{\yields}{\blacktriangleright}
\newcommand{\extensionkw}{closure}
\newcommand{\Sclo}{\textit{S-\extensionkw}}
\newcommand{\Iclo}{\textit{I-\extensionkw}}
\newcommand{\Fclo}{\textit{F-\extensionkw}}
\newcommand{\Mclo}{\textit{M-\extensionkw}}
\newcommand{\Pclo}{\textit{P-\extensionkw}}
\newcommand{\Eclo}{\textit{E-\extensionkw}}

\newcommand{\SExtensionD}[2]{\Sclo(#1)}
\newcommand{\IExtensionD}[2]{\Iclo_{#2}(#1)}
\newcommand{\FExtensionD}[2]{\Fclo(#1)}
\newcommand{\MExtensionD}[2]{\Mclo_{#2}(#1)}
\newcommand{\EExtensionD}[2]{\Eclo(#1)}
\newcommand{\PExtensionD}[2]{\Pclo(#1)}

\newcommand{\derivrule}[3][]{
    \inferrule*[right=#1]
    {#2}
    {#3}
}
\newcommand{\axiomrule}[2][]{
    \inferrule*[right=#1]
    {~}
    {#2}
}

\newcommand{\register}[1]{
    \expandafter\newcommand\csname #1\endcsname{
        \operatorname{#1}
    }
}

\newcommand{\alias}[2]{
    \expandafter\newcommand\csname #1\endcsname{
        \operatorname{#2}
    }
}

\newcommand{\aliasparam}[1]{
    \expandafter\newcommand\csname #1Param\endcsname{
        \operatorname{#1}
    }
}

\newcommand{\ws} {
    % create some whitespace
    \begin{equation*}
    \end{equation*}
}

\newcommand{\golisting}[1] {
    \noindent\begin{minipage}{\linewidth}
        \lstinputlisting[language=Go, tabsize=4, firstline=3]{#1}
    \end{minipage}
}


\begin{figure}

    Implements
    \hfill \fbox{$\Delta \vdash \tau \imp \sigma$}
    \begin{mathpar}
        \gray{
            \inferrule[<:-param]
            { ~ }
            { \Delta \vdash \alpha \imp \alpha }
        }

        \inferrule[<:$_V$]
        { ~ }
        { \Delta \vdash \tau_V \imp \tau_V }

        \inferrule[<:$_{int}$]
        {~}
        {\Delta \vdash \kw{int} <: \kw{int}}

        \inferrule[<:$_{const}$]
        {~}
        {\Delta \vdash \const <: \const}

        \inferrule[<:$_{n}$]
        {~}
        {\Delta \vdash n <: n}

        \gray{
            \inferrule[<:$_I$]
            {
                \methods_\Delta(\tau) \supseteq \methods_\Delta(\tau_I) \\
            }
            { \Delta \vdash \tau \imp \tau_I }
        }
        \\
        \inferrule[<:int-n]
        { ~ }
        { \Delta \vdash n \imp \kw{int} }

        \inferrule[<:const-n]
        { n \ge 0 }
        { \Delta \vdash n \imp \const }

        \inferrule[<:const-param]
        { (\alpha : \const) \in \Delta }
        { \Delta \vdash \alpha \imp \const }
    \end{mathpar}

    Well-formed type
    \hfill \fbox{$\Delta \vdash \tau \ok$}
    \begin{mathpar}
        \inferrule[t-n-type]
        { n \ge 0 }
        { \Delta \vdash n \ok }

        \inferrule[t-int-type]
        {~}
        { \Delta \vdash \kw{int} \ok }

        \inferrule[t-param]
        { (\alpha : \gamma) \in \Delta }
        { \Delta \vdash \alpha \ok }

        \inferrule[t-named]
        {
            \Delta \vdash \ov{\tau \ok}
            \and
            (\type~t[\ov{\Phi}]~T) \in \ov{D}
            \and
            \eta = (\ov{\Phi \by_\Delta \tau})
        }
        { \Delta \vdash t[\ov{\tau}] \ok }
    \end{mathpar}

    Well-formed type formals
    \hfill \fbox{$\Delta \vdash \const \ok$} \qquad \fbox{$\ov{\Phi} \ok$}
    \begin{mathpar}
        \inferrule[t-const]
        {~}
        {
            \Delta \vdash \const \ok
        }

        \inferrule[t-formal]
        {
            (\ov{\alpha~\gamma}) = \ov{\Phi} \\
            \distinct(\ov{\alpha}) \\
            \ov{\Phi} \vdash \ov{\gamma \ok}
        }
        { \ov{\Phi} \ok}

    \end{mathpar}

    Well-formed method specifications and type literals
    \hfill \fbox{$\ov{\Phi} \vdash S \ok$} \qquad \fbox{$\ov{\Phi} \vdash T \ok$}
    \begin{mathpar}
        \inferrule[t-specification]
        {
            \distinct(\ov{x}) \\
            \ov{\Phi} \vdash \ov{\tau \ok}\\
            \ov{\Phi} \vdash \tau \ok\\
            \ov{
                \neg \isconst_{\ov{\Phi}}(\tau)
            }\\
            \neg \isconst_{\ov{\Phi}}(\tau)
        }
        { \ov{\Phi} \vdash m(\ov{x~\tau})~\tau \ok }

        \inferrule[t-struct]
        {
            \distinct(\ov{f}) \\
            \ov{\Phi} \vdash \ov{\tau \ok}\\
            \ov{
                \neg \isconst_{\ov{\Phi}}(\tau)
            }
        }
        { \ov{\Phi} \vdash \struct~\br{\ov{f~\tau}} \ok }

        \inferrule[t-interface]
        {
            \unique(\ov{S}) \\
            \ov{\Phi} \vdash \ov{S \ok}
        }
        { \ov{\Phi} \vdash \interface~\br{\ov{S}} }

        \inferrule[t-array]
        {
            \ov{\Phi} \vdash \tau_n \ok\\
            \isconst_{\ov{\Phi}}(\tau_n)\\
            \ov{\Phi} \vdash \tau \ok\\
            \neg \isconst_{\ov{\Phi}}(\tau)
        }
        {\ov{\Phi} \vdash [\tau_n]\tau \ok}
    \end{mathpar}

    Well-formed declarations \hfill \fbox{$D \ok$}
    \begin{mathpar}
        \inferrule[t-type]
        {
        \ov{\Phi \ok} \\
        \ov{\Phi} \vdash T \ok
        }
        { \type~t[\ov{\Phi}]~T \ok }

        \inferrule[t-func]
        {
        \distinct(x, \ov{x}) \\
        (\type~t_V[\ov{\Phi}]~T) \in \ov{D} \\
        (\ov{\alpha~\gamma}) = \ov{\Phi}\\
        \ov{\neg \isconst_{\ov{\Phi}}(\tau)}\\
        \ov{\Phi} \vdash \ov{\tau \ok} \\
        \ov{\Phi} \vdash \sigma \ok \\
        \ov{\Phi} \stoup
        x : t_V[\ov{\alpha}] \comma \ov{x : \tau} \vdash e : \tau \\
        \ov{\Phi} \vdash \tau \imp \sigma \\
        \neg \isconst_{\ov{\Phi}}(\sigma)
        }
        { \func~(x~t_V[\ov{\alpha}])~m(\ov{x~\tau})~\sigma~\br{\return~e} \ok }
    \end{mathpar}
\end{figure}

\begin{figure}

    \begin{mathpar}
        \inferrule[t-func-arrayset]
        {
        \sigma = \elementtype(t_A) \\
        \tau \imp \sigma \\
        \ov{\Phi} = \typeparams(t_A)\\
        (\ov{\alpha~\gamma}) = \ov{\Phi}
        }
        {
        \func~(x~t_A[\ov{\alpha}]) ~m(x_1~\kw{int},~x_2~\tau) ~t_A~
        \br{ x[x_1] = x_2;~\return~x }
        }
    \end{mathpar}

    Expressions \hfill \fbox{$\Delta \stoup \Gamma \vdash e : \tau$}
    \begin{mathpar}
        \inferrule[t-int-literal]
        {~}
        { \Delta;\Gamma \vdash n : n }

        \gray{
            \inferrule[t-var]
            {
                (x : \tau) \in \Gamma
            }
            { \Delta \stoup \Gamma \vdash x : \tau }
        }

        \inferrule[t-call]
        {
            (m(\ov{x~\sigma})~\sigma) \in \methods_\Delta(\tau)  \\
            \Delta \stoup \Gamma \vdash e : \tau \\
            \Delta \stoup \Gamma \vdash \ov{e : \tau} \\
            \Delta \vdash (\ov{\tau \imp \sigma})
        }
        { \Delta \stoup \Gamma \vdash e.m(\ov{e}) : \sigma }

        \inferrule[t-array-literal]
        {
        \Delta \vdash \tau_A \ok \\
        \ov{\Phi} = \typeparams(t_A)\\
        \tau_A = t_A[\ov{\sigma}] \\
        \eta = (\ov{\Phi \by \sigma})\\
        \sigma = \elementtype(t_A)[\eta] \\
        \Delta ; \Gamma \vdash \ov{e : \tau} \\
        \Delta \vdash \ov{\tau <: \sigma}
        }
        { \Delta;\Gamma \vdash \tau_A\br{\ov{e}} : \tau_A }

        \inferrule[t-array-index]
        {
        \ov{\Phi} = \typeparams(t_A)\\
        \tau_A = t_A[\ov{\tau}] \\
        \eta = (\ov{\Phi \by \tau})\\
        \tau = \elementtype(t_A)[\eta]\\
        \Delta; \Gamma \vdash e_1 : \tau_A\\
        \Delta; \Gamma \vdash ~e_2 : \kw{int} \\
        }
        { \Delta;\Gamma \vdash e_1[e_2] : \tau }

        \inferrule[t-array-index-literal]
        {
        \ov{\Phi} = \typeparams(t_A)\\
        \tau_A = t_A[\ov{\tau}] \\
        \eta = (\ov{\Phi \by \tau})\\
        \tau = \elementtype(t_A)[\eta]\\
        \Delta; \Gamma \vdash e_1 : \tau_A\\
        \Delta; \Gamma \vdash ~e_2 : n \\
        0 \le n < \len(t_A)
        }
        { \Delta;\Gamma \vdash e_1[e_2] : \tau }

        \gray{
            \inferrule[t-struct-literal]
            {
                \Delta \vdash \tau_S \ok
                \quad
                \Delta \stoup \Gamma \vdash \ov{e : \tau}
                \quad
                (\ov{f~\sigma}) = \fields(\tau_S)
                \quad
                \Delta \vdash \ov{\tau \imp \sigma}
            }
            { \Delta \stoup \Gamma \vdash \tau_S\br{\ov{e}} : \tau_S }
        }

        \gray{
            \inferrule[t-field]
            {
                \Delta \stoup \Gamma \vdash e : \tau_S
                \quad
                (\ov{f~\tau}) = \fields(\tau_S)
            }
            { \Delta \stoup \Gamma \vdash e.f_i : \tau_i }
        }
    \end{mathpar}

    Programs  \hfill \fbox{$P \ok$}
    \begin{mathpar}
        \gray{
            \inferrule[t-prog]
            {
                \distinct(\tdecls(\ov{D})\black{, \kw{int}}) \\
                \distinct(\mdecls(\ov{D})) \\
                \ov{D \ok} \\
                \emptyset \stoup \emptyset \vdash e : \tau
            }
            { \package~\main;~\ov{D}~\func~\main()~\br{\un=e} \ok }
        }
    \end{mathpar}

    \caption{FGG typing  with arrays}
\end{figure}
\end{document}
