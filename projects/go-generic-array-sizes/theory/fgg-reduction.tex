\begin{figure}
    \begin{mathpar}
        \inferrule
        {
            (\ov{\alpha~\gamma}) = \ov{\Phi} \\
            \eta = (\ov{\alpha \by \tau})
        }
        {(\ov{\Phi \by \tau}) = \eta}

        \gray{
            \inferrule
            {
                (\type~t_S\black{[\ov{\Phi}]}~\struct~\br{\ov{f~\black{\tau}}}) \in \ov{D} \\
                \black{\eta = (\ov{\Phi \by \sigma})}
            }
            {\fields(t_S\black{[\ov{\sigma}]}) = (\ov{f~\black{\tau}})\black{\llbracket\eta\rrbracket}}
        }

        \gray{
            \inferrule
            {
                (\func~(x~t_V\black{[\ov{\alpha}]})~m(\ov{x~\black{\tau}})~\black{\tau}~\br{\return~e}) \in \ov{D} \\
                \black{\theta = (\ov{\alpha \by \sigma})}
            }
            {\mbody(t_V\black{[\ov{\sigma}]}.m) = (x:t_V\black{[\ov{\sigma}]},\ov{x:\black{\tau}}).e\black{\llbracket\theta\rrbracket}}
        }

        \gray{
        \inferrule
        {
        (\type~t_A\black{[\ov{\Phi}]}~ [n]\black{\tau}) \in \ov{D}\\
        \black{\tau_A = t_A[\ov{\tau}]}
        }
        { \{ i \in \mathbb{Z} \mid 0 \le i < n \} = \indexbounds(\black{\tau}_A)}
        }

        \inferrule
        {
        (\type~t_A[\ov{\Phi}, \alpha~\const, \ov{\Phi}']~ [\alpha]\tau) \in \ov{D}\\
        \tau_A = t_A[\ov{\tau}, n, \ov{\tau}']\\
        |\ov{\Phi}| = |\ov{\tau}|\\
        |\ov{\Phi}'| = |\ov{\tau}'|
        }
        { \{ i \in \mathbb{Z} \mid 0 \le i < n \} = \indexbounds(\tau_A)}

        \gray{
        \inferrule
        {
        (\func~(x~t_A\black{[\ov{\alpha}]}) ~m(x_1~\kw{int},~x_2~\black{\tau}) ~t_A\black{[\ov{\alpha}]}~
        \br{ x[x_1] = x_2;~\return~x }) \in \ov{D}\\
        \black{\tau_A = t_A[\ov{\tau}]}
        }
        {\isarraysetmethod(\black{\tau}_A.m)}
        }
    \end{mathpar}
    \caption{FGG auxiliary functions for reduction with arrays}
\end{figure}


\begin{figure}
    \begin{center}
        \gray{Value \qquad $v$ ::= $\black{\tau_V}\br{\ov{v}} \mid n$}
    \end{center}

    \begin{center}
        \gray{
            \begin{minipage}{0.5\textwidth}
                \begin{tabular}{ll}
                    Evaluation context          & $E$ ::=                      \\
                    \quad Hole                  & \quad $\Hole$                \\
                    \quad Method call receiver  & \quad $E.m(\ov{e})$          \\
                    \quad Method call arguments & \quad $v.m(\ov{v},E,\ov{e})$
                \end{tabular}
            \end{minipage}
            \begin{minipage}{0.4\textwidth}
                \begin{tabular}{ll}
                    ~                                                                 \\
                    \quad Value literal  & \quad $\black{\tau_V}\br{\ov{v},E,\ov{e}}$ \\
                    \quad Select         & \quad $E.f$                                \\
                    \quad Index receiver & \quad $E[e]$                               \\
                    \quad Index argument & \quad $\black{\tau_A}\br{\ov{v}}[E]$
                \end{tabular}
            \end{minipage}
        }
    \end{center}

    Reduction \hfill \fbox{$d \becomes e$}
    \begin{mathpar}

        \gray{
            \inferrule[r-field]
            { (\ov{f~\black{\tau}}) = \fields(\black{\tau_S}) }
            { \black{\tau_S}\br{\ov{v}}.f_i \becomes v_i }

            \inferrule[r-index]
            {
                n \in \indexbounds(\black{\tau_A})
            }
            { \black{\tau_A}\br{\ov{v}}[n] \becomes v_n }

            \inferrule[r-call]
            { (x : \black{\tau_V},\ov{x: \black{\tau}}).e = \mbody(\vtype(v).m) }
            { v.m(\ov{v}) \becomes e[x \by v, \ov{x \by v}] }
        }

        \gray{
            % potential ambiguity with regular r-call?
            % no, because isarraysetmethod(t_A.m) and body(t_A.m) are mutually exclusive
            \inferrule[r-array-set]
            {
                n \in \indexbounds(\black{\tau_A}) \\
                \isarraysetmethod(\black{\tau_A}.m) \\
            }
            { \black{\tau_A}\br{\ov{v}}.m(n, v) \becomes \black{\tau_A}\br{\ov{v}}[n := v]}

            \inferrule[r-context]
            { d \becomes e }
            { E[d] \becomes E[e] }
        }

    \end{mathpar}
    \caption{FGG reduction with arrays}
\end{figure}
