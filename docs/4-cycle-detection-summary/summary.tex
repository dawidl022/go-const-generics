\documentclass[12pt]{article}
\usepackage[margin=1in]{geometry}
\usepackage{mathpartir}
\usepackage{amsfonts}
\usepackage{amsmath}
\usepackage{amssymb}
\usepackage{stmaryrd}
\usepackage{xcolor}
\usepackage{hyperref}
\usepackage{listings}

\newcommand{\squeeze}{\hspace{-1.5em}}
\newcommand{\gap}{\hspace{0.75em}}
\newcommand{\Strut}{\vphantom{\ov{f}}}
\newcommand{\calD}{\mathcal{D}}
\newcommand{\calS}{\mathcal{S}}
\newcommand{\ov}{\overline}
\newcommand{\at}{\mathrel{/}}
\newcommand{\kw}[1]{\texttt{\bf #1}}
\newcommand{\id}[1]{\texttt{#1}}
\newcommand{\meta}{\mathit}
\newcommand{\ok}{~\meta{ok}}
\newcommand{\val}{~\meta{value}}
\newcommand{\fields}{\meta{fields}}
\newcommand{\methods}{\meta{methods}}
\newcommand{\vtype}{\meta{type}}
\newcommand{\mbody}{\meta{body}}
\newcommand{\tdecls}{\meta{tdecls}}
\newcommand{\mdecls}{\meta{mdecls}}
\newcommand{\bounds}{\meta{bounds}}
\newcommand{\indexbounds}{\meta{indexBounds}}
\newcommand{\len}{\meta{lenType}}
\newcommand{\notref}{\meta{notReferenced}}
\newcommand{\isconst}{\meta{isConst}}
\newcommand{\isarraysetmethod}{\meta{isArraySetMethod}}
\newcommand{\flatten}{\meta{flatten}}
\newcommand{\unique}{\meta{unique}}
\newcommand{\distinct}{\meta{distinct}}
\newcommand{\elementtype}{\meta{elementType}}
\newcommand{\lentype}{\meta{lenType}}
\newcommand{\typeparams}{\meta{typeParams}}
\newcommand{\type}{\kw{type}}
\newcommand{\struct}{\kw{struct}}
\newcommand{\interface}{\kw{interface}}
\newcommand{\func}{\kw{func}}
\newcommand{\return}{\kw{return}}
\newcommand{\package}{\kw{package}}
\newcommand{\main}{\kw{main}}
\newcommand{\const}{\kw{const}}

\newcommand{\un}{\id{\textunderscore}}
\newcommand{\prog}{\rhd}
\newcommand{\br}[1]{\id{\{}#1\id{\}}}
\newcommand{\lst}[1]{[#1]}
\newcommand{\set}[1]{\{#1\}}
\newcommand{\an}[1]{\langle #1 \rangle}
\newcommand{\imp}{\mathbin{\id{<:}}}
\newcommand{\notimp}{\mathbin{\not\!\!\imp}}
\newcommand{\becomes}{\longrightarrow}
\newcommand{\by}{\mathbin{:=}}
\newcommand{\gray}[1]{{\color{gray}#1}}
\newcommand{\black}[1]{{\color{black}#1}}
\newcommand{\comma}{,\,}
\newcommand{\stoup}{;\,}
\newcommand{\Hole}{\Box}
\newcommand{\trep}{\ensuremath{\mathsf{Rep}}}

\newcommand{\sem}[1]{\llbracket #1 \rrbracket}
\newcommand{\bodies}{\meta{bodies}}
\newcommand{\fix}{\kw{fix}}
\newcommand{\lam}[1]{\lambda #1.\,}
\newcommand{\derivrule}[3][]{
    \inferrule*[right=#1]
    {#2}
    {#3}
}
\newcommand{\axiomrule}[2][]{
    \inferrule*[right=#1]
    {~}
    {#2}
}

\newcommand{\register}[1]{
    \expandafter\newcommand\csname #1\endcsname{
        \operatorname{#1}
    }
}

\newcommand{\alias}[2]{
    \expandafter\newcommand\csname #1\endcsname{
        \operatorname{#2}
    }
}

\newcommand{\aliasparam}[1]{
    \expandafter\newcommand\csname #1Param\endcsname{
        \operatorname{#1}
    }
}

\newcommand{\ws} {
    % create some whitespace
    \begin{equation*}
    \end{equation*}
}

\newcommand{\golisting}[1] {
    \noindent\begin{minipage}{\linewidth}
        \lstinputlisting[language=Go, tabsize=4, firstline=3]{#1}
    \end{minipage}
}


\title{Summary of Go cycle detection rules}
\author{Dawid Lachowicz}

\begin{document}

\maketitle

\section{Syntax}

\begin{minipage}[t]{\textwidth}
    \begin{tabular}[t]{ll}
        Type                   & $\tau, \sigma$ ::=                                        \\
        \quad Type parameter   & \quad $\alpha$                                            \\
        \quad Named type       & \quad $t[\ov{\tau}]$                                      \\\\
        Type Literal           & $T$ ::=                                                   \\
        \quad Structure        & \quad $\struct~\br{\ov{f~\tau}}$                          \\
        \quad Interface        & \quad $\interface~\br{\ov{S}}$                            \\
        \quad Array            & \quad$[n]\tau$                                            \\
        \\
        Declaration            & $D$ ::=                                                   \\
        \quad Type declaration & \quad $\type~t[\ov{\Phi}]~T$                              \\
        \\
        Program                & $P$ ::= $\package~\main;~\ov{D}~\func~\main()~\br{\un=e}$
    \end{tabular}
\end{minipage}
\hspace{-0.5\textwidth}
\begin{minipage}[t]{0.4\textwidth}
    \begin{tabular}[t]{ll}
        Type name         & $t, u$                                       \\
        Type parameter                                                   \\
        declaration       & $\Phi$ ::= $\alpha~\gamma$                   \\
        Type constraint   & $\gamma$ ::= $t[\ov{\tau}]$                  \\
        \\
        Method signature  & $M$ ::= $(\ov{x~\black{\tau}})~\black{\tau}$ \\
        Interface element & $S$ ::= $mM$                                 \\
        \\
        Field name        & $f$                                          \\
        Method name       & $m$                                          \\
        Variable          & $x$
    \end{tabular}
\end{minipage}

\section{Typing rules}

\noindent Well-formed declarations \hfill \fbox{$D \ok$}
\begin{mathpar}
    \inferrule[T-Type]
    {
    \ov{\Phi \ok}
    \\
    \ov{\Phi} \vdash~T \ok
    \\
    \notcont(t,~T)
    }
    { \type~t[\ov{\Phi}]~T \ok }
\end{mathpar}

\noindent Programs  \hfill \fbox{$P \ok$}
\begin{mathpar}
    \inferrule[T-Prog]
    {
        \distinct(\tdecls(\ov{D}), \kw{int}) \\
        \distinct(\mdecls(\ov{D})) \\
        \ov{D \ok} \\
        \black{\emptyset \stoup}~\emptyset \vdash e : \black{\tau}
    }
    { \package~\main;~\ov{D}~\func~\main()~\br{\un=e} \ok }
\end{mathpar}

The remaining type checking rules (e.g. ones used in the examples) are provided
in section \ref{sec:remaining-rules}. They are omitted here to focus the
discussion on cycle detection.

\section{Cycle detection rules}

\subsection{Type containment rules}

\noindent Type declaration containment \hfill \fbox{$\notcont(\ov{t}, T)$}
\begin{mathpar}
    \inferrule
    {~}
    {
        \notcont(\ov{t_r},~\interface~\br{\ov{S}})
    }
    \\
    \inferrule
    {
        \notcont(\ov{t_r}, \tau)
    }
    {
        \notcont(\ov{t_r},~[n]\tau)
    }

    \inferrule
    {
        \forall \tau \in \ov{f~\tau}.\notcont(\ov{t_r}, \tau)
    }
    {
        \notcont(\ov{t_r},~\struct \br{\ov{f~\tau}})
    }
\end{mathpar}

\noindent Type declaration containment recursion \hfill \fbox{$\notcont(\ov{t}, \tau)$}
\begin{mathpar}
    \inferrule
    {~}
    {
        \notcont(\ov{t_r}, \kw{int})
    }

    \inferrule
    {~}
    {
        \notcont(\ov{t_r}, \alpha)
    }

    \inferrule
    {
    (\type~t[\ov{\Phi}]~T) \in \ov{D}
    \\
    t \notin \ov{t_r}
    \\
    \eta = (\ov{\Phi \by \tau})
    \\
    \notcont(\ov{t_r}, t, T \llbracket\eta\rrbracket)
    }
    {
    \notcont(\ov{t_r}, t[\ov{\tau}])
    }
\end{mathpar}

\begin{itemize}
    \item Type containment checks can be thought of as checking the structure of
          type literals, applicable to both generic and non-generic code.

    \item The $\notcont$ relation can be explained as ``asserting a type/type
          literal doesn't contain any of the \emph{already seen} types where they are
          not allowed to occur in the type's structure''.

    \item As recursion progresses, the encountered type names $t$ are added to
          the set of already seen type names ($\notcont(\ov{t_r}, t, T
              \llbracket\eta\rrbracket)$). This is so we can detect indirect cycles
          without getting the algorithm into an infinite loop.

    \item Basic interfaces (and other pointer types) can refer to the type being
          defined in their structure, and therefore are base cases of the
          recursion.

    \item $\notcont(\ov{t_r}, \kw{int})$ can be more widely applied to primitive
          (predeclared) types in Go. Since primitive types may be redefined in a Go
          program, we'd need to check whether a type name (e.g. $\kw{int}$) indeed
          still points at a primitive type.

    \item Type parameters shadow type names. Whenever a type parameter is
          encountered during the containment check, a base case of the recursion is
          reached.

    \item The $T \llbracket\eta\rrbracket$ notation indicates that the type
          literal has its type parameters $\alpha$ substituted with type arguments
          $\tau$.

    \item Note in particular, that these rules do not recurse on type parameter
          constraints of the checked types.
          \href{https://github.com/golang/go/issues/65714}{Issue \#65714}
          demonstrates a family of programs where the type checker sometimes
          rejects programs based on cycles mixing containment rules and type
          parameter constraint reference rules (described in the next section),
          depending on the ordering of type declarations.
\end{itemize}

% TODO mention that non-basic interfaces (e.g. embedded or type set interfaces)
% need to be checked for containment cycles

\subsection{Lifting type parameter reference rules}

\begin{itemize}
    \item The Go spec currently states: ``Within a type parameter list of a
          generic type T, a type constraint may not (directly, or indirectly
          through the type parameter list of another generic type) refer to T.''
          We can in fact life this restriction, i.e. allowing type parameter
          constraints to refer to the type T being defined.
    \item The next section contains a few examples of programs that either the
          the compiler currently struggles with, or are currently disallowed by
          the compiler due to the above-mentioned rule in the spec, but do not
          actually cause any problems during type-checking and could be allowed.
          Both positive (well-typed programs) and negative (badly-typed
          programs) examples are presented, to demonstrate that the rules
          presented in this document are sufficient for both cases.
    \item Some self-referential type parameters may be uninstantiable, however,
          Go already permits defining non-instantiable types, and defining empty
          type sets is also allowed. This is fine since types need to be
          explicitly instantiated, and checks also occur upon type
          instantiation. Type containment checks are critical however, since
          values can be instantiated implicitly via zero values, and so the type
          declaration (or instantiation in the case of generic types) must
          ensure that valid (non-infinite) zero values can be constructed.
\end{itemize}

\golisting{./examples/empty-type-set.go}

\section{Examples}

\subsection{Containment check in generic type}

\noindent\begin{minipage}[t]{.45\linewidth}
    \lstinputlisting[language=Go, tabsize=4, firstline=3]
    {./examples/generic-containment-any.go}

    The definition of \texttt{Bar} should be rejected, since it creates a type
    containment cycle via \texttt{Foo}. The compiler correctly reports \texttt{Bar}
    as an invalid recursive type.
\end{minipage}
\hfill
\noindent\begin{minipage}[t]{.45\linewidth}
    \lstinputlisting[language=Go, tabsize=4, firstline=3, lastline=35]
    {./examples/generic-containment-clean.go}

    Using e.g. a pointer to \texttt{T} in the definition of \texttt{Foo} would
    make the program valid. We can even change the type constraint of \texttt{T}
    from \texttt{any} to \texttt{Bar}, and the type checker should behave the
    same (since \texttt{Bar} is a subtype of \texttt{any}). However, as of Go
    1.23, such a program is either rejected or accepted by the type checker
    depending on the order of the struct declarations (similar to
    \href{https://github.com/golang/go/issues/65714}{Issue \#65714}).

\end{minipage}

\subsubsection{Type checking derivation with contradiction}

Below, we derive a contradiction while type checking the above program on the
left, showing that the $\notcont$ rule is sufficient for detecting structural
cycles in generic types.

\register{Foo}
\alias{BarType}{Bar}
\register{any}
\aliasparam{T}

\begin{align*}
    D_0    & = \type~\any~\interface~\br{}                       \\
    D_1    & = \type~\Foo[\TParam~\any]~\struct~\br{ x~\TParam } \\
    D_2    & = \type~\BarType~\struct~\br{ x~\Foo[\BarType] }    \\
    \ov{D} & = \{D_0, D_1, D_2\}
\end{align*}

\begin{mathpar}
    \derivrule[T-Type]{
        \axiomrule[T-Interface]{
            \emptyset \vdash \interface~\br{} \ok
        }
        \\
        \axiomrule{
            \notcont(\any, \interface~\br{})
        }
    }{
        D_0 \ok
    }

    \derivrule[T-Named]{
        \axiomrule{
            D_0 \in \ov{D}
        }
    }{
        (\TParam~\any) \vdash \any \ok
    }

    \derivrule[T-Formal]{
        \axiomrule{
            \distinct(\TParam)
        }
        \\
        (\TParam~\any) \vdash \any \ok
    }{
        (\TParam~\any) \ok
    }

    \derivrule[T-Struct]{
        \axiomrule{
            \distinct(f)
        }
        \\
        \derivrule[T-Param]{
            \axiomrule{
                (\TParam~\any) \in (\TParam~\any)
            }
        }{
            (\TParam~\any) \vdash \TParam \ok
        }
    }{
        (\TParam~\any) \vdash \struct~\br{ x~\TParam }  \ok
    }

    \derivrule{
        \axiomrule{
            \notcont(\Foo,\TParam)
        }
    }{
        \notcont(\Foo, \struct~\br{ x~\TParam })
    }

\end{mathpar}
\begin{mathpar}
    \derivrule[T-Type]{
        (\TParam~\any) \ok \\
        (\TParam~\any) \vdash \struct~\br{ x~\TParam }  \ok \\
        \notcont(\Foo, \struct~\br{ x~\TParam })
    }{
        D_1 \ok
    }

    \derivrule[T-Named]{
        D_2 \in \ov{D}
    }{
        \emptyset \vdash \BarType \ok
    }

    \derivrule{
        \eta = (\TParam \by \BarType)
    }{
        \eta = ((\TParam~\any) \by \BarType)
    }

    \derivrule[$\imp_{\text{I}}$]{
        \axiomrule{
            \methods_\emptyset(\BarType) \supseteq \methods_\emptyset(\any)
        }
    }{
        \emptyset \vdash (\BarType\imp \any)
    }

    \derivrule{
        \eta = ((\TParam~\any) \by \BarType)
        \\
        \derivrule{
            \emptyset \vdash (\BarType\imp \any)
        }{
            \emptyset \vdash (\TParam \imp \any)[\eta]
        }
    }{
        \eta = ((\TParam~\any) \by_\emptyset \BarType)
    }

    \derivrule[T-Named]{
        \emptyset \vdash \BarType \ok
        \\
        (\type \Foo[\TParam~\any]) \in \ov{D}
        \\
        \eta = ((\TParam~\any) \by_\emptyset \BarType)
    }{
        \emptyset \vdash \Foo[\BarType] \ok
    }

    \derivrule[T-Struct]{
        \axiomrule{
            \distinct(x)
        }
        \\
        \emptyset \vdash \Foo[\BarType] \ok
    }{
        \emptyset \vdash \struct~\br{ x~\Foo[\BarType] } \ok
    }

    \derivrule{
        \derivrule{
            \derivrule{
                \derivrule{
                    \perp
                }{
                    \BarType \notin \{ \BarType, \Foo \}
                }
            }{
                \notcont(\BarType, \Foo, \BarType)
            }
        }{
            \notcont(\BarType, \Foo, \struct~\br{ x~\BarType })
        }
    }{
        \notcont(\BarType, \Foo, \struct~\br{ x~\TParam }[\eta])
    }

    \derivrule{
        (\type~\Foo[\TParam~\any]~\struct~\br{ x~\TParam }) \in \ov{D}
        \\
        \Foo \notin \{ \BarType \}
        \\
        \eta = ((\TParam~\any) \by \BarType)
        \\
        \notcont(\BarType, \Foo, \struct~\br{ x~\TParam }[\eta])
    }{
        \notcont(\BarType, \Foo[\BarType])
    }

    \derivrule{
        \notcont(\BarType, \Foo[\BarType])
    }{
        \notcont(\BarType, \struct~\br{ x~\Foo[\BarType] })
    }

    \derivrule[T-Type]{
        \emptyset \vdash \struct~\br{ x~\Foo[\BarType] } \ok
        \\
        \notcont(\BarType, \struct~\br{ x~\Foo[\BarType] })
    }{
        D_2 \ok
    }
\end{mathpar}


\subsection{Issue \#51244}

\href{https://github.com/golang/go/issues/51244}{Issue \#51244} demonstrates
another case where the compiler's cycle detection accepts or rejects the program
depending on the order of type declarations. According to the above defined
rules, the program should be accepted. One of the participants of the issue
(\href{https://github.com/findleyr}{findleyr}) also suggested removing the
restriction of cycles through type parameter constraints.

\subsubsection{Accepting type checking derivation}

\golisting{./examples/issue-51244.go}

\noindent
Below is a derivation of the program in the issue, with minor adjustments to
fit a Featherweight Go-like syntax used in this document.

\register{A}
\register{M}
\register{B}
\register{T}

\begin{align*}
    D_0    & = \type~\A~\interface~\br{ \M(b~\B[\T])~\T }    \\
    D_1    & = \type~\B[a~\A]~\struct~\br{ }                 \\
    D_2    & = \type~\T~\struct~\br{ }                       \\
    D_3    & = \func~(t~\T)~\M(b~\B[\T])~\T~\br{~\return~t~} \\
    \ov{D} & = \{D_0, D_1, D_2, D_3\}
\end{align*}

\begin{mathpar}
    \derivrule[T-Interface]{
        \axiomrule{
            \unique(\M(b~\B[\T])~\T)
        }
        \\
        \emptyset \vdash \M(b~\B[\T])~\T \ok
    }{
        \emptyset \vdash \interface~\br{ \M(b~\B[\T])~\T } \ok \\
    }

    \derivrule{
        \derivrule{
            \eta = (a \by \T)
        }{
            \eta = ((a~\A) \by \T)
        }
        \\
        \derivrule{
            \derivrule[$\imp_I$]{
                \derivrule{
                    \axiomrule{
                        \{~\M(b~\B[\T])~\T~\} \supseteq \{~\M(b~\B[\T])~\T~\}
                    }
                }{
                    \methods_\emptyset(\T) \supseteq \methods_\emptyset(\A)
                }
            }{
                \emptyset \vdash T \imp A
            }
        }{
            \emptyset \vdash (a \imp A)\llbracket\eta\rrbracket
        }
    }{
        \eta = ((a~\A) \by_\emptyset \T)
    }

    \derivrule[T-Named]{
        \emptyset \vdash \T \ok
        \\
        (\type~\B[a~\A]~\struct~\br{ }) \in \ov{D}
        \\
        \eta = ((a~\A) \by_\emptyset \T)
    }{
        \emptyset \vdash \B[\T] \ok
    }

    \derivrule[T-Specification]{
        \axiomrule{
            \distinct(b)
        }
        \\
        \emptyset \vdash \B[\T] \ok
        \\
        \emptyset \vdash \T \ok
    }{
        \emptyset \vdash \M(b~\B[\T])~\T \ok
    }
\end{mathpar}

\begin{mathpar}
    \derivrule[T-Type]{
        \emptyset \vdash \interface~\br{ \M(b~\B[\T])~\T } \ok~~~~~
        \axiomrule{
            \notcont(\A,~\interface~\br{ \M(b~\B[\T])~\T })
        }
    }{
        D_0 \ok
    }

    \derivrule[T-Formal]{
        \axiomrule{
            \distinct(a)
        }
        \\
        \derivrule[T-Named]{
            \axiomrule{
                (\type~\A~\interface~\br{ \M(b~\B[\T])~\T }) \in \ov{D}
            }
        }{
            (a~\A) \vdash \A \ok
        }
    }{
        (a~\A) \ok
    }

    \derivrule[T-Type]{
        (a~\A) \ok
        \\
        \axiomrule[T-Struct]{
            (a~\A) \vdash \struct~\br{ } \ok
        }
        \\
        \axiomrule{
            \notcont(\B,~\struct~\br{ })
        }
    }{
        D_1 \ok
    }

    \derivrule[T-Type]{
        \axiomrule[T-Struct]{
            \emptyset \vdash \struct~\br{ } \ok
        }
        \\
        \axiomrule{
            \notcont(\T,~\struct~\br{ })
        }
    }{
        D_2 \ok
    }

    \derivrule{
        \axiomrule{
            (\type~T~\struct~\br{ }) \in \ov{D}
        }
    }{
        \emptyset = \typeparams(T)
    }

    \derivrule[T-Var]{
        \axiomrule{
            (t:\T) \in (t:\T,b:\B[\T])
        }
    }{
        \emptyset;t:\T,b:\B[\T] \vdash t : \T
    }

    \axiomrule[$\imp_V$]{
        \emptyset \vdash T \imp T
    }

    \derivrule[T-Func]{
        \axiomrule{
            \distinct(t, b)
        }
        \\
        \emptyset = \typeparams(T)
        \\
        \emptyset \vdash \M(b~\B[\T])~\T \ok
        \\
        \emptyset;t:\T,b:\B[\T] \vdash t : \T
        \\
        \emptyset \vdash T \imp T
    }{
        D_3 \ok
    }
\end{mathpar}


\subsection{Self-referential type parameter constraints}

The following two examples show the rules applying to types where the type
parameter constraints refer to the type being defined (currently disallowed by
spec). While the type parameters are recursive, the type checking algorithms do
not recurse infinitely. The intuition behind this is that while type checking
some declaration $D$, lookups in the global $\ov{D}$ do not depend on the types
in $\ov{D}$ to have already been type checked. Similarly, when type checking
some type parameter constraints $\ov{\Phi}$, lookups in the same $\ov{\Phi}$ do
not require the type parameter constraints in $\ov{\Phi}$ to have already been
type checked. This is crucial in order to allow for backward and forward
references, and also self-references (recursion).

\subsubsection{Negative example}

In this example, the type \texttt{E} is not well-typed, since it ``flips'' the
bounds between \texttt{Foo} and \texttt{Bar}. A contradiction is reached during
the type checking derivation.

% TODO check what happens in a simpler program with one type Foo, and E:
% type E[F1 Foo[F2], F2 Foo[F1]].

\golisting{./examples/self-ref-invalid.go}

\register{F}
\register{E}

\begin{align*}
    D_0    & = \type~\Foo[\F~\Foo[\F]]~\interface~\br{~m(f~\F)~\F~}         \\
    D_1    & = \type~\BarType[\B~\BarType[\B]]~\interface~\br{~m(b~\B)~\B~} \\
    D_2    & = \type~\E[\F~\Foo[\B],~\B~\BarType[\F]]~\struct~\br{ }        \\
    \ov{D} & = \{D_0, D_1, D_2\}
\end{align*}

\begin{mathpar}
    \derivrule[T-Param]{
        \axiomrule{
            (\F:\Foo[\F]) \in (\F~\Foo[\F])
        }
    }{
        (\F~\Foo[\F]) \vdash \F \ok
    }

    \derivrule{
        \derivrule{
            \eta_0 = (\F \by \F)
        }{
            \eta_2 = ((\F \Foo[\F]) \by \F)
        }
        \\
        \derivrule{
            \derivrule[$\imp_I$]{
                \derivrule{
                    \derivrule{
                        \axiomrule{
                            (\F:\Foo[\F]) \in (\F~\Foo[\F])
                        }
                    }{
                        \methods_{(\F~\Foo[\F])}(\F) = \methods_{(\F~\Foo[\F])}(\Foo[\F])
                    }
                }{
                    \methods_{(\F~\Foo[\F])}(\F) \supseteq \methods_{(\F~\Foo[\F])}(\Foo[\F])
                }
            }{
                (\F~\Foo[\F]) \vdash (\F \imp \Foo[\F])
            }
        }{
            (\F~\Foo[\F]) \vdash (\F \imp \Foo[\F])\llbracket \eta_0 \rrbracket
        }
    }{
        \eta_0 = ((\F~\Foo[\F]) \by_{(\F~\Foo[\F])} \F)
    }

    \derivrule[T-Named]{
        (\F~\Foo[\F]) \vdash \F \ok
        \\
        \axiomrule{
            (\type~\Foo[\F~\Foo[\F]]~\interface~\br{~m(f~\F)~\F~}) \in \ov{D}
        }
        \\
        \eta_0 = ((\F~\Foo[\F]) \by_{(\F~\Foo[\F])} \F)
    }{
        (\F~\Foo[\F]) \vdash \Foo[\F] \ok
    }

    \derivrule[T-Formal]{
        \axiomrule{
            \distinct(F)
        }
        \\
        (\F~\Foo[\F]) \vdash \Foo[\F] \ok
    }{
        (\F~\Foo[\F]) \ok
    }

    \derivrule[T-Specification]{
        \axiomrule{
            \distinct(f)
        }
        \\
        \derivrule[T-Param]{
            \axiomrule{
                (\F:\Foo[\F]) \in (\F~\Foo[\F])
            }
        }{
            (\F~\Foo[\F]) \vdash \F \ok
        }
    }{
        (\F~\Foo[\F]) \vdash m(f~\F)~\F \ok
    }
\end{mathpar}

\begin{mathpar}
    \derivrule[T-Interface]{
        \axiomrule{
            \unique(m(f~\F)~\F)
        }
        \\
        (\F~\Foo[\F]) \vdash m(f~\F)~\F \ok
    }{
        (\F~\Foo[\F]) \vdash \interface~\br{~m(f~\F)~\F~} \ok
    }

    \axiomrule{
        \notcont(Foo,~\interface~\br{~m(f~\F)~\F~})
    }

    \derivrule[T-Type]{
        (\F~\Foo[\F]) \ok
        \\
        (\F~\Foo[\F]) \vdash \interface~\br{~m(f~\F)~\F~} \ok
        \\
        \notcont(Foo,~\interface~\br{~m(f~\F)~\F~})
    }{
        D_0 \ok
    }
\end{mathpar}

An analogous tree can be derived for $D_1$ (\texttt{Bar} is isomorphic to
\texttt{Foo}).

\begin{mathpar}
    \derivrule[T-Param]{
        \axiomrule{
            (\B~\BarType[\F]) \in (\F~\Foo[\B],~\B~\BarType[\F])
        }
    }{
        (\F~\Foo[\B],~\B~\BarType[\F]) \vdash \B \ok
    }

    \axiomrule{
        (\type~\Foo[\F~\Foo[\F]]~\interface~\br{~m(f~\F)~\F~}) \in \ov{D}
    }

    \derivrule{
        (\B:\BarType[\F]) \in (\F~\Foo[\B],~\B~\BarType[\F])
    }{
        \methods_{(\F~\Foo[\B],~\B~\BarType[\F])}(\B) = \methods_{(\F~\Foo[\B],~\B~\BarType[\F])}(\BarType[\F])
    }

    \derivrule{
        \derivrule{
            \axiomrule{
                (\type~\BarType[\B~\BarType[\B]]~\interface~\br{~m(b~\B)~\B~}) \in \ov{D}
            }
            \\
            \derivrule{
                \eta_3 = (\B \by \F)
            }{
                \eta_3 = (\B~\BarType[\B] \by \F)
            }
        }{
            \methods_{(\F~\Foo[\B],~\B~\BarType[\F])}(\BarType[\F]) = \br{~m(b~\B)~\B~}\llbracket \eta_3 \rrbracket
        }
    }{
        \methods_{(\F~\Foo[\B],~\B~\BarType[\F])}(\BarType[\F]) = \br{~m(b~\F)~\F~}
    }

    \derivrule{
        \derivrule{
            \axiomrule{
                \type~\Foo[\F~\Foo[\F]]~\interface~\br{~m(f~\F)~\F~}
            }
            \\
            \derivrule{
                \eta_2 = (\F \by \B)
            }{
                \eta_2 = ((\F~\Foo[\F]) \by \B)
            }
        }{
            \methods_{(\F~\Foo[\B],~\B~\BarType[\F])}(\Foo[\B]) = \br{~m(f~\F)~\F~}\;\llbracket \eta_2 \rrbracket
        }
    }{
        \methods_{(\F~\Foo[\B],~\B~\BarType[\F])}(\Foo[\B]) = \br{~m(f~\B)~\B~}
    }

    \derivrule{
        \derivrule[$\imp_I$]{
            \derivrule{
                \derivrule{
                    \derivrule{
                        \bot
                    }{
                        \br{~m(b~\F)~\F~} \supseteq \br{~m(f~\B)~\B~}
                    }
                }{
                    \methods_{(\F~\Foo[\B],~\B~\BarType[\F])}(\BarType[\F]) \supseteq \methods_{(\F~\Foo[\B],~\B~\BarType[\F])}(\Foo[\B])
                }
            }{
                \methods_{(\F~\Foo[\B],~\B~\BarType[\F])}(\B) \supseteq \methods_{(\F~\Foo[\B],~\B~\BarType[\F])}(\Foo[\B])
            }
        }{
            (\F~\Foo[\B],~\B~\BarType[\F]) \vdash (\B \imp \Foo[\B])
        }
    }{
        (\F~\Foo[\B],~\B~\BarType[\F]) \vdash (\F \imp \Foo[\B])\llbracket \eta_2 \rrbracket
    }
\end{mathpar}

\begin{mathpar}
    \derivrule{
        \derivrule{
            \eta_2 = (\F \by \B)
        }{
            \eta_2 = ((\F~\Foo[\F]) \by \B)
        }
        \\
        (\F~\Foo[\B],~\B~\BarType[\F]) \vdash (\F \imp \Foo[\B])\llbracket \eta_2 \rrbracket
        \\
        (\F~\Foo[\B],~\B~\BarType[\F]) \vdash (\B \imp \BarType[\F])\llbracket \eta_2 \rrbracket
    }{
        \eta_2 = ((\F~\Foo[\F]) \by_{(\F~\Foo[\B],~\B~\BarType[\F])} \B)
    }
    \derivrule[T-Named]{
        (\F~\Foo[\B],~\B~\BarType[\F]) \vdash \B \ok
        \\
        (\type~\Foo[\F~\Foo[\F]]~\interface~\br{~m(f~\F)~\F~}) \in \ov{D}
        \\
        \eta_2 = ((\F~\Foo[\F]) \by_{(\F~\Foo[\B],~\B~\BarType[\F])} \B)
    }{
        (\F~\Foo[\B],~\B~\BarType[\F]) \vdash \Foo[\B] \ok
    }

    \derivrule[T-Formal]{
        \axiomrule{
            \distinct(\F,\B)
        }
        \\
        (\F~\Foo[\B],~\B~\BarType[\F]) \vdash \Foo[\B] \ok
        \\
        (\F~\Foo[\B],~\B~\BarType[\F]) \vdash \BarType[\F] \ok
    }{
        (\F~\Foo[\B],~\B~\BarType[\F]) \ok
    }

    \derivrule[T-Type]{
        \axiomrule{
            (\F~\Foo[\B],~\B~\BarType[\F]) \vdash \struct~\br{ } \ok
        }
        \\
        \axiomrule{
            \notcont(\E,~\struct~\br{ })
        }
        \\
        (\F~\Foo[\B],~\B~\BarType[\F]) \ok
    }{
        D_2 \ok
    }
\end{mathpar}


\subsubsection{Positive example}

\golisting{./examples/self-ref-valid.go}

\begin{align*}
    D_0    & = \type~\Foo[\F~\Foo[\F]]~\interface~\br{~m(f~\F)~\F~}         \\
    D_1    & = \type~\BarType[\B~\BarType[\B]]~\interface~\br{~m(b~\B)~\B~} \\
    D_2    & = \type~\E[\F~\Foo[\F],~\B~\BarType[\B]]~\struct~\br{ }        \\
    D_3    & = \type~\T~\struct~\br{ }                                      \\
    D_4    & = \func~(t~\T)~m(t2~\T)~\T~\br{~\return~t~}                    \\
    \ov{D} & = \{D_0, D_1, D_2, D_3, D_4\}                                  \\
    e      & = \E[\T,~\T]\br{}
\end{align*}

The derivations for $D_0$, $D_1$, and $D_3$ were shown in previous examples.

\begin{mathpar}
    \derivrule[T-Specification]{
        \axiomrule{
            \distinct(t2)
        }
        \\
        \derivrule[T-Named]{
            (\type~\T~\struct~\br{ }) \in \ov{D}
        }{
            \emptyset \vdash \T \ok
        }
    }{
        \emptyset \vdash m(t2~\T)~\T \ok
    }

    \derivrule[T-Var]{
        \axiomrule{
            (t:\T) \in (t:\T,t2:\T)
        }
    }{
        \emptyset;t:\T,t2:\T \vdash t: \T
    }

    \derivrule[T-Func]{
        \axiomrule{
            \distinct(t,t2)
        }
        \\
        \axiomrule{
            \emptyset = \typeparams(\T)
        }
        \\
        \emptyset \vdash m(t2~\T)~\T \ok
        \\
        \emptyset;t:\T,t2:\T \vdash t: \T
        \\
        \axiomrule[$\imp_V$]{
            \T \imp \T
        }
    }{
        D_4 \ok
    }
\end{mathpar}

$(\F~\Foo[\F],~\B~\BarType[\B]) \vdash \Foo[\F] \ok$ can be derived analogously
to $(\F~\Foo[\F]) \vdash \Foo[\F] \ok$ in the previous example. The same applies
for $(\F~\Foo[\F],~\B~\BarType[\B]) \vdash \BarType[\B] \ok$.

\begin{mathpar}
    \derivrule[T-Formal]{
        \axiomrule{
            \distinct(\F,\B)
        }
        \\
        (\F~\Foo[\F],~\B~\BarType[\B]) \vdash \Foo[\F] \ok
        \\
        (\F~\Foo[\F],~\B~\BarType[\B]) \vdash \BarType[\B] \ok
    }{
        (\F~\Foo[\F],~\B~\BarType[\B]) \ok
    }

    \derivrule[T-Type]{
        (\F~\Foo[\F],~\B~\BarType[\B]) \ok
        \\
        \axiomrule{
            \F~\Foo[\F],~\B~\BarType[\B] \vdash \struct\br{} \ok
        }
        \\
        \axiomrule{
            \notcont(\E,~\struct\br{})
        }
    }{
        D_2 \ok
    }

    \derivrule{
        \axiomrule{
            \emptyset = \typeparams(\T)
        }
    }{
        \methods_\emptyset(T) = \br{~m(t2~\T)~\T~}
    }

    \derivrule{
        \derivrule{
            (\type~\Foo[\F~\Foo[\F]]~\interface~\br{~m(f~\F)~\F~}) \in \ov{D}
            \\
            \derivrule{
                \eta_5 = (\F \by \T)
            }{
                \eta_5 = (\F~\Foo[\F] \by \T)
            }
        }{
            \methods_\emptyset(\Foo[\T]) = \br{~m(f~\F)~\F~} \llbracket \eta_5 \rrbracket
        }
    }{
        \methods_\emptyset(\Foo[\T]) = \br{~m(f~\T)~\T~}
    }

\end{mathpar}

$\emptyset \vdash (\T \imp \BarType[\T])$ can be derived analogously to $\emptyset \vdash (\T \imp \Foo[\T])$.

\begin{mathpar}

    \derivrule{
        \derivrule[$\imp_I$]{
            \derivrule{
                \axiomrule{
                    \br{~m(t2~\T)~\T~} \supseteq \br{~m(f~\T)~\T~}
                }
            }{
                \methods_\emptyset(T) \supseteq \methods_\emptyset(\Foo[\T])
            }
        }{
            \emptyset \vdash (\T \imp \Foo[\T])
        }
    }{
        \emptyset \vdash (\F \imp \Foo[\F])[\eta_4]
    }

    \derivrule{
        \emptyset \vdash (\T \imp \BarType[\T])
    }{
        \emptyset \vdash (\B \imp \BarType[\B])[\eta_4]
    }

    \derivrule{
        \derivrule{
            \eta_4 = (\F \by \T,~\B \by \T)
        }{
            \eta_4 = (\F~\Foo[\F] \by \T,~\B~\BarType[\B] \by \T)
        }
        \\
        \emptyset \vdash (\F \imp \Foo[\F])[\eta_4]
        ~~~~
        \emptyset \vdash (\B \imp \BarType[\B])[\eta_4]
    }{
        \eta_4 = (\F~\Foo[\F] \by_\emptyset \T,~\B~\BarType[\B] \by_\emptyset \T)
    }

    \derivrule[T-Named]{
        \emptyset \vdash \T \ok
        \\
        \axiomrule{
            (\type~\E[\F~\Foo[\F],~\B~\BarType[\B]]~\struct~\br{ }) \in \ov{D}
        }
        \\
        \eta_4 = (\F~\Foo[\F] \by_\emptyset \T,~\B~\BarType[\B] \by_\emptyset \T)
    }{
        \emptyset \vdash \E[\T,\T] \ok
    }

    \derivrule[T-Struct-Literal]{
        \emptyset \vdash \E[\T,\T] \ok
    }{
        \emptyset;\emptyset \vdash \E[\T,\T]\br{} : \E[\T,\T]
    }
\end{mathpar}


\section{Remaining rules}
\label{sec:remaining-rules}

\subsection{Syntax}

\begin{minipage}[t]{\textwidth}
    \begin{tabular}[t]{ll}
        Structure type name      & $t_S, u_S$                                           \\
        Interface type name      & $t_I, u_I$                                           \\
        Array type name          & $t_A, u_A$                                           \\
        Value type name          & $t_V, u_V$ ::= $t_S \mid t_A$                        \\
        Type name                & $t, u$ ::= $t_V \mid t_I$                            \\
        Declaration              & $D$ ::=                                              \\
        \quad Type declaration   & \quad $\type~t[\ov{\Phi}]~T$                         \\
        \quad Method declaration & \quad $\func~(x~t_V[\ov{\alpha}])~mM~\br{\return~e}$ \\
    \end{tabular}
\end{minipage}
\hspace{-0.5\textwidth}
\begin{minipage}[t]{0.4\textwidth}
    \begin{tabular}[t]{ll}
        Structure type & $\tau_S, \sigma_S$ ::= $t_S[\ov{\tau}]$     \\
        Interface type & $\tau_I, \sigma_I$ ::= $t_I[\ov{\tau}]$     \\
        Array type     & $\tau_A, \sigma_A$ ::= $t_A[\ov{\tau}]$     \\
        Value type     & $\tau_V, \sigma_V$ ::= $\tau_S \mid \tau_A$ \\
    \end{tabular}
\end{minipage}

Well-formed type formals
\hfill \fbox{$\ov{\Phi} \ok$}
\begin{mathpar}
    \inferrule[T-Formal]
    {
        (\ov{\alpha~\gamma}) = \ov{\Phi} \\
        \distinct(\ov{\alpha}) \\
        \ov{\Phi} \vdash \ov{\gamma \ok}
    }
    { \ov{\Phi} \ok}
\end{mathpar}

Well-formed type
\hfill \fbox{$\Delta \vdash \tau \ok$}
\begin{mathpar}
    \inferrule[t-param]
    { (\alpha : \gamma) \in \Delta }
    { \Delta \vdash \alpha \ok }

    \inferrule[t-named]
    {
        \black{\Delta \vdash \ov{\tau \ok}}
        \and
        (\type~t\black{[\ov{\Phi}]}~T) \in \ov{D}
        \and
        \black{\eta = (\ov{\Phi \by_\Delta \tau})}
    }
    { \black{\Delta \vdash}~t\black{[\ov{\tau}]} \ok }
\end{mathpar}

\begin{mathpar}
    \inferrule
    {
        (\ov{\alpha~\gamma}) = \ov{\Phi} \\
        \eta = (\ov{\alpha \by \tau})
    }
    {(\ov{\Phi \by \tau}) = \eta}

    \inferrule
    {
        (\ov{\alpha~\gamma}) = \ov{\Phi} \\
        \eta = (\ov{\Phi \by \tau}) \\
        \Delta \vdash (\ov{\alpha \imp \gamma})\llbracket\eta\rrbracket \\
    }
    {(\ov{\Phi \by_\Delta \tau}) = \eta}
\end{mathpar}

Well-formed method specifications and type literals
\hfill \fbox{$\ov{\Phi} \vdash S \ok$} \qquad \fbox{$\ov{\Phi} \vdash T \ok$}
\begin{mathpar}
    \inferrule[t-specification]
    {
        \distinct(\ov{x}) \\
        \black{\ov{\Phi} \vdash \ov{\tau \ok}}\\
        \black{\ov{\Phi} \vdash \tau \ok}\\
    }
    { \black{\ov{\Phi} \vdash}~m(\ov{x~\black{\tau}})~\black{\tau} \ok }

    \inferrule[t-struct]
    {
        \distinct(\ov{f}) \\
        \black{\ov{\Phi} \vdash \ov{\tau \ok}}\\
    }
    { \black{\ov{\Phi} \vdash}~\struct~\br{\ov{f~\black{\tau}}} \ok }

    \inferrule[t-interface]
    {
        \unique(\ov{S}) \\
        \black{\ov{\Phi} \vdash}~\ov{S \ok}
    }
    { \black{\ov{\Phi} \vdash}~\interface~\br{\ov{S}} }

    \inferrule[t-array]
    {
    \black{\ov{\Phi} \vdash \tau \ok}\\
    }
    {
    \ov{\Phi} \vdash~[n]\tau \ok
    }
\end{mathpar}

Well-formed method declarations \hfill \fbox{$D \ok$}
\begin{mathpar}
    \inferrule[t-func]
    {
        \distinct(x, \ov{x}) \\
        \black{\ov{\Phi} = \typeparams(t_V)}\\
        \black{(\ov{\alpha~\gamma}) = \ov{\Phi}}\\
        \black{\ov{\Phi} \vdash}~m(\ov{x~\black{\tau}})~\black{\sigma} \ok \\
        \black{\ov{\Phi} \stoup}
        x : t_V\black{[\ov{\alpha}]} \comma \ov{x : \black{\tau}} \vdash e : \black{\tau} \\
        \black{\ov{\Phi} \vdash \tau} \imp \black{\sigma} \\
    }
    { \func~(x~t_V\black{[\ov{\alpha}]})~m(\ov{x~\black{\tau}})~\black{\sigma}~\br{\return~e} \ok }
\end{mathpar}

Implements
\hfill \fbox{$\Delta \vdash \tau \imp \sigma$}
\begin{mathpar}

    \inferrule[$\imp_{\text{Param}}$]
    { ~ }
    { \Delta \vdash \alpha \imp \alpha }

    \inferrule[$\imp_V$]
    { ~ }
    { \black{\Delta \vdash}~\tau_V \imp \tau_V }

    \inferrule[$\imp_I$]
    {
        \methods_\black{\Delta}(\black{\tau}) \supseteq \methods_\black{\Delta}(\black{\tau}_I) \\
    }
    { \black{\Delta \vdash}~\tau \imp \tau_I }

    \inferrule
    {
        \black{ \eta = (\ov{\Phi \by \tau}) }
        \\
        \black{ \ov{\Phi} = \typeparams(t_V) }
    }
    {
        \methods_\black{\Delta}(t_V\black{[\ov{\tau}]}) =
        \{(mM)\black{\llbracket\eta\rrbracket}
        \mid (\func~(x~t_V\black{[\ov{\alpha}]})~mM~\br{\return~e}) \in \ov{D}
        \}
    }

    \inferrule
    {
        \type~t_I\black{[\ov{\Phi}]}~\interface~\br{\ov{S}} \in \ov{D} \\
        \black{ \eta = (\ov{\Phi \by \tau})}}
    {
        \methods_\black{\Delta}(t_I\black{[\ov{\tau}]}) =
        \ov{S}\black{\llbracket\eta\rrbracket}
    }

    \inferrule
    {(\alpha : \black{\gamma}) \in \Delta}
    {\methods_\Delta(\alpha) = \methods_\Delta(\gamma)}

    \inferrule
    {(\type~t[\ov{\Phi}]~T) \in \ov{D}}
    {\ov{\Phi} = \typeparams(t)}
\end{mathpar}

Expressions \hfill \fbox{$\Delta \stoup \Gamma \vdash e : \tau$}
\begin{mathpar}
    \inferrule[t-var]
    {
        (x : \black{\tau}) \in \Gamma
    }
    { \black{\Delta \stoup}~\Gamma \vdash x : \black{\tau} }

    \inferrule[t-struct-literal]
    {
        \black{\Delta \vdash \tau}_S \ok
        \\
        \black{\Delta \stoup}~\Gamma \vdash \ov{e : \black{\tau}}
        \\
        (\ov{f~\black{\sigma}}) = \fields(\black{\tau}_S)
        \\
        \black{\Delta \vdash}~\ov{\black{\tau} \imp \black{\sigma}}
    }
    { \black{\Delta \stoup}~\Gamma \vdash \black{\tau}_S\br{\ov{e}} : \black{\tau}_S
    }
\end{mathpar}

\end{document}
