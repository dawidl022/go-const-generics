\section{Monormorphisation from FGGA to Go}
\label{ch:monomo}

Monormorphisation is an implementation technique for generics (parametric
polymorphism) which involves translating generic code into non-generic code. For
every instantiation of a generic type, a non-generic type is produced in the
output monomorphised program. This approach allows for zero-cost abstractions,
as there is no runtime penalty for executing monomorphised code as opposed to
non-generic code. Monormorphisation is used by languages such as Rust to
implement generics \autocite{rustCompilerGuide}. Monormorphisation was first
formalised in \emph{Featherweight Go} \autocite{fg}.

There are, however, certain limitations with this approach. Monormorphisation
can lead to an explosion in the output code size, and not all programs can be
monomorphised \autocite{fg}. Other approaches have been adopted by many
languages that support parametric polymorphism. Java uses type erasure
\autocite{javaSpec}, C\# uses a hybrid of monomorphisation at runtime (as
opposed to the traditional compile-time) and code sharing
\autocite{clrGenerics}, and Go uses a hybrid of monormorphisation and dictionary
passing \autocite{generics1.18}.

Monormorphisation is appropriate for the restricted use of numerical type
parameters in FGGA, since output code size explosion is not an issue --- for
every concrete type argument $n$, at most 1 output type will be produced for
every type depending on the aforementioned parameter $n$ (directly or
indirectly).

This section presents a formalisation of monomorphisation from FGGA to Go, based
on \emph{Featherweight Go} \autocite{fg}. Because regular type parameters are
now part of Go, they are output as is, and only numerical type parameters are
``eliminated'' in the monomorphisation process.

\subsection{Formalisation}

$\Delta$ is an environment mapping numerical type parameters $\alpha_n$ (i.e.
those bound by \kw{const}) to integer literals $n$.

The monomorphisation process consists of two stages --- type collection and type
translation. The type collection phase begins by collecting all the referenced
named types in the main expression. The types are collected into the set
$\omega$, with one entry for every (type name, sequence of integers) pair. The
sequence of integers in the pair is the named type's type argument list, with
all non integer-value arguments removed.

After the initial collection, we apply the function $G$ on $\omega$, which
collects further types from type declarations ($\Tclo$) and methods
($\Mclo$ and $\ASMclo$) of the types in $\omega$.

This process is repeated, applying $G$ to the output of the previous $G$
application until the fixed point of $G$ is reached, i.e. when applying $G$
produces an output that is the same as its input. This notion is captured in the
I-Prog rule, where $\Omega$ is the fixed point.

Because only referenced types are collected, a side effect of monomorphisation
is that any unused types are eliminated. This makes sense, because in general,
if a generic type is never instantiated, we don't have any candidates it could
monomorphise down to.

The second phase translates the main expression, and the collected types in
$\Omega$ into valid Go code. The main task of this phase is to move all
numerical type arguments from the collected types into the names of the named
types (rule M-Named), and remove any numerical type parameters or arguments from
type parameter/argument lists while preserving any non-numerical type
parameters/arguments (rules M-Named, M-$\alpha$ and M-Constraints). The
remainder of the rules either recursively apply type translation to their
components or are base cases that require no translation at all.

The formal rules describe the process in two distinct stages, whereas the
implementation does both in a single stage (translating as it collects types).
All the information needed to translate a term can be derived from the result of
type collection on that term. E.g. given $\Delta = \emptyset$ the expression $e
    = Arr[2, \kw{int}]\{1, 2\}$, we know it produces $\omega = \{Arr[2]\}$ and
translates to $e^\dagger = \an{Arr, 2}[\kw{int}]\{1, 2\}$, where $\an{t,
        \ov{n}}$ signifies the output type name. In an implementation, the name should
be generated such that ideally it doesn't conflict with any other type name that
the programmer might declare. Additionally, we assume the program has already
been type-checked before being fed into the monomorphiser (the implemenation
should type-check before monomorphising).

$\Delta$ is an environment mapping numerical type parameters $\alpha_n$ (i.e.
those bound by \kw{const}) to integer literals $n$.

% Formal rules describe process in two distinct stages, whereas implementation
% does both in a single stage (translates as it collects types).

% From the symmetry of the formal rules of the two phases it is clear how this
% can naturally be implemented in a single phase.

\begin{figure}
    Type-instance sets \hfill \fbox{$\omega, \Omega$}
    \begin{mathpar}
        \omega, \Omega
        \text{ range over sets containing elements of the form }
        t[\ov{n}]
    \end{mathpar}

    Expressions \hfill \fbox{$ \Delta \vdash e \yields \omega$}

    \begin{mathpar}
        \inferrule[I-int-literal]{~}
        {\Delta \vdash n \yields \emptyset}

        \inferrule[I-var]
        {
            ~
        }
        {
            \Delta \vdash x \yields \emptyset
        }

        \inferrule[I-literal]
        {
        \Delta \vdash t_V[\ov{\tau}] \yields \omega_\tau
        \\
        \Delta \vdash \ov{e \yields \omega}
        }
        {
        \Delta \vdash  t_V[\ov{\tau}]\br{\ov{e}}\yields
        \ov{\omega_\tau}~
        \cup~\ov{\omega}
        }

        \inferrule[I-field]
        {
            \Delta \vdash e \yields \omega
        }
        {
            \Delta \vdash e.f \yields \omega
        }

        \inferrule[I-index]
        {
            \Delta \vdash e \yields \omega
            \\
            \Delta \vdash e' \yields \omega'
        }
        {
            \Delta \vdash e[e'] \yields \omega~\cup~\omega'
        }

        \inferrule[I-call]
        {
            \Delta \vdash e \yields \omega\\
            \Delta \vdash \ov{e \yields \omega}
        }
        {
            \Delta \vdash e.m(\ov{e}) \yields
            \omega~\cup~\ov{\omega}
        }
    \end{mathpar}

    Method specifications and type literals
    \hfill \fbox{$\Delta \vdash S \yields \omega $}
    \quad \fbox{$\Delta \vdash T \yields \omega $}

    \begin{mathpar}
        \inferrule[I-specification]{
            \Delta \vdash \ov{\tau \yields \omega}
            \\
            \Delta \vdash \tau \yields \omega
        }{
            \Delta \vdash m(\ov{x~\tau})~\tau \yields \omega~\cup~\ov{\omega}
        }

        \inferrule[I-struct]{
            \Delta \vdash \ov{\tau \yields \omega}
        }{
            \Delta \vdash \struct~\br{\ov{f~\tau}} \yields \ov{\omega}
        }

        \inferrule[I-array]{
            \Delta \vdash \tau \yields \omega
        }{
            \Delta \vdash [\tau_n]\tau \yields \omega
        }

        \inferrule[I-interface]{
            \Delta \vdash \ov{S \yields \omega}
        }{
            \Delta \vdash \interface~\br{\ov{S}} \yields \ov{\omega}
        }
    \end{mathpar}

    Types \hfill \fbox{$\Delta \vdash \tau \yields \omega $}

    \begin{mathpar}
        \inferrule[I-int]{~}{
            \Delta \vdash \kw{int} \yields \emptyset
        }

        \inferrule[I-n]{~}{
            \Delta \vdash n \yields \emptyset
        }

        \inferrule[I-$\alpha$]{~}{
            \Delta \vdash \alpha \yields \emptyset
        }

        \inferrule[I-const]{~}{
            \Delta \vdash \kw{const} \yields \emptyset
        }

        \inferrule[I-named]{
            \ov{\tau \yields \omega_\tau}
        }{
            \Delta \vdash t[\ov{\tau}] \yields
            \set{\instance(t[\ov{\tau}]\llbracket\Delta\rrbracket)}
            \cup \ov{\omega_\tau}
        }
    \end{mathpar}

    Programs  \hfill \fbox{$P \yields \Omega$}

    \begin{mathpar}
        \inferrule[I-prog]
        {
            \emptyset \vdash e \yields \omega \\
            \Omega = \lim_{n \rightarrow \infty}  G^{n}(\omega)
        }
        {
            \package~\main;~\ov{D}~\func~\main()~\br{\un=e} \yields \Omega
        }
    \end{mathpar}

\end{figure}

\begin{figure}

    Auxiliary functions

    \begin{mathpar}
        \inferrule
        {
            \ov{n} = \ov{n : n \in \ov{\tau}}
        }
        {
            \instance(t[\ov{\tau}]) = t[\ov{n}]
        }

        \inferrule
        {
            \ov{\alpha_n} = \ov{\alpha: \alpha~\kw{const} \in \ov{\Phi}}
        }
        {
            (\ov{\Phi} \by \ov{n}) = (\ov{\alpha_n \by n})
        }
    \end{mathpar}

    \begin{mathpar}
        G(\omega) = \TExtensionD{\omega}{\Delta} \cup
        \MExtensionD{\omega}{\Delta} \cup \ASMExtensionD{\omega}{\Delta}

        \TExtensionD{\omega}{\Delta}   = \bigcup\left\{
        \ov{\omega'}\cup\omega''
        \;\middle|\;
        \begin{array}{c}
            t[\ov{n}] \in \omega
            \comma
            (\type~t[\ov{\Phi}]~T) \in \ov{D}
            \comma
            \ov{\Phi} = (\ov{\alpha~\gamma})
            \comma
            \\
            \ov{(\ov{\Phi} \by \ov{n}) \vdash \gamma \yields \omega' }
            \comma
            (\ov{\Phi} \by \ov{n}) \vdash T \yields \omega''
        \end{array}
        \right\}
        \\
        \MExtensionD{\omega}{\Delta}   = \bigcup\left\{
        \omega'\cup\omega''
        \;\middle|\;
        \begin{array}{c}
            t_V[\ov{n}] \in \omega
            \comma
            (\func~(x~t_V[\ov{\alpha}])~mM~\br{\return~e}) \in \ov{D}
            \comma
            \\
            \ov{\Phi} = \typeparams(t_V)
            \comma
            (\ov{\Phi} \by \ov{n}) \vdash mM \yields \omega'
            \comma
            (\ov{\Phi} \by \ov{n}) \vdash e \yields \omega''
        \end{array}
        \right\}
        \\
        \ASMExtensionD{\omega}{\Delta}  = \bigcup\left\{
        \omega'
        \;\middle|\;
        \begin{array}{c}
            t_V[\ov{n}] \in \omega
            \comma
            \ov{\Phi} = \typeparams(t_V)
            \comma
            (\ov{\Phi} \by \ov{n}) \vdash \tau \yields \omega'
            \comma
            \\
            (\func~(x~t_V[\ov{\alpha}]) ~m(x_1~\kw{int},~x_2~\tau) ~t_V[\ov{\alpha}]~
            \br{ x[x_1] = x_2;~\return~x }) \in \ov{D}
        \end{array}
        \right\}
    \end{mathpar}

    \caption{Type collection phase of FGGA to Go monormorphisation}
\end{figure}

\begin{figure}

    Expressions \hfill \fbox{$ \Delta \vdash e \mapsto e^\dagger $}

    \begin{mathpar}
        \inferrule[m-int-literal]
        {~}
        {
            \Delta \vdash n \mapsto n
        }

        \inferrule[m-var]
        {~}
        {
            \Delta \vdash x \mapsto x
        }

        \inferrule[m-literal]
        {
            \Delta \vdash \tau_V \mapsto \tau_V^\dagger
            \\
            \Delta \vdash \ov{e \mapsto e^\dagger}
        }
        {
            \Delta \vdash  \tau_V\br{\ov{e}}
            \mapsto
            \tau_V^\dagger\br{\ov{e^\dagger}}
        }

        \inferrule[m-field]
        {
            \Delta \vdash e \mapsto e^\dagger
        }
        {
            \Delta \vdash e.f \mapsto e^\dagger.f
        }

        \inferrule[m-index]
        {
            \Delta \vdash e \mapsto e^\dagger
            \\
            \Delta \vdash e' \mapsto e'^\dagger
        }
        {
            \Delta \vdash e[e'] \mapsto e^\dagger[e'^\dagger]
        }

        \inferrule[m-call]
        {
            \Delta \vdash e \mapsto e^\dagger
            \\
            \Delta \vdash \ov{e \mapsto e^\dagger}
        }
        {
            \Delta \vdash e.m(\ov{e}) \mapsto
            e^\dagger.m(\ov{e^\dagger})
        }
    \end{mathpar}

    Method signatures and type literals
    \hfill \fbox{$\Delta \vdash M \mapsto M^\dagger $}
    \quad \fbox{$\Delta \vdash T \mapsto T^\dagger $}

    \begin{mathpar}
        \inferrule[m-signature]{
            \Delta \vdash \ov{\tau \mapsto \tau^\dagger}
            \\
            \Delta \vdash \tau \mapsto \tau^\dagger
        }{
            \Delta \vdash (\ov{x~\tau})~\tau \mapsto
            (\ov{x~\tau^\dagger})~\tau^\dagger
        }

        \inferrule[m-struct]{
            \Delta \vdash \ov{\tau \mapsto \tau^\dagger}
        }{
            \Delta \vdash \struct~\br{\ov{f~\tau}} \mapsto
            \struct~\br{\ov{f~\tau^\dagger}}
        }

        \inferrule[m-array]{
            \Delta \vdash \tau_n \mapsto \tau_n^\dagger
            \\
            \Delta \vdash \tau \mapsto \tau^\dagger
        }{
            \Delta \vdash [\tau_n]\tau \mapsto [\tau^\dagger_n]\tau^\dagger
        }

        \inferrule[m-interface]{
            \Delta \vdash \ov{M \mapsto M^\dagger}
        }{
            \Delta \vdash \interface~\br{\ov{mM}} \mapsto
            \interface~\br{\ov{mM^\dagger}}
        }
    \end{mathpar}

    Types \hfill \fbox{$\Delta \vdash \tau \mapsto \tau^\dagger $}

    \begin{mathpar}
        \inferrule[m-int]{~}{
            \Delta \vdash \kw{int} \mapsto \kw{int}
        }

        \inferrule[m-n]{~}{
            \Delta \vdash n \mapsto n
        }

        \inferrule[m-$\alpha$]{
            ~
        }{
            \Delta \vdash \alpha \mapsto \alpha\llbracket\Delta\rrbracket
        }

        \inferrule[m-named]{
            \Delta \vdash \ov{\tau \mapsto \tau^\dagger}
            \\
            \ov{n} = \ov{n: n \in \ov{\tau^\dagger}}
            \\
            \ov{\tau_c^\dagger} = \ov{
                \tau: (\tau \in \ov{\tau^\dagger}) \wedge (\tau \notin \ov{n})
            }
        }{
            \Delta \vdash t[\ov{\tau}] \mapsto
            \an{t, \ov{n}}[\ov{\tau_c^\dagger}]
        }
    \end{mathpar}

    Type parameter constraints \hfill \fbox{$\Delta \vdash \ov{\Phi} \mapsto \ov{\Phi}^\dagger $}

    \begin{mathpar}
        \inferrule[m-constraints]
        {
            \Delta \vdash \ov{\gamma \mapsto \gamma^\dagger}
            \\
            \ov{\Phi}^\dagger = \ov{(\alpha~\gamma^\dagger):
                ((\alpha~\gamma) \in \ov{\Phi}) \wedge (\gamma \neq \const)}
        }
        {
            \Delta \vdash \ov{\Phi} \mapsto \ov{\Phi}^\dagger
        }
    \end{mathpar}

\end{figure}

\begin{figure}

    Program \hfill \fbox{$P \mapsto P^\dagger$}

    \begin{mathpar}
        \inferrule[m-program]{
            \package~\main;~\ov{D}~\func~\main()~\br{\un=e} \yields \Omega
            \\
            \emptyset \vdash e \mapsto e^\dagger
        }
        {
            \package~\main;~\ov{D}~\func~\main()~\br{\un=e}
            \mapsto
            \package~\main;~\calD(\Omega)~\func~\main()~\br{\un=e^\dagger}
        }
    \end{mathpar}

    Auxiliary functions

    \begin{mathpar}
        \inferrule{
            \ov{\Phi} = (\ov{\alpha~\gamma})
            \\
            \ov{\alpha_c} = \ov{\alpha: (\alpha \in \ov{\alpha}) \wedge
                ((\alpha~\const) \notin \ov{\Phi})}
        }{
            \ov{\Phi} \by \ov{\alpha_c}
        }
        \\
        \calD(\Omega) = \calD_T(\Omega) \cup \calD_M(\Omega) \cup \calD_A(\Omega)
        \\
        \calD_T(\Omega) = \left\{
        \inferrule[]{
        t[\ov{n}] \in \Omega
        \\
        (\type~t[\ov{\Phi}]~T) \in \ov{D}
        \\
        (\ov{\Phi} \by \ov{n}) \vdash \ov{\Phi} \mapsto \ov{\Phi}^\dagger
        \\
        (\ov{\Phi} \by \ov{n}) \vdash T \mapsto T^\dagger
        }{
        \type~\an{t, \ov{n}}[\ov{\Phi}^\dagger]~T^\dagger
        }
        \right\}

        \calD_M(\Omega) = \left\{
        \inferrule[]{
        t_V[\ov{n}] \in \Omega
        \\
        (\func~(x~t_V[\ov{\alpha}])~mM~\br{\return~e}) \in \ov{D}
        \\
        \ov{\Phi} = \typeparams(t_V)
        \\
        \ov{\Phi} \by \ov{\alpha_c}
        \\
        (\ov{\Phi} \by \ov{n}) \vdash M \mapsto M^\dagger
        \\
        (\ov{\Phi} \by \ov{n}) \vdash e \mapsto e^\dagger
        }{
        \func (x~ \an{t_V, \ov{n}}[\ov{\alpha_c}])~mM^\dagger~\br{ \return~e^\dagger }
        }
        \right\}

        \calD_A(\Omega) = \left\{
        \inferrule[]{
        t_V[\ov{n}] \in \Omega
        \\
        (\func~(x~t_V[\ov{\alpha}]) ~m(x_1~\kw{int},~x_2~\tau) ~t_V[\ov{\alpha}]~
        \br{ x[x_1] = x_2;~\return~x }) \in \ov{D}
        \\
        \ov{\Phi} = \typeparams(t_V)
        \\
        \ov{\Phi} \by \ov{\alpha_c}
        \\
        (\ov{\Phi} \by \ov{n}) \vdash \tau \mapsto \tau^\dagger
        }{
        \func~(x~\an{t_V, \ov{n}}[\ov{\alpha_c}])~m(x_1~\kw{int}, x_2~\tau^\dagger)
        ~\an{t_V, \ov{n}}[\ov{\alpha_c}]~
        \br{ x[x_1] = x_2;~\return~x }
        }
        \right\}
    \end{mathpar}

    \caption{Translation phase of FGGA to Go monormorphisation}
\end{figure}


\subsection{Monormorphisation properties}

Any well-typed FGGA program $P$ can be monomorphised into $P^\dagger$.

\begin{theorem}[Totality]
    $P \ok \implies P \mapsto P^\dagger$
\end{theorem}

Any well-typed FGGA program $P$ is well-typed after monomorphisation, both in
FGGA and in Go.

\begin{theorem}[Soundess]
    $P \ok \implies P^\dagger \ok$
\end{theorem}

Any monomorphised program $P^\dagger$ has a one-to-one correspondence in
behaviour (reduction equivalence) to the original program $P$ \autocite{fg}.
I.e. executing program $P$ one step and then monomorphising the resulting
program, is equivalent to monomorphising program $P$ to $P^\dagger$ and
executing $P^\dagger$ one step, modulo dead-type elimination. $P_0^\dagger$
reduces to $P_1^\skull$ --- denoting the program contains dead types, which we
can eliminate by applying the monomorphisation procedure $P_1^\skull \mapsto
    P_1^\dagger$. Figure \ref{fig:bisiumulation} visualises this correspondence.

\begin{theorem}[Bisimulation]
    \begin{mathpar}
        \inferrule{
            P_0 \mapsto P^\dagger_0 \\
            P_0 \becomes P_1 \\
            P^\dagger_0 \becomes P^\skull_1 \\
            P^\skull_1 \mapsto P^\dagger_1
        }{
            P_1 \mapsto P^\dagger_1
        }

        \\

        \phantom{a}\hfill \fbox{$P \becomes P'$}

        \inferrule{
            d \becomes e
        }{ \package~\main;~\ov{D}~\func~\main()~\br{\un=d} \becomes
            \package~\main;~\ov{D}~\func~\main()~\br{\un=e} }
    \end{mathpar}
\end{theorem}

\begin{figure}
    \begin{center}
        \begin{tikzpicture}
            \node(m0) {$P_0^\dagger$};
            \node[anchor=center,right=of m0.center] (mpre1) {$P_1^\skull$};
            \node[anchor=center,right=of mpre1.center] (m1) {$P_1^\dagger$};
            \node[anchor=center,right=of m1.center] (mdots) {$...$};
            \node[anchor=center,right=of mdots.center] (mprei) {$P_n^\skull$};
            \node[anchor=center,right=of mprei.center] (mi) {$P_n^\dagger$};
            %
            \node[anchor=center,above=of m0.center]  (p0) {$P_0$};
            \node[anchor=center,above=of m1.center] (p1) {$P_1$};
            \node[anchor=center,right=of p1.center] (pdots) {$...$};
            \node[anchor=center,above=of mi.center] (pi) {$P_n$};
            %
            \draw[->,densely dashed] (p0) -- (m0);
            %
            \draw[->] (p0) -- (p1);
            \draw[->] (m0) -- (mpre1);
            \draw[->,densely dashed] (mpre1) -- (m1);
            %
            \draw[->,densely dashed] (p1) -- (m1);
            %
            \draw[->] (p1) -- (pdots);
            \draw[->] (m1) -- (mdots);
            %
            \draw[->] (pdots) -- (pi);
            \draw[->] (mdots) -- (mprei);
            \draw[->,densely dashed] (mprei) -- (mi);
            %
            \draw[->,densely dashed] (pi) -- (mi);
        \end{tikzpicture}
    \end{center}

    \caption{Bisimulation. A solid arrow indicates a program reduction step.
        A dashed arrow indicates monormorphisation of a program.}
    \label{fig:bisiumulation}
\end{figure}
