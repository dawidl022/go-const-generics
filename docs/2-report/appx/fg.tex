\section{Featherweight Go with Arrays addendum}

\subsection{FGA Syntax: Expressions}
\label{sec:fg-appx-syntax-expr}

Expressions may take on a number of forms. The most basic expressions
are integer literals (e.g. 0, 1, 2, 10, -1 etc.). Variables, which in the
context of FGA happen to be parameter names, are also basic expressions.

Complex expressions may involve 0 or more subexpressions. The first kind are
method calls, which are recursively defined as an expression, followed by a dot,
followed by a method name (an identifier), followed by a sequence of expressions
enclosed in parentheses. Another recursively defined kind of expression is a
value literal, which consists of a value type name, followed by a sequence of
expressions enclosed in curly brackets. Value literals are used to instantiate
structs or arrays. The select expression consists of an expression, followed by
a dot and the field name (an identifier), which is used to select a field from a
struct. An array index consists of an expression, followed by another expression
enclosed in square brackets. The array index expression is used to retrieve an
element of an array (the first subexpression) at a specified index (the second
subexpression).

\subsection{FGA Reduction}
\label{sec:fg-appx-reduction}

The auxiliary function $\fields$ looks up the struct type name given as an
argument in the sequence of declarations, and returns the sequence of fields in
the definition of the struct type. The rule R-Field says that a select
expression on a struct literal \emph{value} evaluates to the field value
corresponding to the same position in the struct literal as the field name is in
the struct type declaration.

The auxiliary function $\indexbounds$ returns a set of valid index indices of an
array type, i.e. the set of indices that are within the bounds of the array. The
rule R-Index says an array index expression on an array literal \emph{value}
with an integer literal $n$ as the index reduces to the element of the array at
index $n$, if and only if $n$ is within bounds of the array.

The auxiliary function $\mbody$ looks up the method declaration given by the
type name and method name, and returns the expression in the body of that method
with a template for the receiver and parameters. The rule R-Call says that a
method call expression where the receiver is a value and the arguments are also
values, reduces to the $\mbody$ of the method defined on the $\vtype$ of the
receiver, with the actual parameters (receiver and arguments) substituted for
the formal parameters in the $\mbody$ template.

Apart from the computation rules described above, there is also a congruence
rule defined in terms of evaluation contexts, which says that if $d$ evaluates
to $e$, then $d$ in the context of $E$ evaluates to $e$ in the same context $E$.
The evaluation context defines the order of evaluation \autocite{evalContexts},
when there are multiple subexpressions in a single expression, and where at
least one of the subexpressions is not a value.

By the evaluation context rules, a method call must first reduce its receiver to
a value, and subsequently reduce its arguments to values, one by one. A value
literal has its elements reduced to values, one by one. A select expression must
first reduce its receiver to a value. An array index must first reduce its
receiver to an array literal value, and then its index expression to a value.

\subsection{FGA Typing}
\label{sec:fg-appx-typing}

A method specification is considered well-typed if the parameter names are
distinct (i.e. each parameter name is different) and the parameter and return
types are themselves well-typed (rule T-Specification).

A struct type literal is well-formed if all of its fields are distinct and the
field types are themselves well-formed (rule T-Struct). An interface type
literal is well-formed if all of its method specifications are unique (i.e.
each method specified in the interface has a different name) and well-formed
(rule T-Interface). An array type literal is well-formed if the integer literal
defining the size of the array is greater than or equal to zero, and the array
element type is well-formed (rule T-Array).

A type declaration is well-formed if its type literal is well-formed (rule
T-Type). A method declaration is well-formed if the parameters (including the
receiver) are distinct, the receiver, parameter and return types are well-formed
and the method expression's type is a subtype of the return type under the
environment formed from the method parameters (and receiver) (rule T-Func).

The rule T-Var formally defines what it means for a variable $x$ to be of type
$t$ under the environment $\Gamma$, i.e. when the pair $x : t$ is in $\Gamma$. A
method call expression is well-formed and of the type $u$ being the method
specification's return type, when the receiver expression is well-typed and has
a method named $m$ in the method set of its type. In addition, all the argument
expressions must be well-typed, and be subtypes of the formal parameter types
(rule T-Call). A struct literal expression is well-formed and of the struct type
that was instantiated, if the struct type is well-formed, and each element's
type of the struct literal is a subtype of the corresponding field types in the
struct type declaration (rule T-Struct-Literal). A field select expression is
well-typed if the receiver is well-typed and of struct type. The expression is
of the same type as the field corresponding to the selected field name in the
struct type declaration (rule T-Field).

Finally, a program is well-typed when all the type declaration names are
distinct (and do not coincide with the predeclared \kw{int} type name). The
auxiliary function $\tdecls$ returns all type names declared in the program. The
method declarations must also be distinct, in the sense that no two methods
declared on the same type may have the same name. The auxiliary function
$\mdecls$ returns all pairs of $t_V.m$ (receiver type + method name) declared in
the program. All declarations along with the main expression must also be
well-formed (rule T-Prog).
