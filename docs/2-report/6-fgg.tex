\section{Featherweight Generic Go with Arrays}
\label{ch:fgg}

\emph{Featherweight Generic Go} (FGG) is an extension to \emph{Featherweight Go}
that introduces generics via type parameters, formalised in \autocite{fg}. This
section adapts FGG with the syntax and restrictions introduced in the Type
Parameters Proposal and ultimately implemented in Go v1.18
\autocite{genericsProposal}, and extends it with arrays and numerical type
parameters that can be used to create generic array types, referred to as FGGA
going forwards.

Rules or rule fragments that remain unchanged from FGA are shown in grey,
whereas new rules or rule modifications are shown in black.

\subsection{FGGA Syntax}
\label{sec:fgg-syntax}

Syntactically, a type is now one of 4 things --- it is either a type parameter,
which is just an identifier, an integer literal, the keyword $\const$ (which
cannot be used as a regular type in user programs) or a named type. Named types
are defined in terms of type names from FGA, with the addition of a sequence of
zero or more type arguments enclosed in square brackets, following the type
name. The type arguments are themselves types. While it's not explicitly stated
in figure \ref{fig:fgg-syntax}, when there are no type arguments, the square
brackets \emph{must} be omitted, to ensure compatibility with Go. The
consequence of this is that syntactically, there is no way to differentiate
between a type parameter and a named type with no type arguments. A value type
is a named type whose base type name (i.e. outermost type name) is a value type
name. The situation is analogous for interface types. An integer-like type is
either an integer literal, or a type parameter, and can be used as the size of
an array type literal. A bound is used to restrict the set of type arguments
that can be used in place of a type parameter. A type parameter constraint
$\Phi$ is used to define a type parameter along with its constraint (bound).

Method signatures, struct type literals, array type literals and value literals
are updated to used types in the place of type names. Type declarations now have
a sequence of type parameter constraints following the type name, that are
required when instantiating the type. Similarly to type arguments, a type with
no type parameters \emph{must} omit the square brackets.

Unlike in FGG, method receivers are invariant, as such, there is no need to
repeat the type parameter constraints on the receiver type parameters. Instead,
a simple type parameter sequence is used. In Go, these type parameters can be
named differently from type parameters in the type declaration, since the order
of the parameters is sufficient to identify them, however, to keep the rules
simpler, in FGGA the type parameter names must match exactly with the ones
specified in the type declaration. Another feature omitted from the current
implementation of generics in Go, but present in FGG, are method-specific type
parameters. In order to be compatible with the current implementation, these
constructs are also omitted from FGGA.

\documentclass[acmsmall,screen]{acmart}
\usepackage{xcolor}

\begin{document}

\newcommand{\squeeze}{\hspace{-1.5em}}
\newcommand{\gap}{\hspace{0.75em}}
\newcommand{\Strut}{\vphantom{\ov{f}}}
\newcommand{\calD}{\mathcal{D}}
\newcommand{\calS}{\mathcal{S}}
\newcommand{\ov}{\overline}
\newcommand{\at}{\mathrel{/}}
\newcommand{\kw}[1]{\texttt{\bf #1}}
\newcommand{\id}[1]{\texttt{#1}}
\newcommand{\meta}{\mathit}
\newcommand{\ok}{~\meta{ok}}
\newcommand{\val}{~\meta{value}}
\newcommand{\fields}{\meta{fields}}
\newcommand{\methods}{\meta{methods}}
\newcommand{\vtype}{\meta{type}}
\newcommand{\mbody}{\meta{body}}
\newcommand{\tdecls}{\meta{tdecls}}
\newcommand{\mdecls}{\meta{mdecls}}
\newcommand{\bounds}{\meta{bounds}}
\newcommand{\indexbounds}{\meta{indexBounds}}
\newcommand{\len}{\meta{lenType}}
\newcommand{\instance}{\meta{instance}}
\newcommand{\notref}{\meta{notReferenced}}
\newcommand{\isconst}{\meta{isConst}}
\newcommand{\isarraysetmethod}{\meta{isArraySetMethod}}
\newcommand{\flatten}{\meta{flatten}}
\newcommand{\unique}{\meta{unique}}
\newcommand{\distinct}{\meta{distinct}}
\newcommand{\elementtype}{\meta{elementType}}
\newcommand{\lentype}{\meta{lenType}}
\newcommand{\typeparams}{\meta{typeParams}}
\newcommand{\type}{\kw{type}}
\newcommand{\struct}{\kw{struct}}
\newcommand{\interface}{\kw{interface}}
\newcommand{\func}{\kw{func}}
\newcommand{\return}{\kw{return}}
\newcommand{\package}{\kw{package}}
\newcommand{\main}{\kw{main}}
\newcommand{\const}{\kw{const}}

\newcommand{\un}{\id{\textunderscore}}
\newcommand{\prog}{\rhd}
\newcommand{\br}[1]{\id{\{}#1\id{\}}}
\newcommand{\lst}[1]{[#1]}
\newcommand{\set}[1]{\{#1\}}
\newcommand{\an}[1]{\langle #1 \rangle}
\newcommand{\imp}{\mathbin{\id{<:}}}
\newcommand{\notimp}{\mathbin{\not\!\!\imp}}
\newcommand{\becomes}{\longrightarrow}
\newcommand{\by}{\mathbin{:=}}
\newcommand{\gray}[1]{{\color{gray}#1}}
\newcommand{\black}[1]{{\color{black}#1}}
\newcommand{\comma}{,\,}
\newcommand{\stoup}{;\,}
\newcommand{\Hole}{\Box}
\newcommand{\trep}{\ensuremath{\mathsf{Rep}}}

\newcommand{\sem}[1]{\llbracket #1 \rrbracket}
\newcommand{\bodies}{\meta{bodies}}
\newcommand{\fix}{\kw{fix}}
\newcommand{\lam}[1]{\lambda #1.\,}

\newcommand{\yields}{\blacktriangleright}
\newcommand{\extensionkw}{closure}
\newcommand{\Sclo}{\textit{S-\extensionkw}}
\newcommand{\Iclo}{\textit{I-\extensionkw}}
\newcommand{\Fclo}{\textit{F-\extensionkw}}
\newcommand{\Mclo}{\textit{M-\extensionkw}}
\newcommand{\Pclo}{\textit{P-\extensionkw}}
\newcommand{\Eclo}{\textit{E-\extensionkw}}

\newcommand{\SExtensionD}[2]{\Sclo(#1)}
\newcommand{\IExtensionD}[2]{\Iclo_{#2}(#1)}
\newcommand{\FExtensionD}[2]{\Fclo(#1)}
\newcommand{\MExtensionD}[2]{\Mclo_{#2}(#1)}
\newcommand{\EExtensionD}[2]{\Eclo(#1)}
\newcommand{\PExtensionD}[2]{\Pclo(#1)}

\newcommand{\derivrule}[3][]{
    \inferrule*[right=#1]
    {#2}
    {#3}
}
\newcommand{\axiomrule}[2][]{
    \inferrule*[right=#1]
    {~}
    {#2}
}

\newcommand{\register}[1]{
    \expandafter\newcommand\csname #1\endcsname{
        \operatorname{#1}
    }
}

\newcommand{\alias}[2]{
    \expandafter\newcommand\csname #1\endcsname{
        \operatorname{#2}
    }
}

\newcommand{\aliasparam}[1]{
    \expandafter\newcommand\csname #1Param\endcsname{
        \operatorname{#1}
    }
}

\newcommand{\ws} {
    % create some whitespace
    \begin{equation*}
    \end{equation*}
}

\newcommand{\golisting}[1] {
    \noindent\begin{minipage}{\linewidth}
        \lstinputlisting[language=Go, tabsize=4, firstline=3]{#1}
    \end{minipage}
}



\begin{figure}
    \gray{
        \begin{minipage}[t]{\textwidth}
            \begin{tabular}[t]{ll}
                Field name               & $f$                                                          \\
                Method name              & $m$                                                          \\
                Variable name            & $x$                                                          \\
                Structure type name      & $t_S, u_S$                                                   \\
                Interface type name      & $t_I, u_I$                                                   \\
                Array type name          & $t_A, u_A$                                                   \\
                Declared type name       & $t_D, u_D$ ::= $t_S \mid t_I \mid t_A$                       \\
                Type name                & $t, u$ ::= $t_D \mid \kw{int}$                               \\
                Value type name          & $t_V, u_V$ ::= $t_S \mid t_A$                                \\
                \black{Type parameter}   & \black{$\alpha$}                                             \\
                Method signature         & $M$ ::= $(\ov{x~\black{\tau}})~\black{\tau}$                 \\
                Method specification     & $S$ ::= $mM$                                                 \\
                Type Literal             & $T$ ::=                                                      \\
                \quad Structure          & \quad $\struct~\br{\ov{f~\black{\tau}}}$                     \\
                \quad Interface          & \quad $\interface~\br{\ov{S}}$                               \\
                \quad Array              & \quad$[\black{\tau_n}]\black{\tau}$                          \\
                \\
                \\
                \\
                Declaration              & $D$ ::=                                                      \\
                \quad Type declaration   & \quad $\type~t\black{[\ov{\Phi}]}~T$                         \\
                \quad Method declaration & \quad $\func~(x~t_V\black{[\ov{\alpha}]})~mM~\br{\return~e}$ \\
                \quad Array set method   &                                                              \\
                \quad declaration        & \quad $\func~(x~t_A\black{[\ov{\alpha}]})
                    ~m(x_1~\kw{int},~x_2~\black{\tau}) ~t_A\black{[\ov{\alpha}]}~
                \br{ x[x_1] = x_2;~\return~x }$                                                         \\
                Program                  & $P$ ::= $\package~\main;~\ov{D}~\func~\main()~\br{\un=e}$
            \end{tabular}
        \end{minipage}
        \hspace{-0.5\textwidth}
        \begin{minipage}[t]{0.4\textwidth}
            \begin{tabular}[t]{ll}
                \black{Type}                      & \black{$\tau, \sigma$ ::=}                         \\
                \quad \black{Type parameter}      & \quad \black{$\alpha$}                             \\
                \quad \black{Named type}          & \quad \black{$t[\ov{\tau}]$}                       \\
                \quad \black{Integer literal}     & \quad \black{$n$}                                  \\
                \quad \black{Constant}            & \quad \black{\fbox{$\const$}}                      \\
                \black{Value type}                & \black{$\tau_V,\sigma_V$ ::= $t_V[\ov{\tau}]$}     \\
                \black{Interface type}            & \black{$\tau_I,\sigma_I$ ::= $t_I[\ov{\tau}]$}     \\
                \black{Interface-like type}       & \black{$\tau_J,\sigma_J$ ::= $\alpha \mid \tau_I$} \\
                \black{Integer-like type}         & \black{$\tau_n,\sigma_n$ ::= $\alpha \mid n$}      \\
                \black{Bound}                     & \black{$\gamma$ ::= $\tau_I \mid \const$}          \\
                \black{Type parameter constraint} & \black{$\Phi$ ::= $\alpha~\gamma$}                 \\
                Expression                        & $e$ ::=                                            \\
                \quad Integer literal             & \quad$n$                                           \\
                \quad Variable                    & \quad $x$                                          \\
                \quad Method call                 & \quad $e.m(\ov{e})$                                \\
                \quad Value literal               & \quad $\black{\tau_V}\br{\ov{e}}$                  \\
                \quad Select                      & \quad $e.f$                                        \\
                \quad Array index                 & \quad$e$[$e$]
            \end{tabular}
        \end{minipage}
    }
    \caption{FGG syntax with arrays}
\end{figure}
\end{document}


\subsection{FGGA Reduction}

The reduction rules for FGGA are nigh identical to the ones found in FGA, with
the only notable difference being that type names $t$ are replaced with types
$\tau$.

More interesting differences occur in the auxiliary functions. In FGA,
$\indexbounds$ took a type name and performed a simple lookup in the type
declaration to extract the size of the array. In FGGA, there are two cases of
the $\indexbounds$ function. When the size of the array type is an integer
literal, then the bounds are calculated in the same way as in FGA, based on a
lookup of the named type declaration of the array type. If however, the array
type declaration has a type parameter in place of the array size, then the
bounds are calculated from the integer literal type argument in the array type
that corresponds to the type parameter used as the array size in the
declaration. The correspondence is determined by matching on the same position
in the sequence of type parameter constraints and the sequence of type
arguments.

The $\mbody$ auxiliary function has also been updated to perform substitution of
method type parameters with the receiver's type arguments within the method
expression. A map within a pair of double square brackets applied to a term
(e.g. $e\llbracket\theta\rrbracket$ in the $\mbody$ function) denotes a
substitution application in this and following rules. The type arguments are
mapped to the type parameters based on their respective positions.

Similarly, the $\fields$ auxiliary function has been updated to perform a type
parameter substitution with type arguments on the resulting struct fields. This
is performed analogously to how it's done in the $\mbody$ function, except that
type parameters are extracted from the type parameter constraints in the struct
type declaration.

\documentclass[acmsmall,screen]{acmart}
\usepackage{mathpartir}
\usepackage{xcolor}

\begin{document}

\newcommand{\squeeze}{\hspace{-1.5em}}
\newcommand{\gap}{\hspace{0.75em}}
\newcommand{\Strut}{\vphantom{\ov{f}}}
\newcommand{\calD}{\mathcal{D}}
\newcommand{\calS}{\mathcal{S}}
\newcommand{\ov}{\overline}
\newcommand{\at}{\mathrel{/}}
\newcommand{\kw}[1]{\texttt{\bf #1}}
\newcommand{\id}[1]{\texttt{#1}}
\newcommand{\meta}{\mathit}
\newcommand{\ok}{~\meta{ok}}
\newcommand{\val}{~\meta{value}}
\newcommand{\fields}{\meta{fields}}
\newcommand{\methods}{\meta{methods}}
\newcommand{\vtype}{\meta{type}}
\newcommand{\mbody}{\meta{body}}
\newcommand{\tdecls}{\meta{tdecls}}
\newcommand{\mdecls}{\meta{mdecls}}
\newcommand{\bounds}{\meta{bounds}}
\newcommand{\indexbounds}{\meta{indexBounds}}
\newcommand{\len}{\meta{lenType}}
\newcommand{\instance}{\meta{instance}}
\newcommand{\notref}{\meta{notReferenced}}
\newcommand{\isconst}{\meta{isConst}}
\newcommand{\isarraysetmethod}{\meta{isArraySetMethod}}
\newcommand{\flatten}{\meta{flatten}}
\newcommand{\unique}{\meta{unique}}
\newcommand{\distinct}{\meta{distinct}}
\newcommand{\elementtype}{\meta{elementType}}
\newcommand{\lentype}{\meta{lenType}}
\newcommand{\typeparams}{\meta{typeParams}}
\newcommand{\type}{\kw{type}}
\newcommand{\struct}{\kw{struct}}
\newcommand{\interface}{\kw{interface}}
\newcommand{\func}{\kw{func}}
\newcommand{\return}{\kw{return}}
\newcommand{\package}{\kw{package}}
\newcommand{\main}{\kw{main}}
\newcommand{\const}{\kw{const}}

\newcommand{\un}{\id{\textunderscore}}
\newcommand{\prog}{\rhd}
\newcommand{\br}[1]{\id{\{}#1\id{\}}}
\newcommand{\lst}[1]{[#1]}
\newcommand{\set}[1]{\{#1\}}
\newcommand{\an}[1]{\langle #1 \rangle}
\newcommand{\imp}{\mathbin{\id{<:}}}
\newcommand{\notimp}{\mathbin{\not\!\!\imp}}
\newcommand{\becomes}{\longrightarrow}
\newcommand{\by}{\mathbin{:=}}
\newcommand{\gray}[1]{{\color{gray}#1}}
\newcommand{\black}[1]{{\color{black}#1}}
\newcommand{\comma}{,\,}
\newcommand{\stoup}{;\,}
\newcommand{\Hole}{\Box}
\newcommand{\trep}{\ensuremath{\mathsf{Rep}}}

\newcommand{\sem}[1]{\llbracket #1 \rrbracket}
\newcommand{\bodies}{\meta{bodies}}
\newcommand{\fix}{\kw{fix}}
\newcommand{\lam}[1]{\lambda #1.\,}

\newcommand{\yields}{\blacktriangleright}
\newcommand{\extensionkw}{closure}
\newcommand{\Sclo}{\textit{S-\extensionkw}}
\newcommand{\Iclo}{\textit{I-\extensionkw}}
\newcommand{\Fclo}{\textit{F-\extensionkw}}
\newcommand{\Mclo}{\textit{M-\extensionkw}}
\newcommand{\Pclo}{\textit{P-\extensionkw}}
\newcommand{\Eclo}{\textit{E-\extensionkw}}

\newcommand{\SExtensionD}[2]{\Sclo(#1)}
\newcommand{\IExtensionD}[2]{\Iclo_{#2}(#1)}
\newcommand{\FExtensionD}[2]{\Fclo(#1)}
\newcommand{\MExtensionD}[2]{\Mclo_{#2}(#1)}
\newcommand{\EExtensionD}[2]{\Eclo(#1)}
\newcommand{\PExtensionD}[2]{\Pclo(#1)}

\newcommand{\derivrule}[3][]{
    \inferrule*[right=#1]
    {#2}
    {#3}
}
\newcommand{\axiomrule}[2][]{
    \inferrule*[right=#1]
    {~}
    {#2}
}

\newcommand{\register}[1]{
    \expandafter\newcommand\csname #1\endcsname{
        \operatorname{#1}
    }
}

\newcommand{\alias}[2]{
    \expandafter\newcommand\csname #1\endcsname{
        \operatorname{#2}
    }
}

\newcommand{\aliasparam}[1]{
    \expandafter\newcommand\csname #1Param\endcsname{
        \operatorname{#1}
    }
}

\newcommand{\ws} {
    % create some whitespace
    \begin{equation*}
    \end{equation*}
}

\newcommand{\golisting}[1] {
    \noindent\begin{minipage}{\linewidth}
        \lstinputlisting[language=Go, tabsize=4, firstline=3]{#1}
    \end{minipage}
}


\begin{figure}
    \begin{mathpar}
        \inferrule
        {
        (\type~t_A[\ov{\Phi}]~ [n]\tau) \in \ov{D}\\
        \tau_A = t_A[\ov{\tau}]
        }
        { \{ i \in \mathbb{Z} \mid 0 \le i < n \} = \indexbounds(\tau_A)}

        \inferrule
        {
        (\type~t_A[\alpha~\const, \ov{\Phi}]~ [\alpha]\tau) \in \ov{D}\\
        \tau_A = t_A[n, \ov{\tau}]
        }
        { \{ i \in \mathbb{Z} \mid 0 \le i < n \} = \indexbounds(\tau_A)}

        \gray{
        \inferrule
        {
        (\func~(x~\black{t_V[\ov{\alpha}]})~m(\ov{x~\tau})~\tau~\br{\return~e}) \in \ov{D} \\
        \theta = (\ov{\Phi \by \sigma})
        }
        {\mbody(t_V[\ov{\sigma}].m) = (x:t_V[\ov{\sigma}],\ov{x:\tau}).e[\theta]}
        }

        \inferrule
        {
        (\func~(x~t_A[\ov{\alpha}]) ~m(x_1~\kw{int},~x_2~\tau) ~t_A[\ov{\alpha}]~
        \br{ x[x_1] = x_2;~\return~x }) \in \ov{D}\\
        \tau_A = t_A[\ov{\tau}]
        }
        {\isarraysetmethod(\tau_A.m)}
    \end{mathpar}
    \caption{FG auxiliary functions for array reduction}
\end{figure}


\begin{figure}
    \begin{center}
        Value \qquad $v$ ::= $t_S\br{\ov{v}} \mid t_A\br{\ov{v}} \mid n$
    \end{center}

    \begin{center}
        \gray{
            \begin{minipage}{0.5\textwidth}
                \begin{tabular}{ll}
                    Evaluation context          & $E$ ::=                      \\
                    \quad Hole                  & \quad $\Hole$                \\
                    \quad Method call receiver  & \quad $E.m(\ov{e})$          \\
                    \quad Method call arguments & \quad $v.m(\ov{v},E,\ov{e})$
                \end{tabular}
            \end{minipage}
            \begin{minipage}{0.4\textwidth}
                \begin{tabular}{ll}
                    ~                                                                         \\
                    \quad Value literal          & \quad $\black{\tau_V}\br{\ov{v},E,\ov{e}}$ \\
                    \quad Select                 & \quad $E.f$                                \\
                    \quad \black{Index receiver} & \quad \black{$E[e]$}                       \\
                    \quad \black{Index argument} & \quad \black{$\tau_A\br{\ov{v}}[E]$}
                \end{tabular}
            \end{minipage}
        }
    \end{center}

    Reduction \hfill \fbox{$d \becomes e$}
    \begin{mathpar}

        \gray{
            \inferrule[r-field]
            { (\ov{f~t}) = \fields(\tau_S) }
            { \tau_S\br{\ov{v}}.f_i \becomes v_i }
        }

        \inferrule[r-index]
        {
            n \in \indexbounds(\tau_A)
        }
        { \tau_A\br{\ov{v}}[n] \becomes v_n }

        \gray{
            \inferrule[r-call]
            { (x : \black{t_V},\ov{x: t}).e = \mbody(\vtype(v).m) }
            { v.m(\ov{v}) \becomes e[x \by v, \ov{x \by v}] }
        }

        % potential ambiguity with regular r-call?
        % no, because isarraysetmethod(t_A.m) and body(t_A.m) are mutually exclusive
        \inferrule[r-array-set]
        {
            n \in \indexbounds(\tau_A) \\
            \isarraysetmethod(\tau_A.m) \\
        }
        { \tau_A\br{\ov{v}}.m(n, v) \becomes \tau_A\br{\ov{v}}[n := v]}

        \gray{
            \inferrule[r-context]
            { d \becomes e }
            { E[d] \becomes E[e] }
        }

    \end{mathpar}
    \caption{FGG reduction for arrays}
\end{figure}

\end{document}


\subsection{FGGA Typing}

$\Delta$ is defined as a typing environment mapping type parameters $\alpha$
to their bounds $\gamma$. Type parameter constraint sequences $\ov{\Phi}$ may
implicitly coerce to type environments. As before in FGA, $\Gamma$ is defined as
an environment mapping variables $x$ to types $\tau$.

The ``implements'' relation now includes a typing environment $\Delta$. As
before, all types implement themselves (rules $\imp_{\text{Param}}$, $\imp_V$,
$\imp_{int}$, $\imp_{const}$ and $\imp_n$), and integer literal types implement
the \kw{int} type (rule $\imp_{int-n}$). Non-negative integer literals now also
implement the \kw{const} type (rule $\imp_{const-n}$), so that they can be used
as type arguments, where the bound of the type parameter is \kw{const}. Since
the use case of numerical type parameters is to generically size arrays, and
arrays cannot be of a negative size, negative integers do not implement
\kw{const}.


The $methods_\Delta$ auxiliary now also accepts a typing environment as input
(denoted by a subscript $\Delta$), as a type parameter bound lookup may be
necessary to retrieve the methods of a type. For a named value type
$t_V[\ov{\tau}]$ $\methods_\Delta$ returns the set of method specifications of
the base value type $t_V$, with type substitutions performed on them. E.g. if
the declared value type \texttt{Foo} with a type parameter \texttt{T} has a
method with the specification \texttt{f(x T) T}, then
$\methods_\Delta(\texttt{Foo[int]})$, where $\Delta$ is any typing environment
(as it is not relevant in the case of looking up methods of a named value type),
would return \{(\texttt{f(x int) int})\}. The implication of this is that one
instantiation of a generic type could implement a certain interface, while
another could not, depending on the type arguments. A $\methods_\Delta$ lookup
on an interface has similarly been updated to return a type-substituted method
set. Additionally, the method set of a type parameter $\alpha$ is equal to the
method set of its bound $\gamma$, as specified in the typing environment
$\Delta$.

The integer literal type can now appear in user programs, although with
restrictions as to where they can be used. They can only be used as type
arguments in a named type and as the size parameter of an array type literal but
not e.g. as a standalone type of a variable or a return type. Const type
parameters (i.e. those that are bound by \kw{const}) have the exact same
restrictions as integer literal types. To distinguish between these two kinds of
user-program types, the $\isconst_\Delta$ predicate was created, which simply
checks if the type is a subtype of \kw{const}. Because both \kw{const} type
arguments and array sizes must be non-negative, we can also restrict all integer
literal types to be non-negative (rule T-N-Type).

A type parameter $\alpha$ is considered well-typed if it appears in the typing
environment $\Delta$ (rule T-Param). The rule T-Named has been updated to
type-check each type argument in the named type, recursively applying one of
``well-formed type'' rules ($\Delta \vdash \tau \ok$) on each argument, and
checking whether each argument satisfies the type parameter bound via the
$(\ov{\Phi \by_\Delta \tau})$ lookup. The lookup differs from the regular type
substitution map lookup (denoted without the subscript $\Delta$), that it
additionally checks the bounds of each type argument via the subtyping relation,
which itself is done via a type substitution on the type parameters as well as
the bounds, because bounds may refer to other type parameters in the type
parameter constraints sequence. Not only must the type arguments implement the
type-substituted bounds of their corresponding type parameters, but they must
also have an equal ``\kw{const}-ness'', i.e. either both the argument and bound
return true for $\isconst$, or they both return false. The need for this check
can be illustrated with the following example: $2 \imp \texttt{any}$ holds, but
2 cannot be used as a variable type, and so it cannot be used to instantiate a
type parameter bound by \texttt{any}. This is enforced in the rule because
$\isconst_\Delta(2) \neq \isconst_\Delta(\texttt{any})$. Both the type-checking
and non type-checking lookups yield a map from type parameters to types, denoted
by $\eta$.

The rule T-Formal specifies what it means for type parameter constraint
sequences (i.e. the formal type parameters found in a type declaration) to be
well typed. All type parameters must be distinct, and their bounds must be
well-typed, in the context of the typing environment that is created from the
type parameter constraints that are being checked. This environment is necessary
for recursively defined type parameter bounds, e.g. if \texttt{Eq} is an
interface with one type parameter, then we can define another type \texttt{Foo}
with a type parameter $\alpha$ bound by \texttt{Eq[$\alpha$]}, i.e. the type
parameter is referenced within its own bound.

There is a restriction in the current implementation of Go, that FGGA also
adheres to, and that is that no type in any type parameter bound can refer to
the type being declared (directly or indirectly). E.g. the interface type
declaration \texttt{Eq} cannot have a type parameter $\alpha$ bound by
\texttt{Eq[$\alpha$]} itself \autocite{spec}.

The check for this restriction is performed via the $\notref$ auxiliary in the
T-Type rule for each type parameter bound $\gamma$. $\notref$ is defined
recursively, and described in appendix \ref{fgg-appx-typing-not-ref}.

An array size can now be any valid \kw{const} type, and the element type must be
a valid non-\kw{const} type, evaluated in the typing environment $\ov{\Phi}$.
The $\lentype$ auxiliary now has two cases. One as before: when the array size
is an integer literal (i.e. non-generic). The other is the generic case, where
the array size is a type parameter, in which case $lentype$ returns the type
argument that is used to instantiate the array size type parameter (by matching
on the type parameter position). Array literals in FGGA can only be constructed
when the $\lentype$ is an integer literal type --- either when the array length
type is non-generic, or has been instantiated with an integer literal (rule
T-Array-Literal). When the array size is instantiated with a type parameter, the
size is unknown, hence it is not safe to assume the array can hold any elements
(e.g. it could be instantiated later with a size of 0). In practice, Go's zero
values could be used to instantiate generically sized arrays, but since FGGA
does not support them, even empty generic array initialisation is not allowed.

It is worth noting that generic, non-instantiated arrays (i.e. where $\lentype$
does not return an integer literal) can only be indexed using a non-constant
integer of type \kw{int}. This is to stay consistent with current expectations:
indexing into an array using a constant integer literal is only allowed to fail
at compile-time. In line with the Go spirit of type-checking generic code at the
declaration site, as opposed to the call site, indexing into generic arrays
using integer literals is not allowed in FGGA, even if the index would remain
within bounds for all array instantiations in the scope of the current program.
The programmer may still achieve such behaviour by explicitly assigning the
integer literal to a non-const \kw{int} variable and then performing an index
operation.

\begin{figure}
    \begin{mathpar}
        \gray{
        \inferrule
        {(\type~t_A\black{[\ov{\Phi}]}~ [\black{\tau_n}]\black{\tau}) \in \ov{D}}
        {\black{\tau} = \elementtype(t_A)}
        }

        \inferrule
        {
        \tau_A = t_A[\ov{\tau}] \\
        (\ov{\alpha~\gamma}) = \ov{\Phi} \\
        (\type~t_A[\ov{\Phi}]~ [\alpha_i]\tau) \in \ov{D}
        }
        {\tau_i = \len(\tau_A)}


        \gray{
        \inferrule
        {
        \black{\tau_A = t_A[\ov{\tau}]} \\
        (\type~t_A\black{[\ov{\Phi}]}~ [n]\black{\tau}) \in \ov{D}
        }
        {n = \len(\black{\tau}_A)}
        }


        \inferrule
        {(\type~t[\ov{\Phi}]~T) \in \ov{D}}
        {\ov{\Phi} = \typeparams(t)}

        \inferrule
        {\Delta \vdash \tau \imp \const}
        {\isconst_\Delta(\tau)}

        \inferrule
        {
            (\ov{\alpha~\gamma}) = \ov{\Phi} \\
            \eta = (\ov{\Phi \by \tau}) \\
            \Delta \vdash (\ov{\alpha \imp \gamma})\llbracket\eta\rrbracket \\
            \ov{\isconst_\Delta(\alpha) = \isconst_\Delta(\gamma)}\llbracket\eta\rrbracket
        }
        {(\ov{\Phi \by_\Delta \tau}) = \eta}

        \axiomrule{
            \notref_\alpha(\ov{t_r},~n)
        }

        \axiomrule{
            \notref_\alpha(\ov{t_r},~\kw{const})
        }

        \axiomrule{
            \notref_\alpha(\ov{t_r},~\alpha)
        }

        \inferrule
        {
            \ov{\notref_\alpha(\ov{t_r},~\tau)}
            \\
            \ov{t_r \neq t}
            \\
            (\ov{\alpha~\gamma}) = \typeparams(t)
            \\
            \ov{
                \notref_\alpha(\ov{t_r}, t, \gamma)
            }
        }
        {
            \notref_\alpha(\ov{t_r},~t[\ov{\tau}])
        }

        \inferrule
        {~}
        {
            \notref(\ov{t_r}, \alpha)
        }

        \gray{
            \inferrule
            {~}
            {
                \notref(\ov{t_r}, \kw{int})
            }

            \inferrule
            {~}
            {
                \notref(\ov{t_r},~\interface~\br{\ov{S}})
            }
        }
        \\
        \gray{
            \inferrule
            {
                \notref(\ov{t_r}, \black{\tau})
            }
            {
                \notref(\ov{t_r},~[\black{\tau_n}]\black{\tau})
            }

            \inferrule
            {
                \ov{\notref(\ov{t_r}, \black{\tau})}
            }
            {
                \notref(\ov{t_r},~\struct \br{\ov{f~\black{\tau}}})
            }
        }

        \gray{
            \inferrule
            {
                (\type~t_D\black{[\ov{\Phi}]}~T) \in \ov{D}
                \\
                \ov{t_r \neq t_D} \\
                \\
                \black{\eta = (\ov{\Phi \by \tau})}
                \\
                \notref(\ov{t_r}, t_D, T\black{\llbracket\eta\rrbracket})
            }
            {
                \notref(\ov{t_r}, t_D\black{[\ov{\tau}]})
            }

        }

        \gray{
            \inferrule
            {~}
            {\methods_\black{\Delta}(\kw{int}) = \{\}}
        }

        \gray{
            \inferrule
            {~}
            {\methods_\black{\Delta}(n) = \{\}}
        }

        \gray{
            % negative spacing to more ovenly overflow margins on both sides
            \mkern-24mu
            \inferrule
            {
                \black{ \eta = (\ov{\Phi \by \tau}) }
                \\
                \black{ \ov{\Phi} = \typeparams(t_V) }
            }
            {
                \methods_\black{\Delta}(t_V\black{[\ov{\tau}]}) =
                \{(mM)\black{\llbracket\eta\rrbracket}
                \mid (\func~(x~t_V\black{[\ov{\alpha}]})~mM~\br{\return~e}) \in \ov{D}
                \}
                \black{~\cup}
                \\
                \{
                m(x_1~\kw{int},~x_2~\black{\tau\llbracket\eta\rrbracket})~t_V\black{[\ov{\tau}]} \mid
                (\func~(x~t_V\black{[\ov{\alpha}]}) ~m(x_1~\kw{int},~x_2~\black{\tau}) ~t_V\black{[\ov{\alpha}]}~
                \br{ x[x_1] = x_2;~\return~x }) \in \ov{D}
                \}
            }
        }

        \gray{
            \inferrule
            {
                \type~t_I\black{[\ov{\Phi}]}~\interface~\br{\ov{S}} \in \ov{D} \\
                \black{ \eta = (\ov{\Phi \by \tau})}}
            {
                \methods_\black{\Delta}(t_I\black{[\ov{\tau}]}) =
                \ov{S}\black{\llbracket\eta\rrbracket}
            }
        }

        \inferrule
        {(\alpha : \black{\gamma}) \in \Delta}
        {\methods_\Delta(\alpha) = \methods_\Delta(\gamma)}

        \gray{
            \inferrule
            {\distinct(\ov{m})}
            {\unique(\ov{mM})}

            \inferrule
            {a_i \neq a_j~\forall a_i, a_j \in \ov{a},~i \neq j}
            {\distinct(\ov{a})}
        }

        \gray{
            \inferrule
            {}
            {\tdecls(\ov{D}) = \lst{t_D \mid (\type~t_D\black{[\ov{\Phi}]}~T) \in \ov{D}}}
        }

        \gray{
            \inferrule
            {}
            {\mdecls(\ov{D}) = \lst{t_V.m \mid
                    (\func~(x~t_V\black{[\ov{\alpha}]})~mM~\br{\return~e}) \in \ov{D}}}
        }
    \end{mathpar}
    \caption{FGG auxiliary functions for typing with arrays}
\end{figure}

\begin{figure}

    Implements
    \hfill \fbox{$\Delta \vdash \tau \imp \sigma$}
    \begin{mathpar}

        \inferrule[$\imp_{\text{Param}}$]
        { ~ }
        { \Delta \vdash \alpha \imp \alpha }

        \gray{
            \inferrule[$\imp_V$]
            { ~ }
            { \black{\Delta \vdash}~\tau_V \imp \tau_V }
        }

        \gray{
            \inferrule[$\imp_{int}$]
            {~}
            {\black{\Delta \vdash}~\kw{int} \imp \kw{int}}
        }

        \inferrule[$\imp_{const}$]
        {~}
        {\Delta \vdash \const \imp \const}

        \gray{
            \inferrule[$\imp_{n}$]
            {~}
            {\black{\Delta \vdash}~n \imp n}
        }

        \gray{
            \inferrule[$\imp_{int-n}$]
            { ~ }
            { \black{\Delta \vdash}~n \imp \kw{int} }
        }

        \gray{
            \inferrule[$\imp_I$]
            {
                \methods_\black{\Delta}(\black{\tau}) \supseteq \methods_\black{\Delta}(\black{\tau}_I) \\
            }
            { \black{\Delta \vdash}~\tau \imp \tau_I }
        }

        \inferrule[$\imp_{const-n}$]
        { n \ge 0 }
        { \Delta \vdash n \imp \const }

        \inferrule[$\imp_{const-\text{Param}}$]
        { (\alpha : \const) \in \Delta }
        { \Delta \vdash \alpha \imp \const }
    \end{mathpar}

    Well-formed type
    \hfill \fbox{$\Delta \vdash \tau \ok$}
    \begin{mathpar}
        \inferrule[t-n-type]
        { n \ge 0 }
        { \Delta \vdash n \ok }

        \gray{
            \inferrule[t-int-type]
            {~}
            { \black{\Delta \vdash}~\kw{int} \ok }
        }

        \inferrule[t-param]
        { (\alpha : \gamma) \in \Delta }
        { \Delta \vdash \alpha \ok }

        \gray{
            \inferrule[t-named]
            {
                \black{\Delta \vdash \ov{\tau \ok}}
                \and
                (\type~t\black{[\ov{\Phi}]}~T) \in \ov{D}
                \and
                \black{\eta = (\ov{\Phi \by_\Delta \tau})}
            }
            { \black{\Delta \vdash}~t\black{[\ov{\tau}]} \ok }
        }
    \end{mathpar}

    Well-formed type formals
    \hfill \fbox{$\Delta \vdash \const \ok$} \qquad \fbox{$\ov{\Phi} \ok$}
    \begin{mathpar}
        \inferrule[t-const]
        {~}
        {
            \Delta \vdash \const \ok
        }

        \inferrule[t-formal]
        {
            (\ov{\alpha~\gamma}) = \ov{\Phi} \\
            \distinct(\ov{\alpha}) \\
            \ov{\Phi} \vdash \ov{\gamma \ok}
        }
        { \ov{\Phi} \ok}

    \end{mathpar}

    Well-formed method specifications and type literals
    \hfill \fbox{$\ov{\Phi} \vdash S \ok$} \qquad \fbox{$\ov{\Phi} \vdash T \ok$}
    \begin{mathpar}
        \gray{
            \inferrule[t-specification]
            {
                \distinct(\ov{x}) \\
                \black{\ov{\Phi} \vdash \ov{\tau \ok}}\\
                \black{\ov{\Phi} \vdash \tau \ok}\\
                \black{\ov{
                        \neg \isconst_{\ov{\Phi}}(\tau)
                    }}\\
                \black{\neg \isconst_{\ov{\Phi}}(\tau)}
            }
            { \black{\ov{\Phi} \vdash}~m(\ov{x~\black{\tau}})~\black{\tau} \ok }
        }

        \gray{
            \inferrule[t-struct]
            {
                \distinct(\ov{f}) \\
                \black{\ov{\Phi} \vdash \ov{\tau \ok}}\\
                \black{
                    \ov{
                        \neg \isconst_{\ov{\Phi}}(\tau)
                    }
                }
            }
            { \black{\ov{\Phi} \vdash}~\struct~\br{\ov{f~\black{\tau}}} \ok }
        }

        \gray{
            \inferrule[t-interface]
            {
                \unique(\ov{S}) \\
                \black{\ov{\Phi} \vdash}~\ov{S \ok}
            }
            { \black{\ov{\Phi} \vdash}~\interface~\br{\ov{S}} }
        }

        \gray{
        \inferrule[t-array]
        {
        \black{\ov{\Phi} \vdash \tau_n \ok}\\
        \black{\isconst_{\ov{\Phi}}(\tau_n)}\\
        \black{\ov{\Phi} \vdash \tau \ok}\\
        \black{\neg \isconst_{\ov{\Phi}}(\tau)}
        }
        {
        \black{\ov{\Phi} \vdash}~[\black{\tau_n}]\black{\tau} \ok
        }
        }
    \end{mathpar}
    \caption{FGG typing  with arrays (1 of 3)}
\end{figure}

\begin{figure}


    Well-formed declarations \hfill \fbox{$D \ok$}
    \begin{mathpar}
        \inferrule[t-type]
        {
        \ov{\Phi \ok}
        \\
        \ov{\Phi} = (\ov{\alpha~\gamma})
        \\
        \ov{\notref(t, \gamma)}
        \\
        \ov{\Phi} \vdash T \ok
        }
        { \type~t[\ov{\Phi}]~T \ok }

        \inferrule[t-func]
        {
        \distinct(x, \ov{x}) \\
        \ov{\Phi} = \typeparams(t_V)\\
        (\ov{\alpha~\gamma}) = \ov{\Phi}\\
        \ov{\Phi} \vdash m(\ov{x~\tau})~\sigma \ok \\
        \ov{\Phi} \stoup
        x : t_V[\ov{\alpha}] \comma \ov{x : \tau} \vdash e : \tau \\
        \ov{\Phi} \vdash \tau \imp \sigma \\
        }
        { \func~(x~t_V[\ov{\alpha}])~m(\ov{x~\tau})~\sigma~\br{\return~e} \ok }
    \end{mathpar}

    \begin{mathpar}
        \inferrule[t-func-arrayset]
        {
        \sigma = \elementtype(t_A) \\
        \ov{\Phi} = \typeparams(t_A)\\
        (\ov{\alpha~\gamma}) = \ov{\Phi} \\
        \ov{\Phi} \vdash \tau \imp \sigma
        }
        {
        \func~(x~t_A[\ov{\alpha}]) ~m(x_1~\kw{int},~x_2~\tau) ~t_A~
        \br{ x[x_1] = x_2;~\return~x }
        }
    \end{mathpar}
    \caption{FGG typing with arrays (2 of 3)}
\end{figure}

\begin{figure}


    Expressions \hfill \fbox{$\Delta \stoup \Gamma \vdash e : \tau$}
    \begin{mathpar}
        \inferrule[t-int-literal]
        {~}
        { \Delta;\Gamma \vdash n : n }

        \gray{
            \inferrule[t-var]
            {
                (x : \tau) \in \Gamma
            }
            { \Delta \stoup \Gamma \vdash x : \tau }
        }

        \inferrule[t-call]
        {
            (m(\ov{x~\sigma})~\sigma) \in \methods_\Delta(\tau)  \\
            \Delta \stoup \Gamma \vdash e : \tau \\
            \Delta \stoup \Gamma \vdash \ov{e : \tau} \\
            \Delta \vdash (\ov{\tau \imp \sigma})
        }
        { \Delta \stoup \Gamma \vdash e.m(\ov{e}) : \sigma }

        \inferrule[t-array-literal]
        {
        \Delta \vdash \tau_A \ok \\
        \ov{\Phi} = \typeparams(t_A)\\
        \tau_A = t_A[\ov{\sigma}] \\
        \eta = (\ov{\Phi \by \sigma})\\
        \sigma = \elementtype(t_A)[\eta] \\
        \len(\tau_A) \imp n_\tau \\
        |\ov{e}| = \len(\tau_A)\\
        \Delta ; \Gamma \vdash \ov{e : \tau} \\
        \Delta \vdash \ov{\tau <: \sigma}
        }
        { \Delta;\Gamma \vdash \tau_A\br{\ov{e}} : \tau_A }

        \inferrule[t-array-index]
        {
        \ov{\Phi} = \typeparams(t_A)\\
        \tau_A = t_A[\ov{\tau}] \\
        \eta = (\ov{\Phi \by \tau})\\
        \tau = \elementtype(t_A)[\eta]\\
        \Delta; \Gamma \vdash e_1 : \tau_A\\
        \Delta; \Gamma \vdash ~e_2 : \kw{int} \\
        }
        { \Delta;\Gamma \vdash e_1[e_2] : \tau }

        \inferrule[t-array-index-literal]
        {
        \ov{\Phi} = \typeparams(t_A)\\
        \tau_A = t_A[\ov{\tau}] \\
        \eta = (\ov{\Phi \by \tau})\\
        \tau = \elementtype(t_A)[\eta]\\
        \Delta; \Gamma \vdash e_1 : \tau_A\\
        \Delta; \Gamma \vdash ~e_2 : n \\
        \len(\tau_A) \imp n_\tau \\
        0 \le n < \len(\tau_A)
        }
        { \Delta;\Gamma \vdash e_1[e_2] : \tau }

        \gray{
            \inferrule[t-struct-literal]
            {
                \Delta \vdash \tau_S \ok
                \quad
                \Delta \stoup \Gamma \vdash \ov{e : \tau}
                \quad
                (\ov{f~\sigma}) = \fields(\tau_S)
                \quad
                \Delta \vdash \ov{\tau \imp \sigma}
            }
            { \Delta \stoup \Gamma \vdash \tau_S\br{\ov{e}} : \tau_S }
        }

        \gray{
            \inferrule[t-field]
            {
                \Delta \stoup \Gamma \vdash e : \tau_S
                \quad
                (\ov{f~\tau}) = \fields(\tau_S)
            }
            { \Delta \stoup \Gamma \vdash e.f_i : \tau_i }
        }
    \end{mathpar}

    Programs  \hfill \fbox{$P \ok$}
    \begin{mathpar}
        \gray{
            \inferrule[t-prog]
            {
                \distinct(\tdecls(\ov{D})\black{, \kw{int}}) \\
                \distinct(\mdecls(\ov{D})) \\
                \ov{D \ok} \\
                \emptyset \stoup \emptyset \vdash e : \tau
            }
            { \package~\main;~\ov{D}~\func~\main()~\br{\un=e} \ok }
        }
    \end{mathpar}

    \caption{FGG typing with arrays (3 of 3)}
\end{figure}


\subsection{FGGA Properties}

As before, an array index expression or an array-set method call expression
panics if they contain an array type $\tau_A$, and an array index $n$, where $n
    \notin \indexbounds(\tau_A)$. An expression $e$ panics if $e = E[d]$, where $E$ is
any evaluation context, and $d$ is an expression that panics.

The progress and preservation properties covered in \emph{Featherweight Go}
apply to FGGA and are defined as follows \autocite{fg}:

\begin{theorem}[Preservation]
    If\/ $\emptyset;\emptyset \vdash d : \sigma$ and $d \becomes e$
    then\/ $\emptyset;\emptyset \vdash e : \tau$ for some $\tau$
    with\/ $\tau \imp \sigma$.
\end{theorem}

\begin{theorem}[Progress]
    If\/ $\emptyset;\emptyset \vdash d:\sigma$ then
    either\/ $d$ is a value,
    $d \becomes e$ for some $e$,
    or\/ $d$ panics.
\end{theorem}
