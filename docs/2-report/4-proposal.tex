\section{Generically sized arrays}
\label{ch:proposal}

This section details a minimal design for the addition of generically sized
arrays and numerical type parameters to Go. We also compare this design against
the existing proposal by \autocite{goArrayProposal} and discuss future
extensions to this feature. Many of the points discussed in this section were
published as a proposal in the Go open source community, as an output of this
work, in order to foster discussion and gather feedback \autocite{myProposal}.

\subsection{Type-set interfaces and the \kw{const} interface}

We've already looked at type set-based interfaces introduced in Go 1.18. If we
treat array lengths as types, then we now have a conceptual set of types 0, 1, 2,
3, etc. Similarly, we can conceptually define the \kw{const} type as the type
set of all array lengths:

\golisting{./examples/reversed/const_type.go}

Where the \texttt{...} means the pattern repeats for all the non-negative
integers. In practice, \kw{const} would be another ``special'' predeclared
interface type, just like the existing \texttt{comparable} that ``denotes the
set of all non-interface types that are strictly comparable'' \autocite{spec}.
Just like \texttt{comparable}, \kw{const} would be an interface that can only be
used as a type parameter constraint, and not e.g. as the type of a variable,
function parameter or return type. This \texttt{comparable} interface is not
defined in terms of regular Go code but rather exists on the level of the
language itself. \kw{const} would follow the same pattern. The choice of the
identifier \kw{const} is to ensure backwards compatibility with existing
programs, as this keyword is currently not allowed to be used as a type name.
Just like a union type set, \kw{const} can be instantiated with one of the
elements of the union, i.e. a non-negative integer literal. This restricts
numerical type arguments to strictly compile-time constant integers. Such a type
parameter could then be used as the size of an array. With such an extension, we
can express the \texttt{reversed} function from the previous section as follows:

\golisting{./examples/reversed/const_generic_type.go}

Note how once again, the body of the function remains unchanged. The only
difference is that \texttt{N} is now a type parameter bound by the
\texttt{const} interface. The above function can operate on any array of any
size and any element type.

The rest of this work looks at the theory and implementation of the \kw{const}
type into the existing Go language. We will examine a language called
Featherweight Generic Go With Arrays (FGGA), which is a subset of Go, modulo the
addition of numerical type parameters. Since FGGA only considers ``classic''
(method set) interfaces, \kw{const} will not be an interface type in FGGA, but
rather in its own category. This category can be thought of as non-method set
interfaces, since the two have the same restrictions, i.e. they can only be used
as type parameter bounds.

\subsection{Comparison with existing proposal}

Shortly after the Type Parameters proposal was published
\autocite{genericsProposal}, a proposal to extend generics to array sizes was
published \autocite{goArrayProposal}. It would allow type set interfaces of the
following form:

\golisting{./examples/werner/array_interface.go}

In this design, the parameterisation of the array size is implicit using
\texttt{...} and does not appear in the list of type parameters, meaning the
numerical type parameter cannot be referenced directly. Instead, the proposal
author suggests using the feature as follows:

\golisting{./examples/werner/matrix.go}

The example can be simplified slightly, by inlining the type set
directly in into the type parameter constraint (allowed as of Go 1.18):

\golisting{./examples/werner/matrix_inline.go}

It could then be instantiated as follows:

\golisting{./examples/werner/matrix_variable.go}

Compare this with the proposal presented in this work. Using the \kw{const}
interface, the analogous code would be as follows:

\golisting{./examples/werner/my_matrix.go}

This approach mandates much less boilerplate than the current proposal, as the
type consumer is not forced to create ``dummy'' type arguments, and the type
provider is not forced to retrieve an implicit numerical parameter through the
\texttt{len} function. Explicit numerical type parameters would make generic
arrays a first-class feature of Go, consistent with the rest of the language.
All the existing compound data types in Go can already be fully type
parameterised (slices: \texttt{[]T}, maps: \texttt{map[K]V} and channels:
\texttt{chan T}), except for arrays, so this work would bridge that gap
(\texttt{[N]T}), without making the feature feel like a workaround. In addition,
explicit numerical type parameters make the code more readable, as the
programmer can immediately see when a type is parameterised on integers.

Not only would arrays become first class, but so would numerical type
parameters. Currently, arrays are the only types that accept a numerical type
parameter, to parameterise the length of an array type. The \kw{const} interface
would allow any type or function to accept a constant integer (or another
\kw{const} bounded type parameter) as a type argument.

The benefit of Werner's proposal is that it uses existing Go syntax:
\texttt{[...]T} can already be used to denote an array's type when constructing
an array literal:

\golisting{./examples/werner/array_literal.go}

where \texttt{myArray} has an inferred type of \texttt{[3]int}. It's worth
noting, however, that this syntax is used for type inference, as opposed to
denote the type of a value, similar to how in some cases type arguments can be
omitted, where the compiler is able to infer what they are.

Another shortcoming of the implicit \texttt{[...]} syntax to parameterise array
sizes, is that it becomes unclear whether two type parameters of the same
constraint \texttt{[...]T} (as shown in figure \ref{fig:werner-ambiguity}) have
the same length. If yes, then how do we express two type parameters of the
different lengths? If not, then how do we express that two type parameters must
have the same length? How about when we use the shorthand syntax to collapse the
bounds of multiple type parameters? Explicit numerical type parameters make this
differentiation trivial, enhancing the readability of the code.

\begin{figure}
    \golisting{./examples/werner/array_pair.go}
    \golisting{./examples/werner/array_pair_collapsed.go}
    \caption{Examples of ambiguity when using \texttt{[...]T} to constrain
        multiple type parameters}
    \label{fig:werner-ambiguity}
\end{figure}

The proposal mentions making \texttt{len} applicable to array \emph{types} in
addition to array values (as seen in Werner's examples presented above),
however, as pointed out in the GitHub
issue\footnote{\url{https://github.com/golang/go/issues/44253\#issuecomment-820999513}},
this is unnecessary as the desired behaviour can already be achieved by applying
\texttt{len} to an instantiated value of an array type parameter.

\subsection{Allowed \texttt{const} type arguments}
\label{sec:allowed-const-type-arguments}

We've already seen how constant non-negative integers can be used as numerical
type arguments. Additionally, since a numerical type parameter stands in for a
constant integer, it can itself be used as a type argument. This is consistent
with how in Go type parameters satisfy their own bounds, and is the basis for
creating nested generic structures. Array type literal length parameters are
generalized to accept a \kw{const} type parameter, as seen in the matrix
example, to fit into this new definition.

How about an expression like \texttt{N + 1} or \texttt{2 * N}? Should we allow
them as \kw{const} type arguments? The question can be phrased as: is an
expression containing a \kw{const} type parameter and constant operations (i.e.
ones that can be computed at compile time, if we substitute the type parameter
for a concrete integer), a constant expression? Going forwards, when the answer
to the above questions is ``no'', it will be referred to as the conservative
approach, whereas a ``yes'' answer will be referred to as the liberal approach.

Constant expressions evaluate to constant integers, of which non-negative ones
can be used as \kw{const} type arguments. This brings us to our first problem,
not all constant expressions yield non-negative integers, and we cannot tell
what the sign of a ``constant'' expression involving a type parameter will be
until the user has instantiated a generic type or function. Go type checks
generic functions/types at the declaration site, rather than the call site, so
we need to ensure our approach fits that model.

Consider the signature of a function that returns the head and tail of an array
\autocite{rustConstBlog}:

\golisting{./examples/expressions/headtail.go}

Since an array length cannot be negative, it is only valid to pass arrays of
size 1 or more to this function. If the argument array's length was 0, then
\texttt{N - 1} would evaluate to \texttt{-1}, which is an invalid array size.
Conceptually, we can think of this constraint as a new interface, a
\emph{subtype} of \kw{const}:

\golisting{./examples/expressions/constplus1.go}

This leads us to at least three potential solutions to the above problem. The
first would be to fail the type checking of such a function (at declaration
site), since the operation is not valid for all instantiations of the \kw{const}
interface. Just as underflow can occur when performing operations that can
decrease the value of a numerical type parameter, overflow could occur if the
instantiated type argument is large enough and a constant expression makes it
overflow (i.e. fail to fit into an \texttt{int} type, whose size is platform
dependant). This is where the first potential solution falls short, since
overflow could occur with expressions such as \texttt{N + 1} or \texttt{2 * N},
those expressions would also not be allowed by same the argument of not being
valid for all instantiations, and in effect, we're back to the conservative
approach.

The other 2 solutions revolve around constraining the type parameter bound more
tightly (as shown in the \texttt{constPlus1} interface code). This could be done
implicitly: the compiler could detect that the operation is only permitted for
numerical type parameter instantiations greater than 0, and implicitly make the
type parameter constraint bounds tighter. I.e. the function still type checks at
the declaration site, however, callers would only be able to pass in non-zero
sized arrays, which can be checked at compile time. Tools such as language
servers could show these tighter constraints to the programmer. The final
solution is to require an explicit tighter constraint, through some sort of new
refinement syntax, e.g.:

\golisting{./examples/expressions/headtailexplicit.go}

which places a lower bound of 1 on the \kw{const} interface using the well-known
slicing notation.

The downside of the liberal model is that it makes the compiler implementation
significantly more complex, since it needs to determine what (if any)
combination of numerical type parameter values/ranges will type-check, for every
possible constant (compile-time) operation (e.g. plus, times, bitwise XOR, type
conversion etc.).
% TODO one or two examples:
% one similar to Axel's using existing operations
% one nested example, i.e. a constrained numerical type parameter is the target
% (e.g. factorisation)
It can be shown that such checks could be used to perform computation at
compile-time, e.g. to solve SAT formulas \autocite{goArraySAT}, which can
undermine Go's promise of fast compile times, if the programmer (accidentally or
deliberately) misuses the type system. As such, in practice, the most reasonable
roadmap to implement numerical type parameters in Go would be to start with the
conservative approach. Then, if its usefulness outweighs the complexity that
would be introduced into a language that strives for simplicity, progress to an
explicit liberal model. Finally, if there is demand for it, consider the
implicit liberal model, as a type inference feature, and have tools such as
language servers show the inferred bounds to the programmer. Due to the
complexity of the liberal model, Rust has also opted for the conservative
approach for the time being \autocite{rustConstBlog}.

Appendix \ref{sec:proposal-appx-bounds} discusses some more complex scenarios
under the liberal approach. Investigating in full detail the feasibility and
safety of the liberal approach is a topic for future work. The following
sections of this work consider the conservative model only.

\subsection{Allowed operations on generic arrays}

By the conservative approach, any operation on generic arrays must type check
for all instantiations of the array. In particular, indexing an array with a
constant is checked at compile time, and since the minimum array length that
needs to be supported is 0, no constant is safe to index into a generic array.
The programmer may wish to move this check to runtime, by first assigning the
constant to a non-constant \kw{int} variable, and then performing the index
operation. The liberal approach could be used to set a lower bound on an array
length, to allow index bound checks on generic arrays to be performed at
compile time.

\begin{figure}
    \golisting{./examples/index/constant.go}
    \golisting{./examples/index/nonconstant.go}
    \golisting{./examples/index/liberal.go}
    \caption{Indexing operations on generic arrays}
\end{figure}

Appendix \ref{sec:proposal-appx-slice} discusses the slicing operation, and
\ref{sec:proposal-appx-len} discusses how the built-in \texttt{len} function can
be made to work with generically sized arrays.

\subsection{Summary of proposal}

Rust's conservative approach avoids many problems and, as such forms a good
starting point for introducing generically sized arrays and numerical type
parameters in Go. The liberal approach is a topic for future work, as it has the
potential to allow for more complex programs to be checked at compile time.

Constant expressions can be used to instantiate a \kw{const} type parameter and
array lengths. Lone \kw{const} type parameters can be used to instantiate
\kw{const} type parameters and array lengths. Additionally \kw{const} type
parameters can be used as non-const \kw{int} type values. More complex
expressions involving a combination of type parameters, constant expressions,
and constant operations (such as \texttt{+} or \texttt{len}), yield a
non-constant \kw{int} value, so they can be assigned to a variable of
non-constant \kw{int} type, but not used to instantiate \kw{const} type
parameters or array lengths.
