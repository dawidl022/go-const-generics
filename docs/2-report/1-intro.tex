\section{Introduction}

Generics have been introduced in the Go programming language following the
theoretical work by \textcite{fg} in \emph{Featherweight Go}. The \emph{Type
      Parameters Proposal} lists several generic programming constructs not supported
by the initial implementation of generics (as of Go 1.18). Among them is ``no
parameterization on non-type values such as constants.''
\autocite{genericsProposal} The most notable use case of such type parameters
would be for arrays. In Go, the size of an array is part of its type
\autocite{spec}. As such, if the programmer wishes to write a function that
operates on arrays or a data type that contains arrays, it is necessary to
hard-code the size of the operated/contained array. This imposes a limitation on
what abstraction may be introduced where arrays are concerned. Extending the
generic type system in Go to support constants as type parameters aims to
resolve this issue.

\subsection{Background}

Arrays are a primitive data structure found in many programming languages.
However, various languages treat arrays differently. In Java, arrays are
objects, and variables of array type are references to those objects. The size
(or length) of an array is not part of its type. However, it is a property of the
array object instance and cannot be changed after initialisation of the instance
\autocite{javaSpec}. C\# treats arrays analogously \autocite{cSharpArrays}. In
languages like Go and Rust, arrays are value types, and the size of the array
\emph{is} part is of its type \autocites{spec}{rustSpec}. In C, the size of an
array is also part of its type. However, expressions of array type are converted
to pointers to the first element of the array \autocite{cSpec}. Since
dynamically typed languages do not have types associated with variables, those
languages will not be discussed here.

% TODO explore more languages

% TODO talk about first-class vs non first-class arrays, static and only dynamic
% arrays, and whether the size is part of the arrays type or not, and which
% languages are similar to what Go does. Mention that Go has slices for most
% practical use cases. Mention however, the use cases where arrays can come
% handy in Go.

% TODO explain how this problem is tackled in other languages

% C: Arrays are just pointers to the first element (no abstraction)
% "The C programming language"; https://stefansf.de/post/arrays-are-second-class-citizens/

\subsection{Problem Statement}

Arrays in the Go programming language have a number of use cases and situational
benefits over slices --- their dynamic counterparts. One practical scenario is
when wanting to use a collection as a map key. Since the comparison operators
are fully defined for arrays, they may be used as map keys, which is a very
common data structure used in Go programs. The comparison operators are not
fully defined for slices. Hence, they cannot be used as map keys in the same way.
\autocite{spec}. Arrays may also be useful when value semantics are desired,
i.e. assigning an array to another variable or passing it as an argument to a
function makes a copy. For slices, which hold references to underlying arrays,
copies need to be performed manually, which can be more verbose and error-prone.
Since the size of an array is part of its type, it can also serve as
documentation to the reader of the code.

In current day Go (version 1.21 as of the time of writing), there is no way of
abstracting over arrays of any size, e.g. it is impossible to define a function
that operates on an array of any size. While workarounds exist that make use of
type set interfaces introduced in the \emph{Type Parameters Proposal}
\autocite{genericsProposal}, this solution is not very elegant since it requires
manually enumerating all the array sizes the function can operate on, and also
cannot be used in functions exposed as part of a library, as there is no way of
knowing what array size the library consumer will use.

This work aims to address this problem by introducing constant (integer) type
parameters that may be used to define generically-sized array types. A proposal
exists to allow for generic parameterisation of array sizes in Go, demonstrating
the demand for this feature \autocite{goArrayProposal}. However, the design
proposed in this work differs from the existing proposal.

\subsection{Aim}

The aim is to produce a set of formal rules specifying the syntax, reduction
semantics and type system of a subset of the Go programming language, extended
with support for generically sized arrays. The rules are to be verified by
implementing an interpreter that includes the new language feature, as well
as a monomorphiser that translates the extended language into regular Go.

As an extension to the project, a formal proof of correctness of the language
extension may be carried out. Another extension would be to implement the
proposed design into the mainstream Go compiler.

\subsection{Objectives}

\begin{itemize}
      \item Investigate how the problem of abstracting over arrays of any size
            has been solved in other statically-typed programming languages, and
            what research has been done in this area.
      \item Formalise the syntax, reduction and typing rules of arrays in Go,
            based on \emph{Featherweight Go}, referred to as FGA going forward
            \autocite{fg}.
      \item Design and formalise the syntax, reduction and typing rules of
            generically sized arrays, based on \emph{Featherweight Generic Go},
            updated with the existing state of generics in Go, referred to as
            FGGA going forward \autocite{fg}.
      \item Implement and test a type checker and interpreter for FGA extended
            with support for arrays.
      \item Implement and test a type checker and interpreter for FGGA extended
            with support for generically sized arrays.
      \item Implement and test a monomorphiser that translates code written in
            the extended FGGA into regular Go code.
      \item Submit a proposal for adding the language extension to the open
            source community, addressing the challenges previously discussed by
            the community, and address any feedback for my design.
\end{itemize}

\subsection{Research Questions}

\begin{itemize}
      \item How have other statically typed languages where the size of an array
            is part of its type tackled this issue, if at all?
      \item How can the Go type system be extended to support generically sized
            arrays, compatible with the existing implementation of generics in
            Go?
      \item How can the proposed design be verified to be correct?
\end{itemize}


\subsection{Report Structure}

The \ordinaltoname{\getrefnumber{ch:background-research}} chapter explores how
other languages have tackled the problem of abstracting over arrays of any size,
as well as past research in the area of programming languages, in particular
relating to the Go programming language. The
\ordinaltoname{\getrefnumber{ch:examples}} chapter presents a rationale for
introducing generically sized arrays in Go through examples. The
\ordinaltoname{\getrefnumber{ch:proposal}} chapter goes into detail about the
proposed language extension, and explores certain design considerations. Chapter
\numtoname{\getrefnumber{ch:fg}} introduces the formal rules for ``Featherweight
Go with Arrays'' --- a subset of Go containing arrays, based on the paper goingź
by the name ``Featherweight Go'' \autocite{fg}. Chapter
\numtoname{\getrefnumber{ch:fgg}} extends the rules with generics as found in
today's version of Go and the proposed design for generically sized arrays.
Chapter \numtoname{\getrefnumber{ch:monomo}} formally presents a way of
translating FGGA to regular Go via monomorphisation. Finally, the
\ordinaltoname{\getrefnumber{ch:interpter-impl}} chapter gives an overview of
the implementation of two interpreters and a monomorphiser, corresponding
to the rules found in chapters \numtoname{\getrefnumber{ch:fg}},
\numtoname{\getrefnumber{ch:fgg}} and \numtoname{\getrefnumber{ch:monomo}}.
